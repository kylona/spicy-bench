\section{Related Work}
There is an existing extension for JPF for the X10 Language
\cite{conf:icst:GligoricMM12,x10}. Habanero is closely related to X10 in many of its constructs. In the extension, JPF operates directly on the actual
X10 runtime system. To accomplish the integration, JPF is modified, the X10
runtime is modified, the X10 compiler is extended, and a new static
analysis is presented to help control state explosion. The extension
represents a significant effort that affects all aspects of the X10
framework to enable JPF verification. 

\begin{comment}
The approach in this paper avoids such an
invasive approach by creating a new HJ runtime just for
verification. The X10 extension  does leverage the low overhead of threads
in JPF to create a thread for each task. The VR runtime system uses
the same approach.
\end{comment}

There is a formal model for the Chapel language with an accompanying
model checker that employs symbolic execution \cite{chapel}. The
formal model is an intermediate representation (IR) suitable for
concurrent constructs. The approach compiles Chapel programs into the
IR and the model checker then verifies the IR for deadlock and
data-race freedom. Creating a compiler and model checker is a
significant undertaking beyond the approach in this paper. More
critically, the verification tool models the runtime including the
number of available worker threads to service tasks; thus, the
verification results are dependent on the number of worker threads in
the configuration rather than the semantics of the Chapel language. 

\begin{comment}
In this paper, correctness is a property of the HJ language semantics
with the given HJ program and not any aspect of the runtime
implementation. Verifying the HJ runtime system implements HJ semantics
is a verification problem separate from verifying that an HJ program
is data-race free.
\end{comment}

Another approach to verifying concurrent languages is to leverage the
production level language runtime system itself
\cite{Vakkalanka:2008:DVM:1427782.1427794,Vo:2009:FVP:1594835.1504214,5644885,6113841}. These
approaches typically require instrumentation of the source program,
wrappers to intercept calls into the runtime, and a way to control
runtime behavior. Although they are typically able to generate states
faster than JPF, verification results are dependent on the employed
runtime correctly implementing the language semantics. 

\begin{comment}
Such a
dependence can be avoided, as in this work, by creating a
non-performance oriented runtime that is simple enough to manually
verify for correctness \cite{Morse:2012:MAM:2189257.2189279}. It is
much easier to access the state and direct behavior, as a model
checker, in such systems.
\end{comment}

Recent work proves the problem of state-reachability to be decidable
and EXPSPACE hard for finite-valued programs in languages such as
X10/Habanero \cite{Bouajjani:2012:ARP:2103621.2103681}. The result is
limited to a subset of the powerful task constructs in such languages
and justifies a model checking effort. The computability and
complexity of the more advanced constructs such as phasers is yet to
be determined.
