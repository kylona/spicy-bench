\section{Discussion}
\label{sec:discussion}

\subsection{Comparison to other languages}

Our analysis operates on a language that is similar to Habanero Java,
which is itself based on X10. Unlike our language, Habanero Java
distinguishes between futures and asynchronous procedures. The effect
is that every wait on a region waits for everything in that region. On
the other hand, Habanero Java allows asynchronous procedures' return
value handlers to have side effects, which can create data races.

Cilk has the same ability to post asynchronous threads and wait on all
of them to terminate with the same nondeterminism in the order that
threads join. In Cilk, return value handlers are called inlets and can
combine with nondeterministic thread joining order to create data
races. Cilk++ and Cilk Plus lack inlets.

Multilisp also includes futures. Its pcall mechanism is equivalent to
posting the evaluation of each argument and then waiting for them to
all return.

It is possible to identify data races in side-effecting return value
handlers by enumerating all possible orderings. However, the number of
orderings is the factorial of the number of return value handlers.

One alternative approach to this problem is the replacement that
Cilk++ and Cilk Plus use; instead of allowing return value handlers at
all, they use associative operations to combine results from their
child tasks. Enforcing such a restriction on return value handlers
instead of eliminating them entirely allows programmers to handle
most tasks naturally with similarly strong results.

Reactive systems, such as ReactJS, keep a collection of tasks to
execute. At the end of the main function, the eventloop keyword is
reached and parallel tasks execute atomically in a nondeterministic
order. Each task may post additional tasks to the work queue. Our
language could model reactive systems directly with a nondeterministic
\post\ statement. Because of the structured nature of the parallelism
in such languages, it is possible to create a computation graph with
the main function preceding each task and with each task running
without order with respect to its peers. Tasks that generate others
(parents) must execute before their children do. However, many
reactive systems are designed to run indefinitely rather than to
terminate, which complicates the application of the computation graph
model.

\subsection{Side effects in return value handlers}
\label{sec:side-effects-rvhs}

Return value handlers execute in the context of the waiting thread.
When a thread waits on a region with multiple tasks, the order in
which they join is nondeterministic. This creates a situation where
nondeterminism occurs within a single thread. As such, it is possible
to introduce data races on thread-local variables if the return value
handlers can create side effects.

Naturally, it is possible to expand the definitions of \rv\ and \wv\
so that they consider all variables and not just shared variables. In
order to distinguish between identically-named variables in different
tasks, \rv\ and \wv\ would need to take the task ID into account. This
extension would make the language and analysis more general but could
make the analysis much slower.
