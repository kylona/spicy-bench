%% For double-blind review submission, w/o CCS and ACM Reference (max submission space)
\documentclass[sigplan,10pt,review,anonymous,table]{acmart}\settopmatter{printfolios=true,printccs=false,printacmref=false}
%% For double-blind review submission, w/ CCS and ACM Reference
%\documentclass[sigplan,10pt,review,anonymous]{acmart}\settopmatter{printfolios=true}
%% For single-blind review submission, w/o CCS and ACM Reference (max submission space)
%\documentclass[sigplan,10pt,review]{acmart}\settopmatter{printfolios=true,printccs=false,printacmref=false}
%% For single-blind review submission, w/ CCS and ACM Reference
%\documentclass[sigplan,10pt,review]{acmart}\settopmatter{printfolios=true}
%% For final camera-ready submission, w/ required CCS and ACM Reference
%\documentclass[sigplan,10pt]{acmart}\settopmatter{}


%% Conference information
%% Supplied to authors by publisher for camera-ready submission;
%% use defaults for review submission.
\acmConference[PPoPP'18]{ACM SIGPLAN Conference on Principles and Practices of Parallel Programming}{February 24--28, 2018}{V\"{o}sendorf, Austria}
\acmYear{2018}
\acmISBN{} % \acmISBN{978-x-xxxx-xxxx-x/YY/MM}
\acmDOI{} % \acmDOI{10.1145/nnnnnnn.nnnnnnn}
\startPage{1}

%% Copyright information
%% Supplied to authors (based on authors' rights management selection;
%% see authors.acm.org) by publisher for camera-ready submission;
%% use 'none' for review submission.
\setcopyright{none}
%\setcopyright{acmcopyright}
%\setcopyright{acmlicensed}
%\setcopyright{rightsretained}
%\copyrightyear{2017}           %% If different from \acmYear

%% Bibliography style
\bibliographystyle{ACM-Reference-Format}
%% Citation style
%\citestyle{acmauthoryear}  %% For author/year citations
%\citestyle{acmnumeric}     %% For numeric citations
%\setcitestyle{nosort}      %% With 'acmnumeric', to disable automatic
                            %% sorting of references within a single citation;
%% e.g., \cite{Smith99,Carpenter05,Baker12}
                            %% rendered as [14,5,2] rather than [2,5,14].
%\setcitesyle{nocompress}   %% With 'acmnumeric', to disable automatic
                            %% compression of sequential references within a
                            %% single citation;
                            %% e.g., \cite{Baker12,Baker14,Baker16}
                            %% rendered as [2,3,4] rather than [2-4].


%%%%%%%%%%%%%%%%%%%%%%%%%%%%%%%%%%%%%%%%%%%%%%%%%%%%%%%%%%%%%%%%%%%%%%
%% Note: Authors migrating a paper from traditional SIGPLAN
%% proceedings format to PACMPL format must update the
%% '\documentclass' and topmatter commands above; see
%% 'acmart-pacmpl-template.tex'.
%%%%%%%%%%%%%%%%%%%%%%%%%%%%%%%%%%%%%%%%%%%%%%%%%%%%%%%%%%%%%%%%%%%%%%


%% Some recommended packages.
\usepackage{booktabs}   %% For formal tables:
                        %% http://ctan.org/pkg/booktabs
\usepackage{subcaption} %% For complex figures with subfigures/subcaptions
                        %% http://ctan.org/pkg/subcaption


%% BEGIN: Packages and definitions for Radha's paper to build
% Packages needed for Radha's paper
\usepackage{paralist,comment,algorithm,algpseudocode,mathpartir,listings,graphicx,epstopdf,multicol}

\definecolor{dkgreen}{rgb}{0,0.6,0}
%\definecolor{gray}{rgb}{0.5,0.5,0.5}
\definecolor{mauve}{rgb}{0.58,0,0.82}
\definecolor{light-gray}{gray}{0.88}

\lstset{frame=tb,
  language=Java,
  aboveskip=3mm,
  belowskip=3mm,
  showstringspaces=false,
  columns=flexible,
  numbers=none,
  numberstyle=\tiny\color{gray},
  commentstyle=\color{dkgreen},
  stringstyle=\color{mauve},
  breaklines=true,
  breakatwhitespace=true,
  tabsize=3,
  morekeywords={proc, post, await, ewait, var, skip, assume, call, return, async, finish, future, isolated, forall}
}

% Commands needed for Radha's paper
\newcommand{\figref}[1]{Fig.~\ref{#1}}
\newcommand{\lemmaref}[1]{Lem.~\ref{#1}}
\newcommand{\tableref}[1]{Table~\ref{#1}}
\newcommand{\secref}[1]{Sec.~\ref{#1}}
\newcommand{\defref}[1]{Def.~\ref{#1}}
\newcommand{\lineref}[1]{Line~\ref{#1}}
\newcommand{\corref}[1]{Cor.~\ref{#1}}
\newcommand{\thmref}[1]{Theorem~\ref{#1}}
\newcommand{\algoref}[1]{Algo.~\ref{#1}}

\newcommand{\setof}[1]{\ensuremath{\left \{ #1 \right \}}}
\newcommand{\tuple}[1]{\ensuremath{\langle #1 \rangle }}

\usepackage{marginnote}
\newcommand{\marginX}{\marginnote{{\quad\textbf{!}\quad}}}
\newcommand{\egm}[1]{\mbox{}{\color{pink!70!black}{\marginX{}[\textbf{Eric}: #1]}}}
\newcommand{\peter}[1]{\mbox{}{\color{pink!70!black}{\marginX{}[\textbf{Peter}: #1]}}}

\input{defs}
%% END: Packages and definitions for Radha's paper to build

\begin{document}

%% Title information

\title[Model-checking Task Parallel Programs]{Model-checking Task Parallel Programs for Data-races using Computation Graphs}
%\titlenote{This research is funded by NSF Grant 1302524}%% \titlenote is optional;
                                        %% can be repeated if necessary;
                                        %% contents suppressed with 'anonymous'
%\subtitle{Subtitle}                     %% \subtitle is optional


%\subtitlenote{with subtitle note}       %% \subtitlenote is optional;
                                        %% can be repeated if necessary;
                                        %% contents suppressed with 'anonymous'


%% Author information
%% Contents and number of authors suppressed with 'anonymous'.
%% Each author should be introduced by \author, followed by
%% \authornote (optional), \orcid (optional), \affiliation, and
%% \email.
%% An author may have multiple affiliations and/or emails; repeat the
%% appropriate command.
%% Many elements are not rendered, but should be provided for metadata
%% extraction tools.

\author{Radha Nakade and Eric Mercer and Peter Aldous}
%\authornote{}
\orcid{}
\affiliation{%
  \institution{Brigham Young University}
  \streetaddress{}
  \city{Provo} 
  \state{UT} 
  \postcode{}
}
\email{radha,egm,aldous@cs.byu.edu}

\author{Jay McCarthy}
%\authornote{}
\affiliation{%
  \institution{University of Massachusetts Lowell}
  \streetaddress{}
  \city{Lowell} 
  \state{Massachusetts} 
  \postcode{}
}
\email{jay.mccarthy@gmail.com}

% The default list of authors is too long for headers}
\renewcommand{\shortauthors}{R. Nakade, E. Mercer, P. Aldous, and J. McCarthy}

%% Paper note
%% The \thanks command may be used to create a "paper note" ---
%% similar to a title note or an author note, but not explicitly
%% associated with a particular element.  It will appear immediately
%% above the permission/copyright statement.
%\thanks{with paper note}                %% \thanks is optional
                                        %% can be repeated if necesary
                                        %% contents suppressed with 'anonymous'


%% Abstract
%% Note: \begin{abstract}...\end{abstract} environment must come
%% before \maketitle command
\begin{abstract}
Data-race detection is the problem of determining if a concurrent program has a data-race in some execution and input; it has been long studied and often solved for different contexts and goals. The research in this paper reprises the problem of data-race detection but in the context of model checking as opposed to run-time monitoring or static analysis. The programming model is task parallel where computations are hierarchically divided into non-isolated parallel executing tasks and is suitable to describing real-world languages (i.e., Cilk, X10, Chapel, Habanero, etc.). The model semantics are defined to construct a computation graph and a naive algorithm to detect data-race on a computation graph is given. A fragment of the programming model is defined such that the algorithm becomes sound and complete for computation graphs from a single observed program execution. Model checking is applied to programs outside this fragment to enumerate all possible computation graphs. The approach is evaluated in a Java implementation of Habanero using the JavaPathfinder model checker. The results, when compared to existing data-race detectors in Java Pathfinder, show a significant reduction in the time required for data race detection.

\begin{comment}
Task parallel programming languages provide a way for creating asynchronous tasks that can run concurrently. The advantage of using task parallelism is that the programmer can write code that is independent of the underlying hardware. The runtime determines the number of processor cores that are available and the most efficient way to execute the tasks. When two or more concurrently executing tasks access a shared memory location and if at least one of the accesses is for writing, data race is observed in the program. Data races can introduce non-determinism in the program output making it important to have data race detection tools. To detect data races in task parallel programs, a new sound and complete technique based on computation graphs is presented in this work. The data race detection algorithm runs in $\mathcal{O}$(N$^2$) time where N is number of nodes in the graph. A computation graph is a directed acyclic graph that represents the execution of the program. For detecting data races, the computation graph stores shared heap locations accessed by the tasks. An algorithm for creating computation graphs augmented with memory locations accessed by the tasks is also described here. This algorithm runs in $\mathcal{O}$(N) time where N is the number of operations performed in the tasks. A scheduling algorithm that creates all possible computation graph structures for programs containing critical sections is also presented here. This work also presents an implementation of this technique for the Java implementation of the Habanero programming model. The results of this data race detector are compared to Java Pathfinder's precise race detector extension and permission regions based race detector extension. The results show a significant reduction in the time required for data race detection using this technique.

Cut from current abstract

A data-race is where two concurrent executions access the same memory location with at least one of the two accesses being a write. A data-race introduces non-determinism in program execution complicating both test and debug since the programmer is not able to directly control or observe the internals of the the execution run-time. The computation graph captures the partial-order between tasks with memory references. 
\end{comment}

\end{abstract}

%
% The code below should be generated by the tool at
% http://dl.acm.org/ccs.cfm
% Please copy and paste the code instead of the example below. 
%
\begin{CCSXML}
<ccs2012>
<concept>
<concept_id>10011007.10010940.10010992.10010998</concept_id>
<concept_desc>Software and its engineering~Formal methods</concept_desc>
<concept_significance>500</concept_significance>
</concept>
</ccs2012>
\end{CCSXML}

\ccsdesc[500]{Software and its engineering~Formal methods}
%% End of generated code

% We no longer use \terms command
%\terms{Theory}
\keywords{Task Parallel Languages, Data-race, Model Checking, Computation Graphs}


%% \maketitle
%% Note: \maketitle command must come after title commands, author
%% commands, abstract environment, Computing Classification System
%% environment and commands, and keywords command.
\maketitle

\section{Introduction}
A \emph{data-race} is where two concurrent executions access the same memory location with at least one of the two accesses being a write. It introduces non-determinism into the program execution as the behavior may depend on the order in which the concurrent executions access memory. Data-race is problematic because it is not possible to directly control or observe the run-time internals to know if a data-race exists let alone enumerate program behaviors when one does. 

The problem of \emph{Data-race detection}, given a program with its input, is to determine if there exists an execution containing a data-race. The research presented in this paper is concerned with data-race detection for \emph{task parallel models} that impose structure on parallelism by constraining how threads are created and joined and how shared memory is accessed (e.g., Cilk, X10, Chapel, Habanero, etc.). These models rely on run-time environments to implement task abstractions to represent concurrent executions \cite{blumofe1996cilk,charles2005x10,cave2011habanero,imam2014habanero}. The language restrictions on parallelism and shared memory interactions enable properties like \emph{determinism} (i.e., the computation is independent of the execution) or the ability to \emph{serialize} (i.e., removing all task related keywords yields a serial solution). Such properties are predicated on the input programs being data-race free which is not always the case since programmers both intentionally and unintentionally move outside the programming model.

Data-race detection in task parallel models generally prioritizes performance and the ability to scale to many tasks. The predominant \emph{SP-bags} algorithm, with its variants, exploits assumptions on task creation and joining for efficient on-the-fly detection with low overhead \cite{Feng1997EDD258492258493,Cheng1998DDR277651277696,Bender2004OMS10079121007933,Async-Finish-Race,Utterback2016PGP29357642935801}; millions of task are feasible with varying degrees of slow-down (i.e., slow-down increases as parallelism constraints are relaxed) \cite{drdForFutures,Surendran2016}. Other approaches use access histories \cite{mellor1991fly,raman2012scalable} or programmer annotations \cite{westbrook2012practical, westbrook2012permission}.
Performance is a priority, and many solutions are only \emph{complete}, meaning that nothing can be concluded about other executions of the same program, or become complete when parallelism is outside the assumptions.

The research presented in this paper reprises data-race detection in task parallel models in the context of model checking under two assumptions: first, data-race is largely independent of the size of the problem instance; and second, it is possible to instantiate small problem instances. These assumptions are indeed implicit in other model checking solutions for task parallel models---a small problem instance is the best solution to state explosion \cite{gligoric2012x10x,zirkel2013automated,anderson2014jpf}. Prior approaches, however, extensively modify the language run-time and the model checker necessitating source code for both \cite{gligoric2012x10x}, require the user to specify the number of processors modeled making it difficult to generalize \cite{zirkel2013automated}, or rely on user annotations to indicate sharing so data-race may be missed \cite{anderson2014jpf}. The solution here is to make clear the requirements on the run-time, use semantics that are independent of actual hardware, and automatically detect data-race without annotations. 

The approach first defines a \emph{computation graph} to abstractly model parallelism with a naive algorithm to detect data-race. Computation graph construction is then formally defined in a general task parallel model based on partitioning concurrent executions into hierarchical regions with shared locations. Such a model is suitable to describe real-world languages (e.g., Cilk, X10, Chapel, Habanero, etc.).  A fragment of the model is then defined so that data-race detection on a computation graph from a single execution is both sound and complete; this result is similar to existing dynamic analyses for task parallel languages. That fragment is then expanded to show how model checking may be applied to enumerate the space of computation graphs for data-race detection. Finally, the approach is evaluated on a Java implementation of Habenero with Java Pathfinder (JPF). Results over several published benchmarks comparing to JPF's default race detection and a task parallel approach with permission regions show the computation graph to be more efficient in JPF terms with its overhead. The primary contributions are thus
\begin{compactitem}
\item the computation graph construction in terms of general semantics suitable for real-world languages;
\item model checking as a means to exploring the space of computation graphs for a program; and
\item an implementation of the approach for Java Habanero in JPF with results from benchmarks comparing to other solutions in JPF. 
\end{compactitem}
\begin{comment}
Section \ref{sec:drd} defines computation graphs and data-race detection given a computation graph. Section \ref{sec:cg} is the programming model with graph construction. Section \ref{sec:otf-drd} is the model checking algorithm. Section \ref{sec:impl} gives an implementation of the algorithm for Habanero and section \ref{sec:res} discusses the results. Section \ref{sec:rel-work} discusses related work. Section \ref{sec:conclusion} presents the conclusion.
\end{comment}
\begin{comment}
  The increasing use of multi-core processors is motivating parallel programming. Earlier, the speed of processor cores was expected to increase with sustained technological advances and the need for parallel computing was low. Now that processor speeds are no longer increasing, parallelism is the only way of obtaining higher computing performance.

Writing concurrent programs that are free from bugs, however, is very difficult because when programs execute different instructions simultaneously, different thread schedules and memory access patterns are observed that give rise to issues such as data races and deadlocks. Structured parallel languages help users to write parallel programs that are scalable and easy to maintain \cite{blumofe1996cilk, charles2005x10, cave2011habanero}. These properties are achieved by imposing restrictions on the way tasks can be forked and joined.

Data races occur in parallel programs when two or more tasks access a shared memory location such that at least one of the accesses is a write. A race on a shared variable can alter the value of the variable based on the order in which the variable is accessed by the tasks causing the output to be non-deterministic. A data race that is not protected (i.e., marked volatile) also leads to behavior that is not sequentially consistent. It is hard to test all possible outcomes of the program with a data race because the scheduler most often does the same thing. 

A lot of research has gone into the problem of detecting data races in parallel programs. Data race detection techniques are mainly categorized as static, dynamic and model checking. Static race detectors analyze the programs statically and report errors \cite{engler2003racerx,ESC,abadi2006types,naik2006effective,voung2007relay,choi2001static, vechev2011automatic}. Their drawback is that they report data races on variables when in fact there are no data-races; this imprecision makes them hard to use in real world applications. Model checking on the other hand produces precise results but suffers from state space explosion making it impossible to use in large systems\cite{kulikov2010detecting, vakkalanka2008implementing, Godefroid, anderson2014jpf, gligoric2012x10x, zirkel2013automated}.

Dynamic data race detectors analyze the program at runtime and so the data races reported by them are real data races. Dynamic data race detectors however can reason about only a single run \cite{flanagan2009fasttrack, savage1997eraser, mellor1991fly, schonberg1989fly, Feng97efficientdetection, Async-Finish-Race}. Raman et al. created a dynamic race detector for structured parallel programs that can locate races in any schedule of the program by running the program only once using limited access history\cite{raman2012scalable}. Another approach for data race detection for programs with futures uses dynamic task reachibility graphs \cite{drdForFutures}. Both these approaches are efficient, however, they don't provide any functionality to manipulate the scheduler at runtime thereby being unsound for programs with mutual exclusion. When accesses to shared variables are protected, different program outcomes are observed. It is necessary to analyze all program behaviors to ensure data race freedom.

This paper introduces an improved technique for data race detection that combines dynamic race detection for structured parallel languages with model checking to overcome the limitations of both of them. This technique makes use of computation graphs to represent the happens-before relation of the events of the program in the form of a directed acyclic graph \cite{dennis2012determinacy}. The nodes represent the various tasks that are spawned during the program execution and store the references to shared heap locations that have been accessed by those tasks. To detect data races, the task nodes that can execute in parallel are identified in the graph and the memory locations stored in these nodes are compared to detect conflicts. For building such computation graphs, the runtime should have the ability to call-back when threads are forked or joined, and to record memory accesses on heap locations that may be shared.

The model checking part of the solution comes into play for programs with critical sections. In such programs, different computation graph structures can arise based on the order of execution of the critical sections. The technique presented here creates all such computation graphs using a scheduler that checks for critical sections and builds schedules to consider all possible ways in which the program can execute \cite{mercer2015model}. Hence, this method is sound for all programs with a given input.  

This paper presents an implementation of this data race detection technique for the Java implementation of the Habanero programming model. The implementation uses Java Pathfinder (JPF). JPF is a model checker for Java that is fully customizable using various programming patterns and interfaces. The implementation uses JPF's virtual machine to create a specialized runtime for the Habanero language that is targeted specifically for verification\cite{mercer2015model, anderson2014jpf}. 
The performance is compared with two other model checking approaches implemented by JPF: Precise Race Detector and Permission regions \cite{kulikov2010detecting}, \cite{mercer2015model}. The results show a significant reduction in the state space and time needed for verification.

\textbf{Main Contributions:}
  \vspace{-1em}
\begin{enumerate}
\item A data race detection algorithm using computation graphs that runs in $\mathcal{O}$(N$^2$) time where N is number of nodes in the graph.
\item Semantics for task parallel programs that include steps for creating computation graphs.
\item Dynamic improvement to the data race detection algorithm for structured parallel programs.
\item A scheduling algorithm to create all computation graphs for programs containing mutual exclusion.
\item An implementation of the data race detection algorithm for Habanero Java.
\item An empirical study over a set of benchmarks comparing performance of the data race detection algorithm to JPF.
\end{enumerate}
Some proofs are omitted for space but exist in a long technical report to be referenced if the paper is accepted.
\end{comment}

\begin{comment}
  The rest of the paper is divided as follows. Section \ref{sec:drd} introduces the concept of computation graphs for task parallel programs and discusses the data race detection algorithm based on computation graphs. Section \ref{sec:cg} presents syntax and semantics of task parallel languages and discusses the creation of computation graphs. Section \ref{sec:otf-drd} discusses dynamic improvement to the data race detection algorithm using on-the-fly analysis for structured parallel programs and a scheduler for programs with critical sections. Section \ref{sec:impl} gives implementation of the algorithm for HJ and section \ref{sec:res} discusses the results. Section \ref{sec:rel-work} discusses related work. Section \ref{sec:conclusion} presents the conclusion.
\end{comment}

New Outline for PPoPP 2018 Submission:
\begin{enumerate}
\item Contribution and Example showing our approach
\item Formal definition of Computation Graph with Language semantics (including isolation)
  \begin{enumerate}
  \item Proof that computation graph is correct from semantics
  \item Algorithm to detect data race on computation graph with its complexity
  \end{enumerate}
\item Model Checking with Proof
\item Deterministic restriction with Proof
\item Results
\item Discussion
\item Related Work
\end{enumerate}

\section{Example}
The approach to data-race detection in this paper is presented in a very simple example.
Consider the task-parallel program in \figref{fig:example}.
The language used is defined in this paper with a formal semantics to facilitate proofs but has a direct expression in most task parallel languages.
For example, \figref{fig:example-hj} is the equivalent program in the Habanero Java language.

For \figref{fig:example}, execution begins with the procedure \texttt{m}.
The variable \texttt{g} is global.
The \textbf{post}-statement creates a new asynchronous task running procedure \texttt{p} passing 0 for its parameter.
The task handled is stored in the region $r$, also global, and when that task completes and joins with its parent \texttt{m}, it runs the $\lambda$-expression as a return value handler.
In this case, that handler is the no-op \texttt{skip}.

The \textbf{isolated}-statement runs the statements in its scope in mutual exclusion to other \textbf{isolated}-statements.
The \textbf{await}-statement joins all tasks in region \texttt{r} with the task that issued the await.
The issuer may join with a task in the region if that task is at its \textbf{return}-statement.
The expression in the return statement is evaluated at the join and the value is passed to the return value handler in the parent context.
The parent blocks at the \textbf{await}-statement until it has joined with all tasks in the indicated region.

The program in \figref{fig:example} has a schedule dependent data race.
If the scheduler runs the \textbf{isolated}-statement in procedure \texttt{p} before the \textbf{isolated}-statement in procedure \texttt{m} then there is a write-write data-race; otherwise, there is no data-race. 

Related work instruments the program, modifies the runtime environment, or both to track memory references and to build a happens-before relation from the execution to detect data-races on-the-fly.
The data-race detection only reasons about the current input and execution and cannot be generalized to other input and executions.

The approach in this paper, like other solutions, fixes the program input, tracks memory references, and builds a happens-before relation.
Unlike other solutions though, it uses a verification specific runtime with no programmer annotations, and it uses model checking to reason over all
interesting schedules to prove a program data-race free on the given input for any schedule.

The verification runtime stores memory reference and the happens-before relation in the form or a computation graph.
The left part of \figref{fig:example-cg} shows the computation graph for the data-race free schedule of the simple example program.
Every node represents a block of sequential operations and edges order the nodes.
The thick \texttt{p}-labeled line is the result of the \textbf{post}-statement creating a new task, and the dashed boxes are the \textbf{isolated}-statements. 
Intuitively, the computation graph is a Hasse diagram with inverted edges---things at the bottom happen-after things at the top---and with extra information on each node to indicate read and write memory locations.
Such a graph can be analyzed for data-race in $|\glbls| * O(|N|^2)$ time where $|\glbls|$ is the number of shared memory locations and $|N|$ is the number of nodes in the computation graph. The formal semantics in the paper give the computation graph construction and prove it captures the memory references and happens-before relation.

To reason over all schedules, the approach in this paper first assumes two restrictions common in most task parallel languages: if a return value handler side-effects, then it is posted in a region by itself, and all tasks are joined at termination in a deterministic order by a implicit enclosing parent task.
Under these restrictions, the model checker, to prove data-race freedom, must generate a set of schedules that contains all ways allowed by the program semantics to interleave \textbf{isolated}-statements.\footnote{Only schedules around dependent \textbf{isolated}-statements need be considered for a further reduction but requires a full partial order reduction with a suitable definition for dependent relative to \textbf{isolated}-statements.}
This result is the main contribution.

The left part of \figref{fig:example-cg} shows the  computation graph for  the data-race schedule of the simple example program. 
Although, the two schedules in \figref{fig:example-cg} are the only schedules that need to be considered by the model checker, the number of interesting schedules grows exponentially in the number of concurrent dependent \textbf{isolated}-statements.
The growth limits the model checking approach in this paper to programs that can be instantiated on small problem instances;
however, in general, a proof on a small problem instance typically generalizes to large problem instances. This generalization is discussed in \secref{sec:res}.

\begin{figure}
  \begin{center}
    \begin{lstlisting}[mathescape=true]
proc m (var x : int)
   g := 0;
   post r $\leftarrow$ p 0 $\lamdefe{v}{\mathtt{skip}}$;
   [ isolated g := 1 ]
   await r
   return x
    
proc p (var x : int)
   [ isolated skip; ]
   g := 2;
   return 0
\end{lstlisting}
  \end{center}
  \caption{A simple example of a task parallel program with a data-race depending on the execution schedule.}
  \label{fig:example}
\end{figure}


\begin{figure}
  \begin{center}
\begin{lstlisting}
public class Example1{
   static int g = 0;
   public static void main(String[] args) {
     g := 0
     finish {
        async { p(0); }
        isolated{ g := 1; }
     }
     public static void p(int x) {
         isolated{ /* skip */ };
         g := 2;
         return 0;
     }
}
\end{lstlisting}
  \end{center}
%  \vspace{-2em}
  \caption{The Habanero Java equivalence of \figref{fig:example}.}
%   \vspace{-2em}
  \label{fig:example-hj}
\end{figure}



\begin{figure}
  \begin{center}
     \includegraphics[scale=0.7]{../figs/example-cg}
  \caption{Two computation graphs for the program in \figref{fig:example}: the schedule on the left has no data-race and while the right does.}
%  \vspace{-2em}
   \label{fig:example-cg}
   \end{center}
\end{figure}


\section{Computation Graph and Data-race Detection}
\label{sec:semantics}
A Computation Graph for a task parallel program is a directed acyclic
graph representing the concurrent structure of the program execution
\cite{dennis2012determinacy}.  It is modified here to track memory
locations accessed by tasks.

\begin{definition}
A computation graph is a directed acyclic graph (DAG), $\cg = \tuple{\nodes, \edges, \rv, \wv}$, where $\nodes$ is a finite
set of nodes, $\edges \subseteq \nodes \times \nodes$ is a set of
directed edges, $\rv : (\nodes \mapsto \powerset{\glbls})$ maps $\nodes$ to
the unique identifiers for the shared locations read by the tasks,
$\wv : (\nodes \mapsto \powerset{\glbls})$ maps $\nodes$ to the unique
identifiers for the shared locations written by the tasks, and $\glbls$
is the set of the unique identifiers for the shared locations.
\end{definition}

\algoref{algo:drd} is an algorithm to analyze a computation graph for data-race with
\[
\mathit{conflict}(n_i,n_j) = 
\begin{array}{l}
  \rv(n_i) \cap \wv(n_j) \neq \emptyset\ \vee \\
  \rv(n_j) \cap \wv(n_i) \neq \emptyset\  \vee \\
  \wv(n_i) \cap \wv(n_j) \neq \emptyset\  \\
\end{array}
\]
The notation $n_i \nprec n_j$ means that $n_i$ does not precede $n_j$
in the graph. The complexity of the algorithm is
$|\glbls|*O(|\nodes|^2)$ because computing the transitive closure for
\lineref{loc:path} is $O(|\nodes|*|\edges|)$ and the \textit{conflict}
function is constant.  The algorithm is purposely naive and can be
improved if needed \cite{mellor1991fly,raman2012scalable}.  That said,
problem instances are assumed to be small enough for model checking
which dominates running time since it is limited by the number of
computation graphs that must be enumerated rather than the size of
those graphs.

\begin{algorithm}[t]
  \caption{Data Race detection in a computation graph.} \label{algo:drd}
\begin{algorithmic}[1]
\Function{data\_race}{$\tuple{\nodes, \edges, \rv, \wv}$}
\For {\textbf{each} $n_i,n_j \in \nodes$}
\If {$(n_i \nprec n_j) \wedge (n_j \nprec n_i)$}  \label{loc:path} \label{loc:forall}
   \If {\textit{conflict}$(n_i,n_j,\rv,\wv)$}
      \State {report data-race} \label{loc:datarace}
      \EndIf
\EndIf
 \EndFor
\EndFunction  
\end{algorithmic}
\end{algorithm}


\begin{figure}
  \centering
        \includegraphics[width=0.3\textwidth]{../figs/Fig3-1.pdf}
    \caption{Computation Graph Example.}
    \label{fig:cg}
    \vspace{-1em}
\end{figure}

\section{Task Parallel Programs} \label{sec:cg}
This section formally defines how to build a computation graph from an
execution trace of a task parallel program.  The programming model is
derived from Bouajjani and Emmi for isolated parallel tasks
\cite{bouajjani}; this variant removes the isolation between tasks
with the introduction of shared memory. It additionally restricts task
passing to only allow tasks to be passed when a child completes, and
those tasks are only passed to the parent.

\subsection{Syntax}
The surface syntax for the language is given in \figref{fig:syntax}.
A program \textbf{P} is a sequence of procedures.  The procedure name
$p$ is taken from a finite set of names \texttt{Proc}.  Each procedure
has a single $L$-type parameter \texttt{l} taken from a finite set of
parameter names \texttt{Vars}; restricting to a single-parameter simplifies the presentation and has no bearing on results.  The
semantics is abstracted over concrete values and operations, so the
possible types of \texttt{l} are not specified. Procedures may also
reference shared variables taking from a finite set of names, $\glbls$,
where $\glbls \cap \mathtt{Vars} = \emptyset$. The names include a
special reserved variable \texttt{isolate} that is only used by the
semantics for mutual exclusion.

The body of a procedure is inductively defined by \textbf{s}.  The
expression language, $e$, is also abstracted and not specified, but
assume it includes variables, references, and a finite set of values
,\texttt{Vals}, that include at least Boolean values.  The set of all
expressions is given by \texttt{Exprs}.

\begin{figure}
  \begin{center}
\[
  \begin{array}{rcl}
\textbf{P} &::=& (\textbf{proc}~p~(\textbf{var}\ \texttt{l} : L)~s)* \\
\textbf{s} &::=& s;~s \alt \texttt{l} := e \alt \textbf{skip} \alt [ \textbf{if}~e~\textbf{then}~s~\textbf{else}~s ] \\
&\alt& [ \textbf{while}~e~\textbf{do}~s ] \alt \textbf{call}~\texttt{l}\ := p~e \alt \textbf{return}~e \\
&\alt& \post~r \leftarrow p~e~d \alt \await~r \alt \ewait~r \\
&\alt& [ \isolated~s ]
  \end{array}
\]
  \end{center}
  \caption{The surface syntax for task parallel programs.}
  %\vspace{-2em}
  \label{fig:syntax}
\end{figure}

The \post-statement, \await-statement, \ewait-statement, and
\isolated-statement relate to concurrency; the rest of the statements
have their usual sequential meaning.  The \post-statement adds a task
into a region $r$, taken from a finite set of region identifiers,
\texttt{Regs}, by indicating the procedure $p$ for the task with an
expression for the local variable value $e$, and a return value
handler $d$ to run in the context of the parent task.  Let
\texttt{Stmts} be the set of all statements and let $\texttt{Rets}
\subseteq (\texttt{Vals} \rightarrow \texttt{Stmts})$ be the set of
return value handlers.

The \await\ and \ewait\ statements synchronize a task with the
sub-ordinate tasks in the indicated region.  Intuitively, when a task
calls \await\ on region $r$, it is blocked until all the tasks it owns
in $r$ finish execution.  Similarly, when a task issues an
\ewait\ with region $r$, it is blocked until one task it owns in $r$
completes.  A task is termed \emph{completed} when its statement is a
\textbf{return}-statement.

The \isolated-statement provides mutual exclusion relative to other
\isolated-statements.  If $s$ is isolated, then it runs mutually
exclusive to any other statements $s^\prime$ that are also isolated;
however, $s$ does not run mutually exclusive to other non-isolated
statements that may be concurrent with $s$.

In addition to the restrictions created by the syntax, we assume that
all programs adhere to the following additional restriction:
procedures with side-effecting return value handlers must be the only
members of their respective regions. This restriction is common to
task-parallel languages.

\subsection{Tree-based Semantics}

The semantics of task parallel programs is defined over trees of
procedure frames to represent the parallelism in the language rather
than stacks which are inherently sequential.  That means that the
frame of each posted task becomes a child to the parent's frame. The
parent-child relationship is transferred appropriately when a parent
completes without synchronizing with its children.  The evolution of
the program proceeds by a task either taking a sequential step or a
concurrency related step that creates, synchronizes, or orders with
other tasks.  The semantics additional define the construction of the
computation graph that is also part of the program state.

A task $t = \tuple{\ell, s, d, n}$ is a tuple containing the value,
$\ell$, of the procedure local variable \texttt{l}, along with a
statement $s$, a return value handler $d$, and an associated node,
$n$, in the computation graph for this task.

A \emph{tree configuration}, $c = \tuple{t,m}$, is an inductively
defined tree with task-labeled vertexes, $t$, and region labeled edges
given by the \emph{region valuation} function, $m : \texttt{Regs}
\rightarrow \texttt{Configs}$, where \texttt{Configs} is the set of
tree configurations.  For a given vertex $c = \tuple{t,m}$, $m(r)$
returns the collection of sub-trees connected to the $t$-labeled root
by $r$-labeled edges. In this context, $m$ is local to the current
task frame. Let $m_o$ be the initial region valuation such that
$\forall r \in \mathtt{Regs}, m_o(r) = \emptyset$.

Contexts are used to simplify the rules for the semantics. A
\emph{configuration context}, $C$, isolates individual task
transitions in the tree (e.g., $C[\tuple{t,m}] \rightarrow
C[\tuple{t^\prime,m}]$ denotes a transition on a
task). Similarly, a \emph{statement context}, $S$, indicates the next
statement to be executed.  A \emph{task statement context}, $T =
\tuple{\ell, S, d, n}$ is a task with a statement context in place of
a statement, and likewise $T[s]$ indicates that $s$ is the next
statement to be executed in the task. Like configuration contexts,
task statement contexts isolate the statement to be executed (e.g.,
$C[\tuple{T[s_1],m}] \rightarrow C[\tuple{T[s_2],m}]$ denotes a transition that modifies the statement in some way).

The \emph{state} of a task parallel program is $\tuple{\cg, \sigma,
  \tuple{t,m}}$ where $\cg$ is a computation graph, $\sigma:
\glbls \rightarrow \mathtt{Vals}$ is a partial function mapping global
variable names in $\glbls$, and $\tuple{t,m}$ is the configuration context for the root of the tree. 

Let $\llbracket \cdot \rrbracket_e$ be an partial evaluation function
for expressions without any variables. For convenience in the
semantics:
\begin{eqnarray*}
  e(t, \sigma) &=& e(\tuple{\ell, s, d, n}, \sigma) \\
  &=& e(\ell, \sigma) \\
  &=& \llbracket e[\ell / \texttt{l}, \sigma(\mathtt{g}_0) / \mathtt{g}_0, \sigma(\mathtt{g}_1) / \mathtt{g_1}, \ldots]  \rrbracket_e
  \end{eqnarray*}
If $e[\ell / \texttt{l}, \sigma(\mathtt{g}_0) / \mathtt{g}_0,
  \sigma(\mathtt{g}_1) / \mathtt{g_1}, \ldots]$ has any free variables
or other errors, then by definition, \\ $\llbracket e[\ell /
  \texttt{l}, \sigma(\mathtt{g}_0) / \mathtt{g}_0,
  \sigma(\mathtt{g}_1) / \mathtt{g_1}, \ldots] \rrbracket_e$ has no
meaning and is undefined. The function is naturally extended to used
contexts as indicated by $e(T)$.  Finally, let the set of shared
variables that appear in $e$ be denoted by $\eta(e,\sigma)$.

As indicated previously, a task $t$ is completed when its next to be
executed statement $s$ is \textbf{return} $e$. Its return-value
handler statement is $\mathrm{rvh}(t) = d(e(T))$ given the task's
context.  By definition, $\mathrm{rvh}(t)$ undefined when $t$ is not
completed or $e(T)$ is undefined.

Finally, the notation, $\rho^\prime = \rho[n \bowtie \eta(e,\sigma)]$
with $\bowtie \in \{\mapstou,\mapsto\}$, indicates the $\rho^\prime$
is just like $\rho$ only if $\bowtie = \mapstou$, then $\rho^\prime(n)
= \rho(n) \cup \eta(e,\sigma)$, and if $\bowtie = \mapsto$, then
$\rho^\prime(n) = \eta(e,\sigma)$. The notation, $m_1^\prime = m
\setminus (r \mapsto \tuple{t_2,m_2})$ means that $m_1^\prime$ is just
like $m$ only $m(r)$ does not include $\tuple{t_2,m_2}$.

The semantics produce a computation graph as a by-product of reducing
the program. Two additional data are associated with the computation
graph and are used by the semantics in the construction: $\last$ is a
special node used to assert the observed order of \isolated-statements
and $\regnodemap : \mathtt{Regs} \mapsto \powerset{\nodes}$ is a function used
to join tasks in a region at synchronization.  In general, a function
notation is adopted to access members of tuples. For example, the
members of the $cg = \tuple{\nodes,\edges, \rv,\wv,\last,\regnodemap}$ are accessed as $\nodes(\cg)$, $\edges(\cg)$,
$\rv(\cg)$, etc.


\begin{figure}
  \begin{center}
    \mprset{flushleft}
    \begin{mathpar}
      \inferrule[Post]
                {
                  n_0^\prime = \mathrm{fresh}() \\
                  n_1 = \mathrm{fresh}() \\
                  \nodes(\cg^\prime) = \nodes(\cg) \cup \{n_0^\prime, n_1\} \\
                  \edges(\cg^\prime) = \edges(\cg) \cup \{\tuple{n_0, n_0^\prime}, \tuple{n_0, n_1}\}\\
                  \rv(\cg^\prime) = \rv(\cg)[n_0 \mapstou \eta(e,\sigma)]\\\\
                  \ell = e(\ell^\prime,\sigma) \\
                  m^\prime = m[r \mapstou \tuple{\tuple{\ell, s_p, d,n_1},m_o}]
                }
                {
                  \tuple{\cg,\sigma,C[\tuple{\ell^\prime,
                        S[\post~r \leftarrow p~e~d],d^\prime, n_0}, m]} \rightarrow \\
                  \tuple{\cg^\prime,\sigma,C[\tuple{\ell^\prime,
                        S[\textbf{skip}],d^\prime, n_0^\prime}, m^\prime]}
                }
      \and
      \inferrule[Await]
                {
                  \regnodemap(\cg^\prime) = \regnodemap(\cg)[r \mapstou \node(t_2)] \\
                  \rv(\cg^\prime) = \rv(\cg)[\node(t_2) \mapstou \eta(e,\sigma)]\\
                  m_1 = m \setminus (r \mapsto \tuple{t_2,m_2}) \\
                  (m_1 \cup m_2)(r) \neq \emptyset \\
                  s(t_2) = \mathbf{return}\ e \\
                  s = d(e(t_2, \sigma))
                }
                {
                  \tuple{\cg,\sigma,C[\tuple{\ell,
                        S[\await~r],d, \node}, m]} \rightarrow \\
                  \tuple{\cg^\prime, \sigma, C[\tuple{\ell,
                        S[s;~\await~r],d, \node}, m_1 \cup m_2]}
                }
      \and
      \inferrule[Await-done]
                {
                  n^\prime = \mathrm{fresh}() \\
                  \nodes(\cg^\prime) = \nodes(\cg) \cup \{\node^\prime\} \\
                  \edges(\cg^\prime) = \edges(\cg) \cup \{\tuple{\node, \node^\prime},\tuple{\node(t_2), \node^\prime}\} \cup
                  \{\tuple{\node_i,\node^\prime} \mid \node_i \in \regnodemap(\cg)(r)\} \\
                  \regnodemap(\cg^\prime) = \regnodemap(\cg)[r \mapsto \emptyset] \\
                  \rv(\cg^\prime) = \rv(\cg)[\node(t_2) \mapstou \eta(e,\sigma)]\\
                  m_1 = m \setminus (r \mapsto \tuple{t_2,m_2}) \\
                  (m_1 \cup m_2)(r) =
                  \emptyset \\
                  s(t_2) = \mathbf{return}\ e \\
                  s = d(e(t_2, \sigma))
                }
                {
                  \tuple{\cg,\sigma,C[\tuple{\ell,
                        S[\await~r],d, \node}, m]} \rightarrow \\
                  \tuple{\cg^\prime, \sigma, C[\tuple{\ell,
                        S[s],d, \node^\prime}, m_1 \cup m_2]}
                }
      \and
      \inferrule[Isolated]
                {
                  n^\prime = \mathrm{fresh}() \\
                  \nodes(\cg^\prime) = \nodes(\cg) \cup \{n^\prime\} \\
                  \edges(\cg^\prime) = \edges(\cg) \cup \{\tuple{n, n^\prime},  \tuple{\last(\cg),n^\prime}\}\\
   				  \sigma(\mathtt{isolate}) = \falsev \\
                  \sigma^\prime = \sigma[\mathtt{isolate} \mapsto \truev]
                }
                {
                  \tuple{\cg,\sigma,C[\tuple{\ell^\prime,S[\isolated~s],d^\prime, n, 0}, m]} \rightarrow \\
                  \tuple{\cg^\prime, \sigma^\prime, C[\tuple{\ell^\prime,
				        S[s; \textbf{isolated-end}],d^\prime, n^\prime, 1}, m]}
                }
      \and
      \inferrule[Isolated-done]
                {
                  n^\prime = \mathrm{fresh()}\\
                  \last(\cg^\prime) = n \\
                  \nodes(\cg^\prime) = \nodes(\cg) \cup \{n^\prime\} \\
                  \edges(\cg^\prime) = \edges(\cg) \cup \{\tuple{n, n^\prime}\} \\
                  \sigma(\mathtt{isolate}) = \truev \\
                  \sigma^\prime = \sigma[\mathtt{isolate} \mapsto \falsev]
                }
                {
                  \tuple{\cg, \sigma, C[\tuple{\ell^\prime,S[\textbf{isolated-end}],d^\prime, n, 1}, m]} \rightarrow \\
                  \tuple{\cg^\prime, \sigma^\prime, C[\tuple{\ell^\prime,
				   S[\textbf{skip}],d^\prime, n^\prime,0}, m]}
                }
                
\end{mathpar}
  \end{center}
  \caption{The transition rules for the inter-procedural statements, task creation, and task synchronization.}
%    \vspace{-2em}
  \label{fig:inter}
    \label{fig:semantics}
\end{figure}

The semantics is given as a set of transition rules relating states.
\figref{fig:inter} contains a subset of the rules pertinent to the
proof of correctness in the next section. The omitted rules for
intra-procedural statements have their usual definition; additionally, the
inter-procedural statement, $\textbf{call}~\texttt{l}\ := p~e$, is
syntactic sugar for
\[
\post~r_\mathit{call}\leftarrow p~e~\lamdefe{v}{\texttt{l} := v};~
\ewait~r_\mathit{call},
\]
where $r_{call}$ is exclusive to the caller containing no other tasks. The semantics create a computation graph as a byproduct of reducing the program. The computation graph observes rather than determines the execution of the program. 

The \textsc{Post} rule creates a new child task. The rule adds two
fresh nodes $n_0^\prime$ and $n_1$ to the computation graph: node
$n_0^\prime$ represents the statements following \post\ and $n_1$
represents the statements to be executed by the new task. The rule
orders both after the current node for the parent, $n_0$, in the
computation graph.  The read set $\rho$ of node $n_0$ is updated to
include any global variables referenced in the expression,
$\eta\left(e\right)$, for the local parameter value in the new
task. The region mapping $m$ of the parent task is updated to include
the new task with its empty initial region mapping.


The \textsc{Await} rule blocks the execution of the currently
executing task until a task in the indicated region completes.  A new
node to join the two tasks is not created in the computation graph,
nor are the two tasks ordered in the sense of join because the choice
of task $t_2$ in the region is non-deterministic; as such, the
computation graph allows tasks in the region to join in any order
contrary to the observed reduction by the rule. The rule saves the
current node in the graph for $t_2$, $\node(t_2)$, to join later once
the region is empty, and it updates the read set for $t_2$ on the
expression in the \textbf{return}-statement. The rest of the rule
separates out the task from the region, makes sure the region is not
yet empty, and gets the statement for the return value handler. The
new state adds an \textbf{await}-statement after the return value
handler statement since the region is not yet empty, and the region
valuation function in the new state includes any tasks owned by
$t_2$.\footnote{The region valuation $m_2$ may actually include tasks
  posted into $r$ because configurations are local rather than
  global. This locality means that two tasks may post into the same
  region, but neither tasks knows about the tasks posted by the other
  into that region.}


The \textsc{Await-Done} activates when the last task in the region is
joined. It differs from the \textsc{Await} rule in that it
orders all tasks that have joined in the region to happen-before the
new node for the parent in the computation graph, and it does not
insert another \textbf{await}-statement in the new state.

The \textsc{Ewait} and \textsc{Ewait-Done} rules, not shown in the
figure, follow \textsc{Await} and \textsc{Await-Done} respectively
only without the recursive statement when the region is not empty
since it only needs to wait on a single task to complete. The rules
delay the ordering of tasks joined in the region to when the region
becomes empty (i.e., the last task joins) just as done for
\textbf{await}-statements.

If no other isolated statements are running, then the
\textsc{Isolated} rule updates the \texttt{isolated} shared variable
and inserts after the isolated statement $\mathit{s}$ the new
\textbf{isolated-end} keyword. The computation graph gets a new node
to track accesses in the isolated statement with an appropriate edge
from the previous node. A sequencing edge from $\last$ is also added
so the previous isolated statement happens before this new isolated
statement. As a note, $\last$ is initialized to the initial node when
execution starts.

The \textsc{Isolated-End} rule creates a new node in the computation
graph to denote the end of isolation, updates the \texttt{isolated}
shared variable, and it updates $\last$ to properly sequence any
future isolation. \textbf{isolated}-statements are totally ordered in
the computation graph.

The initial task for a program is $t_o = \tuple{\ell, s_o,
  \lamdefe{v}{\mathtt{skip}}, n}$ where $\ell$ is the initial value of
the parameter for the top level procedure $p_o$, $s_o = \post r_0
\leftarrow p_0~\ell~\lamdefe{v}{\mathtt{skip}}; \await~r_0$;
$\await~r_1; \ldots$, and $n$ is a fresh node for the computation
graph (i.e., $n = \mathrm{fresh}()$). The initial state of a program
is given as $\tuple{\cg_o, \sigma_o, \tuple{t_o,m_o}}$ where $\cg_o$
is an initial graph such that $N = \{n\}$, $E = \emptyset$, $\rho(n) =
\emptyset$, and $\omega(n) = \emptyset$; $\sigma_o$ maps all global
variables to an initial default value; and $\forall r \in
\texttt{Regs}, m(r) = \emptyset$.


\section{Proof of correctness}
\label{sec:proof}

For a given program and input, the computation graphs produced by the
tree semantics in Section~\ref{sec:semantics} demonstrate a data race
if and only if a data race is possible for the given program and
input.

\begin{comment}
\begin{definition}\label{def:scheduler}
A scheduler is a program capable of deciding which task to reduce,
given a tree configuration.
\end{definition}
\end{comment}

\begin{definition}[\textit{conflict}]
Two statements conflict if they both access the same shared variable
and at least one of them writes to that variable.
$\rv\left(\statement\right)$ and $\wv\left(\statement\right)$ behave
as expected.
\begin{gather*}
\mathit{conflict}(\statement_i,\statement_j) =
\begin{array}{l}
  \rv(\statement_i) \cap \wv(\statement_j) \neq \emptyset\ \vee \\
  \rv(\statement_j) \cap \wv(\statement_i) \neq \emptyset\ \vee \\
  \wv(\statement_i) \cap \wv(\statement_j) \neq \emptyset\ \\
\end{array}
\end{gather*}
\end{definition}

\begin{definition}[Active statements]
A state \st\ has a set of active statements \activestatements{\st}
that corresponds to the next statement to be reduced in each of the
active tasks in the state.
\end{definition}

\begin{definition}[Concurrency]
Two statements are concurrent if and only if an execution of the
program can result in a state \st\ such that both statements are
active at the same time:

\begin{gather*}
\concurrent{\statement}{\statement'} \bimp \exists \st \left(\st_{0}
\overset{*}{\rightarrow} \st \; \wedge \; \left\{\statement,
\statement'\right\} \subseteq \activestatements{\st}\right) .
\end{gather*}

A state \st\ that satisfies this condition for \statement\ and
\statement' is called a witness state for
\concurrent{\statement}{\statement'}.
\end{definition}

\begin{definition}[Data race]
Two statements $\statement$ and $\statement'$ demonstrate a data race if
and only if they conflict and if they are concurrent:

\begin{align*}
\dr\left(\statement, \statement'\right) =
\concurrent{\statement}{\statement'} \; \wedge
\mathit{conflict}\left(\statement, \statement'\right).
\end{align*}
\end{definition}

Two statements that occur in the same thread of execution cannot be
concurrent, as exactly one statement is active in each active thread
at any point in time. Two statements inside of \isolated-statements
cannot be concurrent; once one has entered its \isolated-statement,
all other threads must block upon reaching an \isolated-statement
until the first thread exits its \isolated-statement.

Before proving the correctness of data races in the computation graph,
it is useful to observe that only nodes that end in a \post-statement
and \isolated-statements can have multiple outgoing edges. In both
cases, the edges go to nodes in different threads of execution.
Similarly, only \isolated-statements and nodes following an \await- or
\ewait-statement (in the parent thread) and following
\textbf{return}-statements (in child threads) have multiple incoming
edges. As with \post\ statements, the edges that converge on a node
come from distinct threads of execution.

\begin{lemma}
\label{lemma:concurrent-to-unordered}
If two statements $\statement \in \node$ and $\statement' \in \node'$
are concurrent, \node\ and \node' are unordered in the computation
graph \cg\ in every witness state \st\ for
\concurrent{\statement}{\statement'}:

\begin{gather*}
\text{Given } \, \statement \in \node \text{ and } \statement' \in
\node', \\
\forall \st \left(\st_{0} \overset{*}{\rightarrow} \st \wedge
\left\{\statement, \statement'\right\} \subseteq
\activestatements{\st} \implies
\unrelated{\node}{\node'}{\prec}{\nprec}\right) .
\end{gather*}
\end{lemma}

\begin{proof}
By inspection of the semantics, outgoing edges are created on active
threads' nodes only on or after each respective node's final
reduction. New nodes are always fresh. As such, the nodes for any two
active statements are unrelated.
\end{proof}

\begin{lemma}
\label{lemma:unordered-to-concurrent}
If two nodes $\node$ and $\node'$ are unordered in a state's
computation graph \cg, every $\statement \in \node$ and $\statement'
\in \node'$ are concurrent:

\begin{gather*}
\text{Given} \, \statement \in \node \text{ and } \statement' \in
\node', \unrelated{\node}{\node'}{\prec}{\nprec} \implies
\concurrent{\statement}{\statement'} .
\end{gather*}
\end{lemma}

\begin{proof}

The two nodes \node\ and \node' are both reachable; otherwise, they
would not have been generated in \cg. Within a node, it is possible to
advance or wait independent of other nodes' behaviors. Accordingly, it
is possible to begin at $\st_{0}$ and advance until \statement\ is
active. Similarly, it is possible to advance until \statement' is
active. What remains to be proven is whether or not it is possible to
reach a state where both \statement\ and \statement' are active; in
other words, if it is possible for some schedule to reach \statement\
in one task and then to reach \statement' in some other task without
advancing the first task any further.

By the construction of \cg, \node\ and \node' must have some least
common ancestor $\node_{A}$ that is also reachable. $\node_{A}$ must
either end in a \post-statement or be an \isolated-statement, as the
reduction rules only allow these two statements to have multiple
outgoing edges. In both cases, the child nodes of $\node_{A}$ must be
in different tasks. Without loss of generality, we say that \task\
either contains \node or is some ancestor of the task that does.
Similarly, we say that \task' either contains \node' or is an ancestor
of the task that does.

We first advance to $\node_{A}$ on some schedule that does not
contradict $\prec$. This is possible because $\node_{A}$, \node, and
\node' were all generated. At this point, execution of \task\ and its
children is independent of \task' and its children because $\node_{A}$
is the least common ancestor of \node\ and \node'; no
\isolated-statements can cause one to block on the other, nodes are
generated fresh and so \post-statements cannot link them, and \await-
and \ewait-statements cannot join them. We advance \task\ and any
relevant children until reaching \node\ and then proceed until
reaching \statement. Then, we advance \task' and any relevant children
until reaching \node' and then proceed until reaching \statement'.

At this point, both \statement\ and \statement' are active, so they
must be concurrent.
\end{proof}

Our proof asserts that the computation graph demonstrates a data race
if and only if such a data race exists. As such, we need not reason
about any behaviors that occur after a data race.

\begin{comment}
\begin{definition}[Schedule]
A schedule is a series of states that starts with the initial state
$\st_{0}$. Each state can be derived from its predecessor in the
schedule:

\begin{gather*}
\tuple{\st_{0}, \st_{1}, \ldots, \st} \text{ where } \st_{0}
\rightarrow \st_{1} \rightarrow \ldots \rightarrow \st .
\end{gather*}
\end{definition}

\begin{definition}[Race-free schedule]
A race-free schedule is a schedule that contains only race-free
reductions.
\end{definition}

Data races occur, in one sense, as an interaction between two
different reductions. However, a single reduction is sufficient to
influence downstream behaviors. It is possible to reason about all
race-free schedules without enumerating them.

\begin{definition}[Race-free state]
A race-free state \rf{\st} is a state in a race-free schedule.
\end{definition}
\end{comment}

\begin{lemma}[Soundness of \textit{conflict} over nodes]
\label{lemma:conflict-sound}
If two nodes conflict, there exists a pair of statements, one from
each node, that conflicts:

\begin{gather*}
\mathit{conflict}\left(\node, \node'\right) \implies \exists
\statement \in \node, \statement' \in \node'
\left(\mathit{conflict}\left(\statement, \statement'\right)\right) .
\end{gather*}
\end{lemma}

\begin{proof}
If two nodes conflict, it is because \rv\ and \wv\ were updated in
some reduction. More specifically, if $\rv\left(\node\right) \ne
\emptyset$, at least one statement $\statement \in \node$ must read a
global variable; the reduction of statements that read a global
variable are the only way that \rv\ updates. The same reasoning
applies to \wv.
\end{proof}

\begin{lemma}[Completeness of \textit{conflict} over nodes]
\label{lemma:conflict-complete}
If two statements conflict, their respective nodes will conflict.

\begin{gather*}
\text{Given } \statement \in \node \wedge \statement' \in \node',
\left(\mathit{conflict}\left(\statement, \statement'\right)\right)
\implies \mathit{conflict}\left(\node, \node'\right) .
\end{gather*}
\end{lemma}

\begin{proof}
If $\statement \in \node$, then \statement\ must have been reduced in
\node. By inspection of the semantic rules, $\rv\left(\node\right)$
and $\wv\left(\node\right)$ must be updated to include \statement\ as
necessary. Accordingly, nodes conflict whenever statements they
include conflict.
\end{proof}

\begin{theorem}[Soundness of computation graph over data races]
If a computation graph reports a data race, there is a data race in
the program on the given input.
\end{theorem}
\begin{proof}
By Lemma~\ref{lemma:unordered-to-concurrent} and
Lemma~\ref{lemma:conflict-sound}.
\end{proof}

\begin{theorem}[Completeness of computation graph over data races]
If there is a data race in the program on the given input, a
computation graph generated by the model checker reports a data race.
\end{theorem}
\begin{proof}
The definition of data race states that the two statements must be
concurrent, which implies that it is possible to reach both of them in
the same execution. As a result, any two statements that conflict and
are both reachable will be members of nodes in some computation graph.
By Lemma~\ref{lemma:conflict-complete}, the computation graph
identifies the conflict.

The model checker ensures that a computation graph exists for every
possible ordering on \isolated-statements. Computation graphs are
strict partial orders on nodes. Imposing an order on isolated edges
takes the form of adding a tuple on two unrelated members of the
strict partial order and calculating its closure to create a second
relation and adding the reverse of the tuple to the original relation
and calculating its transitive closure to create a third. The
properties of strict partial orders guarantee that the second and
third relations are also strict partial orders and that any two
members of the strict partial orders (besides the two in the tuple)
that were unrelated in the original order are unrelated in at least
one of the resulting orders.

As such, at least one of the resulting computation graphs will
correctly identify that two statements belong to unordered
nodes. By Lemma~\ref{lemma:concurrent-to-unordered}, the statements
are concurrent.
\end{proof}

\section{Results}

\begin{table*}[t]
\centering
\caption{Verification of HJ Micro-benchmarks using CGRaceDetector}
\label{tab:perf}
\begin{tabular}{|c|c|c|c|c|c|c|c|c|}
\hline
        &        & \multicolumn{3}{c|}{CGRaceDetector} & \multicolumn{3}{c|}{Precise Race Detector}
 \\ \hline
Test Case Name & SLOC & Tasks & States  & Time   & Error Info & States  & Time   & Error Info 
\\ \hline
Search Count & 50 & 4 & 195 & 0:00:01 & No Race & 145139 & 0:00:45 & No Race 
 \\ \hline
Existence of an occurrence & 45 & 4 & 174 & 0:00:01 & Detected Race & 50197 & 0:00:15 & Detected Race 
\\ \hline
Index of occurrence & 38 & 4 & 197 & 0:00:01 & Detected Race & 68806 & 0:00:29 & Detected Race 
\\ \hline
Existence of occurrence with no task & 45 & 2 & 117 & 0:00:00 & Detected Race & 296 & 0:00:00 & Detected Race
\\ 
creation after instance is found & &  &  &  & & & &
\\ \hline
Search Index With No task creation & 48  & 2 & 119 & 0:00:00 & Detected Race & 326 & 0:00:00 & Detected Race
\\
after Instance is Found &  &  &  & & & & &
\\ \hline
\end{tabular}
\end{table*}

We verified some of the HJ microbenchmarks that make use of only the basic parallel constructs such as async and finish using the CGRaceDetector listener. The CGRaceDetector is able to build computation graphs of the HJ programs by exploring very few states. We compared the output of CGRaceDetector to the output of PreciseRaceDetector and found that CGRaceDetctor was able to correctly identify races in all programs. These micro-benchmarks are variations of a linear search algorithm. The first test finds the count of occurrences of a search string in a given text string. The second test confirms the existence of search string in the given text string. The third test returns the index of occurrence of the search string. In case of multiple occurrences, the output becomes non-deterministic. The fourth test also confirms the existence of the search string in the given text. However, as soon as the search text is found, no more processes are spawned to search the text and the program is terminated. Similarly, in the fifth test, as soon as a process returns the index of occurrence of search text, the program terminates. The results are presented in Table I. The sizes of the programs are indicated by the SLOC column and Tasks columns represents the number of tasks created in every program. The results of CGRaceDetector and Precise Race Detector are compared. The Precise Race Detector systematically explores the entire state space of the program. The CGRaceDetector just uses one thread interleaving to detect data races. Hence, the time required by CGRaceDetector is considerably smaller than the time required by Precise Race Detector to execute.




An informal proof for the optimized schedule factory's soundness and correctness follows. Suppose an HJ program data races on Thread A and B. In order for the two threads to data race, the threads will need to be alive at one point of execution. \figref{fig:proof} shows a simplified execution process for JPF using the optimized scheduler to find a data race with \texttt{ChoiceGenerators} with creation of new threads and shared field accesses.

First, there is a \texttt{ChoiceGenerator} that has these two threads before the shared field is accessed (such as the creation of thread B). Thread A is executed first (i). The shared field is accessed by A, and JPF stores this access using global search object IDs. However, on this execution, JPF is unaware that Thread B accesses the shared field, and continues running. Thread A terminates, and JPF switches to a different thread, though a new \texttt{ChoiceGenerator} is not created. Thread B executes, and runs till B accesses the shared field. JPF stores this access using the global search object ID and finds that A also accesses it. However, since Thread A no longer is alive, a \texttt{ChoiceGenerator} is not created, and B continues to run till it terminates.

At the end of the execution, JPF back tracks to the last \texttt{ChoiceGenerator} (ii) and picks to execute Thread B. JPF executes B until B accesses the shared field. The global search object ID once again determines that Thread A also accesses the field. However, since Thread A is alive, a \texttt{ChoiceGenerator} is created. Thread B continues (iii) execution until termination, and Thread A then executes. As before (i), Thread A executes till it reaches the shared field, but finds that Thread B is no longer alive, so no \texttt{ChoiceGenerator} is created. Thread A terminates, and JPF backtracks to the last \texttt{ChoiceGenerator} and executes a different thread (iv). Thread A is selected and executed. When Thread A reaches the shared field access, JPF checks its global search object ID, finds that Thread B is still alive, and inserts another \texttt{ChoiceGenerator}.

At this point, JPF's \texttt{PreciseRaceDetector} searches at the \texttt{ChoiceGenerator} (and has been doing at all \texttt{ChoiceGenerators} thus far) for two threads that have a shared field access on the same field. If there are, which Thread A and B are, then \texttt{PreciseRaceDetector} makes an additional check for at least one being a write in order to report a data race. Since Thread A and B do, JPF with the optimized scheduler reports a data race.
\begin{figure}[t]
\begin{center}
\includegraphics[width=3.25in]{../figs/InformalProofDiagram}
\end{center}
\vspace{-10pt}
\caption{A JPF search illustrating the scheduler.}
\label{fig:proof}
\end{figure}

\section{Related Work}

There is an existing extension for JPF for the X10 Language
\cite{conf:icst:GligoricMM12,x10}. Habanero is closely related to X10 in many of
its constructs. In the extension, JPF operates directly on the actual X10
runtime system. To accomplish the integration, JPF is modified, the X10 runtime
is modified, the X10 compiler is extended, and a new static analysis is
presented to help control state explosion. The extension represents a
significant effort that affects all aspects of the X10 framework to enable JPF
verification. 

There is a formal model for the Chapel language with an accompanying model
checker that employs symbolic execution \cite{chapel}. The formal model is an
intermediate representation (IR) suitable for concurrent constructs. The
approach compiles Chapel programs into the IR and the model checker then
verifies the IR for deadlock and data-race freedom. Creating a compiler and
model checker is a significant undertaking beyond the approach in this paper.
More critically, the verification tool models the runtime including the number
of available worker threads to service tasks; thus, the verification results are
dependent on the number of worker threads in the configuration rather than the
semantics of the Chapel language. 

\begin{comment} In this paper, correctness is a property of the HJ language
semantics with the given HJ program and not any aspect of the runtime
implementation. Verifying the HJ runtime system implements HJ semantics is a
verification problem separate from verifying that an HJ program is data-race
free.  \end{comment}

Another approach to verifying concurrent languages is to leverage the production
level language runtime system itself
 \cite{Vakkalanka:2008:DVM:1427782.1427794,Vo:2009:FVP:1594835.1504214,5644885,6113841}.
These approaches typically require instrumentation of the source program,
wrappers to intercept calls into the runtime, and a way to control runtime
behavior. Although they are typically able to generate states faster than JPF,
verification results are dependent on the employed runtime correctly
implementing the language semantics. 

Many tools for dynamic race detection have been developed \cite{Eraser,
Eraser-Upgrade, Cilk-Dynamic}. These tools track the set of locks held by each
task during execution and use these sets to determine if a shared resource is
insufficiently protected. These tools produce results that are independent of
thread interleavings. This is an improvement as compared to previous tools that are
dependent upon the thread interleavings of the current
execution \cite{Lamport-Comparison, Mesa, Lamport-Online}. 
Race detection algorithms have also been developed for task-parallel
languages \cite{Async-Finish-Race, SP-BAGS}. These approaches utilize the
structured parallelism of the language to quickly detect races. However, the
results of this approach are also limited to a single execution.

Many different approaches have been developed to statically detect race
conditions in programs \cite{ESC, Warlock, RacerX, Relay}. Each of these techniques
require varying levels of instrumentation by the user. RacerX infers
the resources each lock protects, code contexts which are multithreaded, and
race conditions that have a "dangerous" effect upon the running program.
RacerX relies upon the user to annotate the location of the method for
performing the lock/unlock operations. Any other annotations by the user
are acceptable, but not required.

Relay performs a static lockset analysis. The Relay algorithm computes relative
locksets that belong to each function in the program. This bottom-up approach
scales very nicely, however, this approach remains unsound.

General permission regions (GPR) is another static analysis strategy
that infers the locations in which to place program annotations \cite{Westbrook:2012:PPR:2367163.2367201}. Once the annotations have been
statically injected, a dynamic analysis is run to detect the presence of race
conditions. Unlike, RacerX, GPR doesn't require any user annotations, although
it will honor any annotations introduced by the user. GPR correctly detects race
conditions for most common parallel programming paradigms. 

\begin{comment} Such a dependence can be avoided, as in this work, by creating a
non-performance oriented runtime that is simple enough to manually verify for
correctness \cite{Morse:2012:MAM:2189257.2189279}. It is much easier to access
the state and direct behavior, as a model checker, in such systems.
\end{comment}

Recent work proves the problem of state-reachability to be decidable and
EXPSPACE hard for finite-valued programs in languages such as X10/Habanero
\cite{Bouajjani:2012:ARP:2103621.2103681}. The result is limited to a subset of
the powerful task constructs in such languages and justifies a model checking
effort. The computability and complexity of the more advanced constructs such as
phasers is yet to be determined.

\section{Conclusions \& Future Work}
This paper presents a model checking algorithm to prove when a
Habanero program does not contain any data races, deadlocks, assertion
violations, or exceptions for a given program input. The algorithm,
based on permission regions, only considers scheduling points in the
search tree at the boundaries of permission regions and {\tt
  isolated}-constructs. The paper includes a proof of soundness for
the algorithm, meaning that the algorithm may reject a correct program
due to the size of the permission regions.

The effectiveness of the algorithm is shown in several benchmark
programs that cover many of the Habanero concurrency constructs. The
analysis is done using a new Java library implementation of the Habanero
runtime that is intended for debugging and verification. The new algorithm, with permission regions,
is implemented as an extension to the \jpf\ model checker. The results
from the benchmark programs indicate a significant cost reduction when
using the new algorithm.

Future work includes automating the annotation of permission regions
based on the sharing detection in \jpf; automating the validation of any
counter-example; developing techniques to automatically refine
permission regions from counter-examples when needed; a partial order
reduction over permission regions; static-analysis to prevent
scheduling on regions that cannot race; applying symbolic techniques to reason over input; and studying
benchmarks that are representative of real world Habanero programs.



%% Acknowledgments
\begin{acks}                            %% acks environment is optional
                                        %% contents suppressed with 'anonymous'
  %% Commands \grantsponsor{<sponsorID>}{<name>}{<url>} and
  %% \grantnum[<url>]{<sponsorID>}{<number>} should be used to
  %% acknowledge financial support and will be used by metadata
  %% extraction tools.
  This material is based upon work supported by the
  \grantsponsor{GS100000001}{National Science
    This research is funded by NSF Grant 
    Foundation}{http://dx.doi.org/10.13039/100000001} under Grant
  No.~\grantnum{GS100000001}{1302524}.  Any opinions, findings, and
  conclusions or recommendations expressed in this material are those
  of the author and do not necessarily reflect the views of the
  National Science Foundation.
\end{acks}


%% Bibliography
%\bibliography{bibfile}


%% Appendix
%\appendix
%\section{Appendix}

%Text of appendix \ldots
\bibliography{../bib/paper} 

\end{document}
