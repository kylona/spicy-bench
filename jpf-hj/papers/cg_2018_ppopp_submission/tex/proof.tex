\section{Proof of correctness}
\label{sec:proof}

For a given program and input, the computation graphs produced by the
tree semantics in Section~\ref{sec:semantics} demonstrate a data race
if and only if a data race is possible for the given program and
input.

\begin{comment}
\begin{definition}\label{def:scheduler}
A scheduler is a program capable of deciding which task to reduce,
given a tree configuration.
\end{definition}
\end{comment}

\begin{definition}[\textit{conflict}]
Two statements conflict if they both access the same shared variable
and at least one of them writes to that variable.
$\rv\left(\statement\right)$ and $\wv\left(\statement\right)$ behave
as expected.
\begin{gather*}
\mathit{conflict}(\statement_i,\statement_j) =
\begin{array}{l}
  \rv(\statement_i) \cap \wv(\statement_j) \neq \emptyset\ \vee \\
  \rv(\statement_j) \cap \wv(\statement_i) \neq \emptyset\ \vee \\
  \wv(\statement_i) \cap \wv(\statement_j) \neq \emptyset\ \\
\end{array}
\end{gather*}
\end{definition}

\begin{definition}[Active statements]
A state \st\ has a set of active statements \activestatements{\st}
that corresponds to the next statement to be reduced in each of the
active tasks in the state.
\end{definition}

\begin{definition}[Concurrency]
Two statements are concurrent if and only if an execution of the
program can result in a state \st\ such that both statements are
active at the same time:

\begin{gather*}
\concurrent{\statement}{\statement'} \bimp \exists \st \left(\st_{0}
\overset{*}{\rightarrow} \st \; \wedge \; \left\{\statement,
\statement'\right\} \subseteq \activestatements{\st}\right) .
\end{gather*}

A state \st\ that satisfies this condition for \statement\ and
\statement' is called a witness state for
\concurrent{\statement}{\statement'}.
\end{definition}

\begin{definition}[Data race]
Two statements $\statement$ and $\statement'$ demonstrate a data race if
and only if they conflict and if they are concurrent:

\begin{align*}
\dr\left(\statement, \statement'\right) =
\concurrent{\statement}{\statement'} \; \wedge
\mathit{conflict}\left(\statement, \statement'\right).
\end{align*}
\end{definition}

Two statements that occur in the same thread of execution cannot be
concurrent, as exactly one statement is active in each active thread
at any point in time. Two statements inside of \isolated-statements
cannot be concurrent; once one has entered its \isolated-statement,
all other threads must block upon reaching an \isolated-statement
until the first thread exits its \isolated-statement.

Before proving the correctness of data races in the computation graph,
it is useful to observe that only nodes that end in a \post-statement
and \isolated-statements can have multiple outgoing edges. In both
cases, the edges go to nodes in different threads of execution.
Similarly, only \isolated-statements and nodes following an \await- or
\ewait-statement (in the parent thread) and following
\textbf{return}-statements (in child threads) have multiple incoming
edges. As with \post\ statements, the edges that converge on a node
come from distinct threads of execution.

\begin{lemma}
\label{lemma:concurrent-to-unordered}
If two statements $\statement \in \node$ and $\statement' \in \node'$
are concurrent, \node\ and \node' are unordered in the computation
graph \cg\ in every witness state \st\ for
\concurrent{\statement}{\statement'}:

\begin{gather*}
\text{Given } \, \statement \in \node \text{ and } \statement' \in
\node', \\
\forall \st \left(\st_{0} \overset{*}{\rightarrow} \st \wedge
\left\{\statement, \statement'\right\} \subseteq
\activestatements{\st} \implies
\unrelated{\node}{\node'}{\prec}{\nprec}\right) .
\end{gather*}
\end{lemma}

\begin{proof}
By inspection of the semantics, outgoing edges are created on active
threads' nodes only on or after each respective node's final
reduction. New nodes are always fresh. As such, the nodes for any two
active statements are unrelated.
\end{proof}

\begin{lemma}
\label{lemma:unordered-to-concurrent}
If two nodes $\node$ and $\node'$ are unordered in a state's
computation graph \cg, every $\statement \in \node$ and $\statement'
\in \node'$ are concurrent:

\begin{gather*}
\text{Given} \, \statement \in \node \text{ and } \statement' \in
\node', \unrelated{\node}{\node'}{\prec}{\nprec} \implies
\concurrent{\statement}{\statement'} .
\end{gather*}
\end{lemma}

\begin{proof}

The two nodes \node\ and \node' are both reachable; otherwise, they
would not have been generated in \cg. Within a node, it is possible to
advance or wait independent of other nodes' behaviors. Accordingly, it
is possible to begin at $\st_{0}$ and advance until \statement\ is
active. Similarly, it is possible to advance until \statement' is
active. What remains to be proven is whether or not it is possible to
reach a state where both \statement\ and \statement' are active; in
other words, if it is possible for some schedule to reach \statement\
in one task and then to reach \statement' in some other task without
advancing the first task any further.

By the construction of \cg, \node\ and \node' must have some least
common ancestor $\node_{A}$ that is also reachable. $\node_{A}$ must
either end in a \post-statement or be an \isolated-statement, as the
reduction rules only allow these two statements to have multiple
outgoing edges. In both cases, the child nodes of $\node_{A}$ must be
in different tasks. Without loss of generality, we say that \task\
either contains \node or is some ancestor of the task that does.
Similarly, we say that \task' either contains \node' or is an ancestor
of the task that does.

We first advance to $\node_{A}$ on some schedule that does not
contradict $\prec$. This is possible because $\node_{A}$, \node, and
\node' were all generated. At this point, execution of \task\ and its
children is independent of \task' and its children because $\node_{A}$
is the least common ancestor of \node\ and \node'; no
\isolated-statements can cause one to block on the other, nodes are
generated fresh and so \post-statements cannot link them, and \await-
and \ewait-statements cannot join them. We advance \task\ and any
relevant children until reaching \node\ and then proceed until
reaching \statement. Then, we advance \task' and any relevant children
until reaching \node' and then proceed until reaching \statement'.

At this point, both \statement\ and \statement' are active, so they
must be concurrent.
\end{proof}

Our proof asserts that the computation graph demonstrates a data race
if and only if such a data race exists. As such, we need not reason
about any behaviors that occur after a data race.

\begin{comment}
\begin{definition}[Schedule]
A schedule is a series of states that starts with the initial state
$\st_{0}$. Each state can be derived from its predecessor in the
schedule:

\begin{gather*}
\tuple{\st_{0}, \st_{1}, \ldots, \st} \text{ where } \st_{0}
\rightarrow \st_{1} \rightarrow \ldots \rightarrow \st .
\end{gather*}
\end{definition}

\begin{definition}[Race-free schedule]
A race-free schedule is a schedule that contains only race-free
reductions.
\end{definition}

Data races occur, in one sense, as an interaction between two
different reductions. However, a single reduction is sufficient to
influence downstream behaviors. It is possible to reason about all
race-free schedules without enumerating them.

\begin{definition}[Race-free state]
A race-free state \rf{\st} is a state in a race-free schedule.
\end{definition}
\end{comment}

\begin{lemma}[Soundness of \textit{conflict} over nodes]
\label{lemma:conflict-sound}
If two nodes conflict, there exists a pair of statements, one from
each node, that conflicts:

\begin{gather*}
\mathit{conflict}\left(\node, \node'\right) \implies \exists
\statement \in \node, \statement' \in \node'
\left(\mathit{conflict}\left(\statement, \statement'\right)\right) .
\end{gather*}
\end{lemma}

\begin{proof}
If two nodes conflict, it is because \rv\ and \wv\ were updated in
some reduction. More specifically, if $\rv\left(\node\right) \ne
\emptyset$, at least one statement $\statement \in \node$ must read a
global variable; the reduction of statements that read a global
variable are the only way that \rv\ updates. The same reasoning
applies to \wv.
\end{proof}

\begin{lemma}[Completeness of \textit{conflict} over nodes]
\label{lemma:conflict-complete}
If two statements conflict, their respective nodes will conflict.

\begin{gather*}
\text{Given } \statement \in \node \wedge \statement' \in \node',
\left(\mathit{conflict}\left(\statement, \statement'\right)\right)
\implies \mathit{conflict}\left(\node, \node'\right) .
\end{gather*}
\end{lemma}

\begin{proof}
If $\statement \in \node$, then \statement\ must have been reduced in
\node. By inspection of the semantic rules, $\rv\left(\node\right)$
and $\wv\left(\node\right)$ must be updated to include \statement\ as
necessary. Accordingly, nodes conflict whenever statements they
include conflict.
\end{proof}

\begin{theorem}[Soundness of computation graph over data races]
If a computation graph reports a data race, there is a data race in
the program on the given input.
\end{theorem}
\begin{proof}
By Lemma~\ref{lemma:unordered-to-concurrent} and
Lemma~\ref{lemma:conflict-sound}.
\end{proof}

\begin{theorem}[Completeness of computation graph over data races]
If there is a data race in the program on the given input, a
computation graph generated by the model checker reports a data race.
\end{theorem}
\begin{proof}
The definition of data race states that the two statements must be
concurrent, which implies that it is possible to reach both of them in
the same execution. As a result, any two statements that conflict and
are both reachable will be members of nodes in some computation graph.
By Lemma~\ref{lemma:conflict-complete}, the computation graph
identifies the conflict.

The model checker ensures that a computation graph exists for every
possible ordering on \isolated-statements. Computation graphs are
strict partial orders on nodes. Imposing an order on isolated edges
takes the form of adding a tuple on two unrelated members of the
strict partial order and calculating its closure to create a second
relation and adding the reverse of the tuple to the original relation
and calculating its transitive closure to create a third. The
properties of strict partial orders guarantee that the second and
third relations are also strict partial orders and that any two
members of the strict partial orders (besides the two in the tuple)
that were unrelated in the original order are unrelated in at least
one of the resulting orders.

As such, at least one of the resulting computation graphs will
correctly identify that two statements belong to unordered
nodes. By Lemma~\ref{lemma:concurrent-to-unordered}, the statements
are concurrent.
\end{proof}
