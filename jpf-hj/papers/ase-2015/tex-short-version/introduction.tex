\section{Introduction}
Despite the explosion in multi-core hardware for general purpose
computing, writing programs to take advantage of the available
processing power is often a task reserved for expert
developers. The first programs from the uninitiated often have
more in common with sequential execution than parallel performance due
to excessive synchronization, or worse, those programs are
fraught with concurrency errors due to an absence of needed
synchronization.

The Habanero extreme scale software research project intends to bring
multi-core programming to the masses through languages and frameworks
for non-experts. Habanero Java itself is a task-parallel programming model
built around lightweight asynchronous tasks and data transfers \cite{Cave:2011:HNA:2093157.2093165}. The
programmer in the Habanero framework focuses on the high-level task
constructs using simple annotations and
delegates to the Habanero run-time the burden of how to correctly and
efficiently implement and synchronize those constructs.

The Habanero programming model offers correctness guarantees by defining safe subsets of the
language that preserve properties over concurrent interactions such as
determinism, serialization, and deadlock freedom; however many of
these properties rely on the absence of data races. Regardless of
using a safe-subset of the language, there is no easy way to determine
when and if a program is free of data races. As such, the problem of verification
reduces in practice to \emph{printf}-debugging, inefficient code inspection, and
run-time failures.

Permission regions are program annotations that announce how a task
interacts with specific shared objects (i.e., reading or writing), and over what
region of code that interaction takes place \cite{Westbrook:2011:PRR:2341616.2341627}. During execution,
auxiliary data structures track accesses at the region on the indicated
variables and signal an error on any accesses that conflict with the permission annotations. Permission
regions have been shown effective in dynamically detecting data races
at run-time, while requiring only a small number of programmer annotations, for a Java implementation of Habanero (\hj)
\cite{Westbrook:2011:PRR:2341616.2341627,
  Westbrook:2012:PPR:2367163.2367201}.

This paper presents a sound model checking algorithm to prove a
program, for a given input, free of data races, deadlocks, failed assertions, and exceptions based on
permission regions.  The algorithm treats permission regions as atomic
blocks of read/write operations on shared memory to reduce the number
of schedules that must be considered in the proof. The algorithm also enumerates all
outcomes that arise from non-determinism in sequencing isolated atomic
blocks (i.e., non-determinism that is intended by the programmer) to
verify user defined assertions and exceptions. This paper includes a proof that the
algorithm is sound for any Habanero program with a fixed input. As
such, the cost of model checking a Habanero program is controlled with
the size and number of the permission regions, at the risk of
rejecting some programs that are actually free of data races.

The effectiveness of the algorithm is explored using a new
implementation of \hj\ in the form of a verification library (\hjv) and the Java
Pathfinder model checker (\jpf). \hjv\ is intended for debugging,
testing, and verification so it trades performance for simplicity and
correctness by using Java threads for each task, and using global
locks with conditions for features of Habanero that require mutual
exclusion and complex synchronization. The library supports all of the constructs in the Habanero model including phasers. The
implementation of permission regions with the new model checking
algorithm is in an extension to \jpf\ named
\jpfhj. An empirical study over
several benchmarks comparing the cost of verification between
\jpf\ and \jpfhj\ both using \hjv\ show a significant reduction in the
cost when using \jpfhj\ that is dependent on the size and number of
the permission regions with their interactions. The implementations of both \hjv\ and \jpfhj\ are available 
at \texttt{http://javapathfinder.org/jpf-hj/}


