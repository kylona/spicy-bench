\section{Habanero Java}

\begin{figure}
  \begin{center}
    \begin{lstlisting}[language=Java,escapechar=|,numbers=left,xleftmargin=18pt,numberstyle=\footnotesize]
public static void main(final String[] s) {
  Stack stk = initStack();
  
  launchHabaneroApp(() -> {
    finish(() -> { |\label{line:finish}|

      async(() -> { |\label{line:async}|
        stk.push(5);|\label{line:push}|
      });

      stk.peek();|\label{line:peek}|
    });
  });
}
\end{lstlisting}
  \end{center}
  \caption{An \hj\ program snipet using the \texttt{async} and \texttt{finish} statements.}
  \label{fig:hj-async-finish}
\end{figure}

The Habanero programming model is built around a task-parallel view of
concurrency. \figref{fig:hj-async-finish} is an \hj\ program using
Habanero's most basic task constructs: \emph{finish} and \emph{async}.
The \texttt{finish}-construct is a generalized join operation for
collective synchronization: the parent task executes and then waits
until all tasks created within the {\tt finish}-construct have completed
(including transitively created tasks). 

The \texttt{async}-construct is a mechanism for creating a new
asynchronous task: the calling task (parent) creates a new task
(child) to execute in parallel with the parent.  The child can read or
write any data in the heap and can read, but not write, any local
variable belonging to the parent's lexical scope. A task created in an
\texttt{async}-construct becomes ready for scheduling at the point it
is declared in the program.

The program in \figref{fig:hj-async-finish} enters a {\tt
  finish}-construct (\lineref{line:finish}) where it creates a child task
(\lineref{line:async}) to write to the stack (\lineref{line:push}). the
parent task then inspects the stack (\lineref{line:peek}). The two
stack operations are not ordered and execute logically in
parallel. The parent blocks at the end of the {\tt finish}-construct
until the child task completes.

Other constructs in the Habanero model include: \emph{isolated} and \emph{actors} for
mutual exclusion, \emph{future} for passing data between tasks, and
\emph{phasers} for arbitrary point-to-point synchronization
\cite{Cave:2011:HNA:2093157.2093165}.
