\section{Conclusions \& Future Work}
This paper presents an approach for the test and validation of
task-parallel languages using the Habanero programming model as an
example. The approach creates a specialized implementation of
Habanero that is purposed for validation. The implementation
is not only much simpler than performance oriented implementations, it
facilitates a conventional debugger in controlling the
scheduling order of concurrent tasks. More importantly, the
implementation lends itself to model checking to prove that an input
program is free of deadlock and data-race.
As model checking does not scale to larger programs, this paper
presents a sound algorithm for proving a program free of deadlock and
data-race that uses permission regions to mitigate state
explosion. The algorithm reduces the number of schedules it must
consider by only preempting at the entrance to permission regions. If
the regions are too large though, the algorithm may reject a program
that is actually free of deadlock and data-race. Since the model
checker provides a witness to any detected violation, it is possible to
manually validate the witness to refine the permission regions as needed. The
approach is illustrated with a full implementation in the \jpf\ model
checker and results on several input programs.

Future work includes automating the annotation of permission regions
based on the sharing detection in \jpf; automating the validation of
the counter-example to see if it is real or an artifact  of the
permission regions being too big; developing techniques that use the
counter-example to automatically refine permission regions; and
incorporating into the verification process static-analysis to prevent
scheduling on regions that statically cannot race.

