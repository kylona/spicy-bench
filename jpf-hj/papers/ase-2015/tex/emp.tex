\section{Empirical Study}
\begin{comment}
Briefly present (1 column) JPFs current strategy for managing
sharring. Present tables of results showing JPF performance with and
without permission regions. Include a table showing all the
benchmarks. Indicate the number of iterations and the number of
annotations.
\end{comment}

Currently, JPF inserts scheduling points dynamically as it
discovers sharing. JPF relies upon the placement of these points to
correctly detect race conditions. When a shared variable is discovered, JPF will
insert a scheduling point at the shared variable's location. This scheduling
point will cause JPF to interleave the execution of all runnable threads. If
JPF detects simultaneous read/write access of the shared variable it will notify
the user of a data race. However, if one of the conflicting threads is no longer
runnable (terminated, blocked, etc) then JPF will not report the error. Thus, in
some cases JPF's race detection strategy is incomplete. 

The use of gradual permission regions in verification of HJ programs decreases
the running time by reducing the number of scheduling points that JPF needs to
explore. Table \ref{table:1} compares the running time of example
programs from the Computer Science 322 class taught at Rice University. The
comparison is made between stand-alone JPF, JPF with race detection enable
through PreciseRaceDetector, and HJ-V with the program annotated with permission
regions.

% Please add the following required packages to your document preamble:
% \usepackage{multirow}
\begin{table}[h]
\centering
\caption{Benchmark Comparison of COMP 322 Programs}
\label{table:1}
\begin{tabular}{|c|c|c|c|}
\hline
\multirow{2}{*}{Name} & JPF    & JPF(PRD)   & HJ-V(GPR)   \\ \cline{2-4} 
                      & \multicolumn{3}{c|}{Time (mm:ss)} \\ \hline
Reciprocal Array Sum  & > (30:00) & > (30:00)  &             \\ \hline
For All With Iterable & > (30:00) & > (30:00)  & (01:06)     \\ \hline
Pipeline With Futures &           &            & (00:07) \\ \hline
Integer Counter Isolated &        &            & (00:47) \\ \hline
\end{tabular}
\end{table}

