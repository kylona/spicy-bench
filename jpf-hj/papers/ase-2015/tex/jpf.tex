\begin{algorithm}
\caption{Permission Region Informed Search}\label{algo:search}
\begin{algorithmic}[1]
  \Function{search}{$t$, $h$, $T$}
  \State \textbf{loop:}\ ($h$, $T$) $:=$ \texttt{run}($t$, $h$, $T$)\label{loc:run}
  \State
  \State \textit{s} $:=$ \texttt{status}($t$, $T$)
  \State \textit{data-race} = \textbf{false}
  \If{\textit{s} $=$ PR\_ENTRY} \label{loc:PR}
  \State ($h$, $T$, \textit{data-race}) $:=$ \texttt{acquire}($t$, $h$, $T$)\label{loc:acquire}
  \ElsIf{\textit{s} $=$ PR\_EXIT}
  \State ($h$, $T$) $:=$ \texttt{release}($t$, $h$, $T$)\label{loc:release}
  \State \textbf{goto} loop
  \EndIf
  \State
  \If{\textit{data-race}}\label{loc:datarace}
  \State report data-race and exit
  \EndIf
  \State
  \State $R = $ \texttt{runnable}($T$)
  \If{$R = \emptyset$}
  \If{blocked($T$) $\neq \emptyset$}\label{loc:deadlock}
  \State report deadlock and exit
  \Else
  \State report any detected sharing and exit\label{loc:term}
  \EndIf
  \EndIf
  \State
  \If{$(h,T) \not\in S$}\Comment{$S$ is a global variable}\label{loc:visited}
  \State $S = S \cup \{(h, T)\}$
  \If{\textit{s} $=$ PR\_ENTRY $\vee$ \textit{s} $=$ ISOLATED}
  \ForAll{$t_i \in R$}\label{loc:prsched}
  \State \texttt{search}($t_i$, $h$, $T$)
  \EndFor
  \Else
  \State $t_i$ := \texttt{random}($R$)\label{loc:rand}
  \State \texttt{search}($t_i$, $h$, $T$)
  \EndIf
  \EndIf
\EndFunction
\end{algorithmic}
\end{algorithm}

\begin{algorithm}
\caption{Procedure to Validate a Program}\label{algo:validate}
\begin{algorithmic}[l]
  \Procedure{validate}{$p$}
  \State ($h$, $T$) $:=$ init($p$)
  \State $R := $ runnable($T$)
  \State $t$ $:=$ random($R$)
  \State $S := \emptyset$
  \State search($t$, $h$, $T$)
  \While{\jpf\ reports sharing}
  \State Add permissions regions or isolation for sharing
  \State ($h$, $T$) $=$ init($p$)
  \State $S := \emptyset$
  \State search($t$, $h$, $T$)
  \EndWhile
  \EndProcedure
\end{algorithmic}
\end{algorithm}

\section{\jpfhj\ Search Algorithm}

The default model checking algorithm in \jpf\ is too fine grained to
scale beyond small programs even in the specialized
\hjv\ implementation. The cause of the state explosion is \jpf's
default scheduling algorithm that interleaves the execution of all
threads at every thread event and on every access to shared state. For
example \jpf\ schedules every thread that can run at every lock
acquire, lock release, thread block, thread unblock, field access on
shared objects, etc. to exhaustively enumerate the program schedule
space. As the \hjv\ library relies on locks to synchronize its
internal data structures, the fact that tasks are mapped directly to
threads, and the fact that often there are several byte-codes that
access an object when it is shared, the state explosion is severe.

Permission regions create natural scheduling boundaries for \jpf\ that
can be leveraged to mitigate state explosion while preserving the
essential behaviors of the program that lead to deadlock or
data-race. The intuition is that given a fixed program input, behavior
is only affected by interactions between tasks on shared memory. As
such, it is only necessary to preempt running threads at the
entrance to permission regions and \texttt{isolated}-constructs. If a program has deadlock or
data-race, such deadlock and data-race exists in one of the
schedules that is explored from those preemption points.

\algoref{algo:search} is the pseudo-code for the algorithm to explore all thread schedules
created at entry to permission regions and \texttt{isolated}-constructs. The state of the Java virtual
machine in the pseudo-code is simplified for clarity; it is
represented by a heap, $h$, and a set of threads, $T$. The lowercase
$t$ indicates a thread ID. \lineref{loc:run} updates the heap and pool
of threads by running thread $t$ until it blocks, exits, reaches a
permission region boundary (i.e., entry or exit), or reaches an \texttt{isolated}-construct.

At the entry point of the permission region, \lineref{loc:acquire}
updates the state machine for the acquired object in the heap and
checks to see if the acquisition signals a data-race. At the exit
point of the permission region, \lineref{loc:release} updates the
state machine for the released object in the heap, and the algorithm
restarts thread $t$ running anew at \lineref{loc:run}.

Data-race is reported on
\lineref{loc:datarace}. \lineref{loc:deadlock} reports deadlock. A
deadlock state is indicated when there are no runnable threads (i.e.,
$R = \emptyset$) and there exists threads that are blocked. A report
for either data-race or deadlock includes a witness trace for
validation and debugging. In the absence of deadlock or data-race, and
when there are simply no more threads to run, \lineref{loc:term} terminates the search
and reports any detected sharing that was not annotated by a permission region or covered by an \texttt{isolated}-construct.

The set $S$ on \lineref{loc:visited} is a global set to track the
visited states. \lineref{loc:prsched} does the actual scheduling by
considering all possible runnable threads, including the currently
running thread $t$, as a next thread to run. Note that in the current
state, if the thread $t$ was preempted because it entered a permission
region, then that state reflects the acquired permissions on that
region. In the case that thread $t$ has been blocked,
\lineref{loc:rand} chooses a random runnable thread to schedule next.

\figref{fig:permission-violation-state}, shown previously, is the state space explored by
the search algorithm for the simple example
\figref{fig:hj-async-finish-pr}. Recall that the example has two tasks
that access a shared stack: one reading and the other writing. The
ovals in the diagram represent scheduling points, and as before, the
blocks represent the state of the machine tracking permissions. As
indicated by the pseudo-code, the algorithm only preempts running
threads at the entrance to permission regions. In this example, it
schedules the child task after the main task acquires read permissions
to elicit the violation. By observation, if the permission regions in the annotated program in
\figref{fig:hj-async-finish-pr} were replaced with \texttt{isolated}-constructs, then the explored state space would no longer include the violation, but it would include another schedule that interleaves the atomic blocks defined by the isolated regions.

\algoref{algo:validate} is a procedural flow describing the process of
program validation using the new search in
\algoref{algo:search}. The process leverages \jpf's ability to track
thread IDs of accesses to heap locations. When the \algoref{algo:search}
finishes, \jpf\ reports any heap locations that have been accessed by
more than one distinct thread outside a permission region or an
\texttt{isolated}-construct with the input program location where that
access occurred. Using this information, a user is able to
appropriately annotate that program location, and then repeat the
search. The process terminates when a deadlock or data-race is
discovered, or no more sharing outside of permission regions or \texttt{isolated}-constructs exists.

\begin{theorem}
  The algorithm in \algoref{algo:search} is sound in that it only
  accepts programs that are deadlock and data-race free for programs
  that terminate and that have all sharing annotated with permission
  regions or wrapped in \texttt{isolated}-constructs.
\end{theorem}
\begin{proof}
The soundness proof reasons over a slightly modified version of the
algorithm that is iterative and takes as an additional input
a search tree which is similar to \figref{fig:permission-violation-state} that
captures all possible sequences of release and acquire statements
explored thus far. The algorithm traverses that input tree and at each
leaf node tries to extend that node by one generation if
possible. After the traversal, the algorithm returns the new tree. The
algorithm is called in an iterative manner until the tree reaches a
fix-point (which is guaranteed since the program terminates).

Let $P(n)$ be the statement that
this modified search algorithm returns all interesting sequences of acquire and release
statements of length $n$ or less for a given input program, where interesting means containing a deadlock or data-race.

\noindent\textbf{Basis Step:} the algorithm produces all interesting
sequences length $n \leq 1$. This case is trivially established with
the initial state of the program that represents a sequence of length
$n \leq 1$ and cannot contain a deadlock or data-race since the program has not yet
done anything. As such, it includes all interesting sequences.

\noindent\textbf{Inductive Step:} assume the modified algorithm has
correctly generated a tree representing all interesting sequences of $n$ or less;
it is necessary to show that from that tree it is able to generate all
interesting sequences of length $n+1$ or less.  There are three possible outcomes at any
leaf of the input tree:
\begin{compactenum}
\item the leaf cannot be extended because it is already an interesting sequence having a data-race or deadlock;
\item the leaf cannot be extended because there are no more runnable threads, in which case it is not interesting; or
\item the leaf is able to be extended with one or more immediate descendants.
\end{compactenum}
The first two cases are directly covered by lines 13 through 24 of the
algorithm; there is no way to have any descendants in those
situations and the sequences are already classified as interesting or not.

For the third case, first consider line 28 of the algorithm that creates
the next generation in the tree for permission regions and isolated
blocks. Every runnable thread is scheduled and each of those threads
must reach an immediate successor which may be a deadlock or data-race making an interesting sequence, a preemption, a
block condition, or exit by the constraint that the input program must
terminate. As such, any $n+1$ length sequence that exists, is
generated.

Further, any interesting $n+1$ sequence is generated as well. To see
this outcome, it is important to understand that the order of
acquisition relative to read or write does not matter in detecting a
violation. The state machine in \figref{fig:state-machine} is not
affected by acquisition order; it is only dependent on what read or write
permissions are held at the time of acquisition. As the algorithm always first acquires a
permission and then schedules other threads, it generates all the
interesting $n+1$ sequences if any exist that are interesting due to data-race. If data-race does not exist, then deadlock is detected as usual.

To complete the inductive step, line 32 must be considered, which covers
a blocked or exited thread. The input program has all sharing
annotated or isolated by constraint, meaning that any non-determinism
due to scheduling is covered already by line 28 so all reachable
program paths are considered. If an interesting sequence exists
because of deadlock, then it is either found in the $n+1$ step, by
having selected the correct thread, or in a later step when the
correct thread is chosen. If the deadlock depends on a particular sequence of
thread executions, then those sequences are enumerated by line 28. As
such, the deadlock is either deterministic (i.e., independent of the
schedule) or non-deterministic (i.e., a product of data-race on some
shared object). In the former, the choice of thread does not matter,
and in the latter, line 28 enumerates all possible orders over
isolated blocks and permission region blocks that do not data-race.
\end{proof}

\subsection{JPF Implementation}
The \jpf\ implementation of \algoref{algo:search} exploits another aspect
of the extensible nature of the tool by providing a new
\emph{scheduling-factory}.  A scheduling-factory is activated on
preemption, when a thread is no longer able to run, or if there is
input non-determinism.  \jpf\ uses scheduling-factories to decide what
threads are scheduled by having the scheduling factory insert
\emph{choice-generators} into the state search. The choice-generator
enumerates the available choices, and the search iterates over those
choices starting a new search for each choice.

The default scheduling-factory of \jpf\ is replaced with a new factory that
does not insert any choices on thread actions, locks, synchronization,
or shared access to objects. Anything related to concurrency is turned
off by the new scheduling-factory except for forced context switches
such as a thread exiting or a thread blocking. In those cases, the new
scheduling-factory inserts a choice generator with a single choice that
represents a random thread that is runnable.

To insert the preemption points for permission regions and
\texttt{isolated}-constructs, the byte-code listener from the
implementation of permission regions is extended to also listen for
the \texttt{INVOKE}-bytecode calls to \texttt{isolated}. At the
entrance to permission regions, the permission regions' state-machine for the
object is updated as before, but after the update, a choice-generator
is inserted into the search that includes choices for all runnable
threads. Similarly, a choice-generator is inserted at the
\texttt{isolated} call. The entire implementation is only a few
hundred lines of code but has a significantly reduces the verification
cost.
