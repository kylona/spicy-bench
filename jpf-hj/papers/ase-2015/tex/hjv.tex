\section{Habanero Java Implementation}

\begin{figure}
  \begin{center}
    \begin{lstlisting}
public static void main(final String[] argv) {
  Stack stk = initStack();
  
  launchHabaneroApp(() -> {
    finish(() -> {

      async(() -> {
        stk.push(5);
      });

      stk.peek();
    });
  });
}
\end{lstlisting}
  \end{center}
  \caption{An \hj\ program snippet using the \texttt{async} and \texttt{finish} statements and also showing how to start the Habanero environment.}
  \label{fig:hj-async-finish}
\end{figure}

\begin{figure}
  \begin{center}
    \begin{lstlisting}
public static double
parArraySumFutures(final double[] X) {
  final HjFuture<Double> sum1 = future(() -> {
    // Return sum of lower half of array
    double lowerSum = 0;
    for (int i = 0; i < X.length/2; i++) {
      lowerSum += 1 / X[i];}
    return lowerSum;});
  
  final HjFuture<Double> sum2 = future(() -> {
    // Return sum of upper half of array
    double upperSum = 0;
    for (int i = X.length/2; i < X.length; i++) {
      upperSum += 1 / X[i];}
	return upperSum;});
		
  // Combine sum1 and sum2
  final double sum = sum1.get() + sum2.get();
  return sum;
}
    \end{lstlisting}

  \end{center}
  \caption{\hj\ program snippet using the \texttt{future} statement to sum an array in parallel.}
  \label{fig:hj-future}
\end{figure}

\begin{figure}
  \begin{center}
    \begin{lstlisting}
 public static void main(String[] args) {
   launchHabaneroApp(() -> {
       finish(() -> {
         final HjPhaser ph = newPhaser(SIG_WAIT);
         HjPhaserPair mode(ph.inMode(SIG_WAIT));
         
         asyncPhased(mode, () -> {
           char[] buffer = bufferTwo;
           while (true) {
             produce(buffer);
             buffer = toggle(buffer);
             next();
           }                                
         });

         asyncPhased(mode, () ->  {
           char[] buffer = bufferOne;
           while (true) {
             consume(buffer);
             buffer = toggle(buffer);
             next();
           }
         });
       });
   });
 }
    \end{lstlisting}

  \end{center}
  \caption{An \hj\ program snippet using a phaser to synchronize a producer and consumer.}
  \label{fig:hj-phaser}
\end{figure}

\hjv\ is a Java library implementation of the Habanero model designed
specifically for test and validation. It consists of roughly 1,300
lines of code in 32 classes. Most of the classes address the
programmer interface rather than the library
internals. \figref{fig:hj-async-finish} is an example of the interface
using Java 8 lambdas. The interface in this
implementation is identical to other Java library implementations of
the Habanero model \cite{hj-lib}, so this library is interchangeable
with those libraries.

The \texttt{launchHabaneroApp} call is the entry point into the
library. Its parameter is a function that defines the Habanero program
to run.  The \texttt{finish} and \texttt{async} calls have their usual
meanings, and like the \texttt{launchHabaneroApp} call, they take
functions as their parameter. In the example in \figref{fig:hj-async-finish}, two tasks are created:
the main task at the launch, and a child task on the \texttt{async}
call. Both tasks interact with a shared stack \texttt{stk}. The finish
call does not complete until both tasks have completed.

The implementation of the \texttt{async} and \texttt{finish}
constructs uses Java threads and the ability to join those
threads. Flattening the lambda function in the Java 8 interface to an anonymous inner class elucidates the
structure of the library. That code for the \texttt{async} call in
\figref{fig:hj-async-finish} using an anonymous inner-class is shown below:
\begin{lstlisting}
  async(new HjRunnable() {
    public void run() {
      X.push(5);
    }
  });
\end{lstlisting}
The parameter for the call is an instance of an \texttt{HjRunnable}
object, and an \texttt{HjRunnable} is an extension to the standard Java
thread. The \texttt{run} method for the thread is specialized in the
anonymous inner-class. A programmer may use either syntax with \hjv: lambda or anonymous inner-class. 

Staying at a high-level view of the implementation, tasks are threads
with extra information to implement the Habanero model. To support the
\texttt{finish}-construct, that thread includes the notion of a
\emph{finish-scope}. A finish-scope holds references to any child
thread created within a \texttt{finish}-construct, and a stack of
finish-scopes tracks the nesting of \texttt{finish}-constructs within
a task. When a task is created, it is added to the current running
thread's active finish-scope. In this way, when a parent reaches the
end of a \texttt{finish}-construct, it is able to join on all threads
in the current finish-scope. After joining, the finish-scope is popped
from the stack making the next outer finish-scope the active scope.

The program in \figref{fig:hj-async-finish} has a somewhat
obvious data-race that is unsafe. Data-race intended by the programmer
is made safe, or protected, using the \texttt{isolated}-construct. For the referenced example program, the access to the \texttt{stk} object by the main task should be wrapped as follows:
\begin{lstlisting}
  isolated(() -> {
    X.peek();
  });
\end{lstlisting}
The access in the child task should be wrapped similarly. The two
access are now purely sequential being run in mutual exclusion. The
\texttt{isolated}-construct serializes atomic blocks of code relative
to other isolated blocks.

The implementation of the \texttt{isolated}-construct is
trivial. Every task has access to a \texttt{RunTime} object created
with the call to \texttt{launchHabaneroApp}. That object contains a
single lock that is used to force \texttt{isolated} constructs to be
sequential atomic statements. 

\figref{fig:hj-future} is an example of a program using
\texttt{future}-constructs. The program creates two future tasks to
sum the lower and upper halves of an array in parallel. The parent
task uses those future tasks, blocking until they complete, to combine
the two sums into a final result.

The implementation of the \texttt{future}-construct is again accomplished with
a thread. The thread is similar to the thread for the
\texttt{async}-construct only it has the added ability to
suspend. Suspension is made possible using a condition on a Java
lock. A caller to the \texttt{get} method blocks if the function
passed into the \texttt{future}-construct has not yet completed. Once
the function completes, the lock condition is signaled to free any threads blocked
in the associated \texttt{get} method.

The example in \figref{fig:hj-phaser} is more complex utilizing a
phaser to coordinate a producer task and consumer task. Unlike the
other constructs, the \texttt{asyncPhased} call takes two arguments:
the mode for the single participating phaser (or a list of modes for each of the
participating phasers), and a function defining the body of the
task. The phaser generates the objects to indicate modes using the
\texttt{inMode} method. There are 3 tasks in this example. The main
task creates and registers itself with the phaser in the
\texttt{SIG\_WAIT} mode. It then creates the producer and consumer
tasks. When the main task reaches the end of the \texttt{finish} call,
it automatically de-registers with the phaser and blocks for the
producer and consumer tasks. Each call to \texttt{next()} interacts
with every phaser passed in on the \texttt{asyncPhased} call, so all
phasers associated with the task synchronize at the same program location.

As before, the actual task interacting with the phaser is a Java
thread. Similar to how \texttt{finish}-constructs are implemented, the
extra information in the thread includes references to the phasers
for the task. The call to \texttt{next} iterates two times over the
phasers. The first iteration signals as appropriate, and the
second iteration waits as appropriate (possibly blocking along the way). The phaser itself
uses a Java lock with conditions to track threads that signal, threads
that wait, to block waiting threads as needed, and to keep track of the actual phase.

Other constructs in the Habanero programming model are implemented
similarly. The direct mapping between tasks and threads in the library
makes a conventional debugger more feasible for understanding
computation as well as for debugging deadlock and data-race. The debugger is able
to directly control the scheduling of concurrent threads on a single
processor, and because the run-time is relatively small, it is
possible to step through the run-time to understand its functionality
if needed. As an observation, there is a significant overhead with
creating threads for each task, the least of which impacts run-time
performance. The intent is to debug with \hjv\ and use the performance
oriented implementations for deployment.

Aside from conventional debugging, the library enables direct model
checking using \jpf. \jpf\ is an extensible implementation of a Java
virtual machine written in Java
\cite{Visser:2003:MCP:641151.641186}. Out of the box \jpf\ is able to
model check Java programs for exceptions, assertion violations,
deadlock, and data-race. Model checking brute-force explores all possible thread schedules and reports any violating schedule.  Although model checking is expensive, it
is effective for reasoning over schedules to harden
multi-threaded programs. The \hjv\ implementation has been
extensively model checked by \jpf\ using a set of test input
programs. The test input programs explore the different constructs in
the Habanero model in an effort to elicit unexpected behavior. The
model checking has been extremely helpful in finding bugs, removing
deadlock, and eliminating data-race both in the library implementation
itself and the input programs.

Model checking does not scale to larger programs, as expected, and
\jpf\ is no exception to that rule. A common approach to state
explosion though is to reduce the number of schedules that need to be
checked in model checking. This research leverages permission regions
to define atomic blocks so that \jpf\ does not context switch threads
at every access to shared memory. Rather, it is restricted to only
schedule at the entry points of permission regions and
\texttt{isolated}-constructs.
