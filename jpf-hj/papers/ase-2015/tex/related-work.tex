\section{Related Work}

There is an existing extension for JPF for the X10 Language
\cite{conf:icst:GligoricMM12,x10}. Habanero is closely related to X10 in many of
its constructs. In the extension, JPF operates directly on the actual X10
runtime system. To accomplish the integration, JPF is modified, the X10 runtime
is modified, the X10 compiler is extended, and a new static analysis is
presented to help control state explosion. The extension represents a
significant effort that affects all aspects of the X10 framework to enable JPF
verification. 

There is a formal model for the Chapel language with an accompanying model
checker that employs symbolic execution \cite{chapel}. The formal model is an
intermediate representation (IR) suitable for concurrent constructs. The
approach compiles Chapel programs into the IR and the model checker then
verifies the IR for deadlock and data-race freedom. Creating a compiler and
model checker is a significant undertaking beyond the approach in this paper.
More critically, the verification tool models the runtime including the number
of available worker threads to service tasks; thus, the verification results are
dependent on the number of worker threads in the configuration rather than the
semantics of the Chapel language. 

\begin{comment} In this paper, correctness is a property of the HJ language
semantics with the given HJ program and not any aspect of the runtime
implementation. Verifying the HJ runtime system implements HJ semantics is a
verification problem separate from verifying that an HJ program is data-race
free.  \end{comment}

Another approach to verifying concurrent languages is to leverage the production
level language runtime system itself
 \cite{Vakkalanka:2008:DVM:1427782.1427794,Vo:2009:FVP:1594835.1504214,5644885,6113841}.
These approaches typically require instrumentation of the source program,
wrappers to intercept calls into the runtime, and a way to control runtime
behavior. Although they are typically able to generate states faster than JPF,
verification results are dependent on the employed runtime correctly
implementing the language semantics. 

Many tools for dynamic race detection have been developed \cite{Eraser,
Eraser-Upgrade, Cilk-Dynamic}. These tools track the set of locks held by each
task during execution and use these sets to determine if a shared resource is
insufficiently protected. These tools produce results that are independent of
thread interleavings. This is an improvement as compared to previous tools that are
dependent upon the thread interleavings of the current
execution \cite{Lamport-Comparison, Mesa, Lamport-Online}. 
Race detection algorithms have also been developed for task-parallel
languages \cite{Async-Finish-Race, SP-BAGS}. These approaches utilize the
structured parallelism of the language to quickly detect races. However, the
results of this approach are also limited to a single execution.

Many different approaches have been developed to statically detect race
conditions in programs \cite{ESC, Warlock, RacerX, Relay}. Each of these techniques
require varying levels of instrumentation by the user. RacerX infers
the resources each lock protects, code contexts which are multithreaded, and
race conditions that have a "dangerous" effect upon the running program.
RacerX relies upon the user to annotate the location of the method for
performing the lock/unlock operations. Any other annotations by the user
are acceptable, but not required.

Relay performs a static lockset analysis. The Relay algorithm computes relative
locksets that belong to each function in the program. This bottom-up approach
scales very nicely, however, this approach remains unsound.

General permission regions (GPR) is another static analysis strategy
that infers the locations in which to place program annotations \cite{Westbrook:2012:PPR:2367163.2367201}. Once the annotations have been
statically injected, a dynamic analysis is run to detect the presence of race
conditions. Unlike, RacerX, GPR doesn't require any user annotations, although
it will honor any annotations introduced by the user. GPR correctly detects race
conditions for most common parallel programming paradigms. 

\begin{comment} Such a dependence can be avoided, as in this work, by creating a
non-performance oriented runtime that is simple enough to manually verify for
correctness \cite{Morse:2012:MAM:2189257.2189279}. It is much easier to access
the state and direct behavior, as a model checker, in such systems.
\end{comment}

Recent work proves the problem of state-reachability to be decidable and
EXPSPACE hard for finite-valued programs in languages such as X10/Habanero
\cite{Bouajjani:2012:ARP:2103621.2103681}. The result is limited to a subset of
the powerful task constructs in such languages and justifies a model checking
effort. The computability and complexity of the more advanced constructs such as
phasers is yet to be determined.
