\section{A Deterministic Fragment of the Model}
\label{sec:otf-drd}
Task parallel languages sometimes define fragments that impose restrictions to achieve determinism and the ability to serialize (e.g.,\cite{cave2011habanero}). Data-race detection using \algoref{algo:drd} on the computation graph from an single execution of a program in this fragment is both sound and complete. The fragment is defined by the following language restrictions:
\begin{compactitem}
\item Passing ownership of tasks from a parent to a child task is not allowed. 
\item Tasks whose return value handlers side effect can be posted in single-task regions only (i.e., regions that contain only a single task). A side-effect of a return value handler can be a change in the state of either the local variable or a region variable. 
\item All the tasks are joined to the main task at the end of the program execution. This is ensured by having the initial program configuration as \tuple{$T[\textbf{post} $ r_0 \leftarrow p_0~e~\varepsilon~\vec{r}~\vec{r}~\lambda$v.v; $\textbf{await}~r_0; \textbf{await}~r_1; \ldots], m_0} on some procedure $p_0$, $\vec{r}$ is the region sequence containing all regions and $\forall r \in \mathtt{Regs}, m_0(r) = \emptyset$
\end{compactitem}
\figref{fig:hj-async-fin} is an example from the equivalent fragment in the Habanero model to show the relationship to real-world programming languages. 

In Habanero, the \textbf{async} construct creates a new asynchronous task that runs in parallel with the parent task. The \textbf{finish} construct is used to collectively synchronize children tasks with their parent task. The \textbf{finish} $s$ statement causes the parent task to execute $s$ and then wait until all tasks created inside the finish-block have completed. The future construct lets tasks return values to other tasks with the operation \texttt{f.get()} that blocks until the task associated with $f$ completes. 

\begin{figure}
  \begin{center}
\begin{lstlisting}
public class Example1{
	static int x = 0;
	public static void main(String[] args) {
			finish {
				async { //Task1
					x = x + 1; }
				finish{
					async { //Task2
						x = x + 2;  } } }
			future f = async { //Task3
				return 5; }
			x = f.get(); } }
\end{lstlisting}
  \end{center}
  \vspace{-2em}
  \caption{An example of a Habanero Java Program.}
   \vspace{-2em}
  \label{fig:hj-async-fin}
\end{figure}

\begin{figure}
  \begin{center}
\begin{lstlisting}[mathescape=true]
  proc $main$ (var n : int)
  	$\texttt{l}(r_1) := 0;$
	post $r_1 \leftarrow p_1~0~\varepsilon~\vec{r}~\vec{r}~\lambda n.n$;
	post $r_2 \leftarrow p_2~0~\varepsilon~\vec{r}~\vec{r}~\lambda n.n$;
	await $r_2$;
	await $r_1$;
	post $r_3 \leftarrow p_2~0~\varepsilon~\vec{r}~\vec{r}~\lambda n. r_1 := n$;
	ewait $r_3$;	
  proc $p_1$ (var n : int)
  	$\texttt{l}(r_1) := \texttt{l}(r_1) + 1$
  proc $p_2$ (var n : int)
  	$\texttt{l}(r_1) := \texttt{l}(r_1) + 2$
  proc $p_3$ (var n : int)
  	return 5
\end{lstlisting}
  \end{center}
    \vspace{-2em}
  \caption{Converted version of the Habanero Java program from \figref{fig:hj-async-fin}.}
        \vspace{-1em}
  \label{fig:hj-async-fin-converted}
\end{figure}

\figref{fig:hj-async-fin-converted} is the equivalent program in the model in this presentation. The procedure $main$ posts task from the outer finish block to region $r_1$ and tasks from the inner finish block to region $r_2$. Since, the inner finish block completes execution first, \textbf{await} on region $r_2$ is called before $r_1$. The future posts a task to region $r_3$ followed by an \textbf{ewait} on $r_3$.

Let $\mathcal{G}( P )$ return the set of computation graphs from all possible schedules of the program $P$ from the deterministic fragment of the model, and let $\mathrm{DRF}( G )$ return true if \algoref{algo:drd} reports the graph to be data race free. 

\begin{lemma} 
\label{lem:drf}
 $(\exists G \in \mathcal{G}( P ),\ \mathrm{DRF}( G )) \rightarrow (\forall G \in \mathcal{G}( P ),\ \mathrm{DRF}( G ))$ 
\end{lemma}

The proof is omitted for space (as are the other non-trivial proofs), but the lemma states that if data-race is not detected in an observed execution of a program from the deterministic fragment, then all other possible executions are data-race free as well; in other words, programs from the fragment are deterministic.
Habanero makes this same claim but does not prove it \cite{cave2011habanero}. The corollary regarding data-race programs and the following theorem are trivial from \lemmaref{lem:drf}.

\begin{corollary}\label{cor:drf}
$(\exists G \in \mathcal{G}( P ),\ \neg\mathrm{DRF}( G )) \rightarrow (\forall G \in \neg\mathcal{G}( P ),\ \mathrm{DRF}( G ))$ 
\end{corollary}

\begin{theorem} \label{thm:strcutured-par-progs}
Using the tree semantics with Algorithm \ref{algo:drd} to detect data-race in the resulting computation graph is sound and complete for a task parallel program with a given input when that program is in the deterministic fragment of the language.
\end{theorem}
