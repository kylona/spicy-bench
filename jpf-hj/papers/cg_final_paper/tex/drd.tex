\section{Data Race Detection} \label{sec:drd}
A Computation graph for a task parallel program is a directed acyclic graph that represents the execution of the program \cite{dennis2012determinacy}. It is modified here to track memory locations accessed by tasks. 

\begin{definition}
\textbf{Computation Graph:} A Computation Graph 
$G = \tuple{N, E, \delta, \omega}$ of a task parallel program \textbf{P} with input $\psi$ is a directed acyclic graph where

\begin{itemize}
\item $N$ is a finite set of nodes
\item $E \subseteq N \times N$ is a set of directed edges. 
\item $\delta$ is the function that maps $N$ to the unique identifiers for the shared locations read by the tasks.
\begin{center}
$\delta : (N \mapsto 2^{V})$ 
\end{center}
\item $\omega$ is the function that maps $N$ to the unique identifiers for the shared locations written by the tasks.
\begin{center}
$\omega : (N \mapsto 2^{V})$ 
\end{center}
\end{itemize}
where V is the set of the unique identifiers for the shared locations.
\end{definition}

Fig. \ref{fig:cg} shows a sample computation graph. In this graph, nodes $n_0$, $n_0'$, $n_0''$, $r_1$, and $r_1'$ belong to task $t_0$. $t_0$ spawns two tasks $t_1$ and $t_2$. Node $n_1$ belongs to task $t_1$ and $n_2$ belongs to $t_2$. $r_1$ and $r_1'$ are join nodes for tasks $t_1$ and $t_2$. 

\begin{figure}
  \centering
        \includegraphics[width=0.3\textwidth]{../figs/Fig3-1.pdf}
    \caption{Computation Graph Example.}
    \label{fig:cg}
\end{figure}

Every node in the computation graph represents a block of sequential operations and edges order the nodes. The order between any two nodes $n_1$ and $n_2$ is given as $n_1 \prec n_2$, meaning that $n_1$ happens before $n_2$. Parallel nodes are un-ordered: $n_1 \nprec n_2$ and $n_2 \nprec n_1$. Once these un-ordered nodes are identified, the memory accessed by the operations performed in these nodes  can be checked to detect data races.  For the example in \figref{fig:cg}, nodes $n_1$ and $n_2$ are unordered and both are writing to variable $r_1$---a write-write, $\mathtt{ww}$, data race.

\begin{algorithm}
\caption{Data Race detection in a computation graph } \label{algo:drd}
\begin{algorithmic}[1]
\Function{DetectRace}{$Computation Graph \ G$}
\State N := Topologically ordered nodes in G \label{loc:topo}
\For {i in [1, $|N|$]}
\State $n = N[i]$
\For {j in [i+1, $|N|$]} 
\State $n' = N[j]$
\If {$ (n \nprec n') \land (n' \nprec n)$}  \label{loc:path} \label{loc:forall}
	\State \textbf{bool} $\mathtt{rw} = (\delta(n) \cap \omega(n') \neq \emptyset) $
	\State \textbf{bool} $\mathtt{wr} = (\omega(n) \cap \delta(n') \neq \emptyset) $
	\State \textbf{bool} $\mathtt{ww} = (\omega(n) \cap \omega(n') \neq \emptyset)$
		\If {$( \mathtt{rw} \lor \mathtt{wr}  \lor \mathtt{ww} )$} \label{loc:intersection}
			\State \textbf{Report Data Race and Exit} \label{loc:datarace}
		\EndIf
\EndIf
 \EndFor
 \EndFor
\EndFunction  
\end{algorithmic}
\end{algorithm}

\algoref{algo:drd} detects data races in a computation graph. The nodes in the computation graph are added to a topologically sorted set on \lineref{loc:topo}. The $i^{th}$ node in the set is given by $N[i]$. The nodes are traversed in order and each node is compared to every node that comes later in the topological ordering. \lineref{loc:path} checks if the nodes $n$ and $n'$ are un-ordered. If the nodes are un-ordered, then the sets of memory locations accessed by each node are checked for conflict on \lineref{loc:intersection}. If any of the sets shares an element, then there is a data race. The algorithm checks each node until either a data race is reported or all the nodes have been verified. The algorithm runs in $\mathcal{O}$(N$^2$) time for the number of nodes in the computation graph.
%by virtue of the cost of the topological ordering of nodes. 

\begin{comment}
: $\mathcal{O}$(N$^2$). When nodes are topologically ordered, reachability of nodes can be checked in $\mathcal{O}$(N) time. Therefore, the time required to check if two nodes are executing in parallel is $\mathcal{O}$(N$^2$). The time required to check the intersection of read or write sets of shared locations is $\mathcal{O}$($m_1 +  m_2$) where $m_1$ and $m_2$ are the sizes of the two sets. ($m_1 + m_2$) is much smaller than N. Therefore, the time complexity of Algorithm \ref{algo:drd}  is $\mathcal{O}$(N$^2$).
\end{comment}

\begin{comment}
A data race detection algorithm is sound if it does not miss any data race in a program for a given input (e.g., it may under-approximate the set of data race free programs), and it is complete if it does not report data races in programs that are data race free (e.g., it may over-approximate the set of data race free programs.
\end{comment}

\begin{theorem} \label{thm:graph}
Algorithm \ref{algo:drd} is sound and complete; it neither under-approximates nor over-approximates the set of data-race free programs.
\end{theorem}

\begin{comment}
\begin{proof}
The computation graph is a directed acyclic graph. The transitive closure of the graph gives the reachibility relationship of the tasks. The transitive closure is a strict partial order over the nodes of the graph. The data race detection algorithm checks if nodes $n$ and $n^\prime$ in the graph are unordered on Line \ref{loc:forall}. The statements may be executed in parallel by these nodes. The memory accessed by these tasks is compared and a race is reported if a conflict is detected on Line \ref{loc:datarace}. Therefore, when algorithm \ref{algo:drd} declares a computation graph to be data race free, no race can exist in that graph and when a race is reported by the algorithm, there definitely exists two tasks that execute in parallel and have conflicting accesses to a shared variable. Hence, Algorithm \ref{algo:drd} is sound and complete for a given computation graph.
\end{proof}
\end{comment}
