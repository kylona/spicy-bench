Task parallel programming languages provide a way for creating asynchronous tasks that can run concurrently. The advantage of using task parallelism is that the programmer can write code that is independent of the underlying hardware. The runtime determines the number of processor cores that are available and the most efficient way to execute the tasks. When two or more concurrently executing tasks access a shared memory location and if at least one of the accesses is for writing, data race is observed in the program. Data races can introduce non-determinism in the program output making it important to have data race detection tools. To detect data races in task parallel programs, a new sound and complete technique based on computation graphs is presented in this work. The data race detection algorithm runs in $\mathcal{O}$(N$^2$) time where N is number of nodes in the graph. A computation graph is a directed acyclic graph that represents the execution of the program. For detecting data races, the computation graph stores shared heap locations accessed by the tasks. An algorithm for creating computation graphs augmented with memory locations accessed by the tasks is also described here. This algorithm runs in $\mathcal{O}$(N) time where N is the number of operations performed in the tasks. A scheduling algorithm that creates all possible computation graph structures for programs containing critical sections is also presented here. This work also presents an implementation of this technique for the Java implementation of the Habanero programming model. The results of this data race detector are compared to Java Pathfinder's precise race detector extension and permission regions based race detector extension. The results show a significant reduction in the time required for data race detection using this technique.