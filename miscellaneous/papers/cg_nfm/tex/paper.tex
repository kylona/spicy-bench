%% bare_conf.tex
%% V1.4a
%% 2014/09/17
%% by Michael Shell
%% See:
%% http://www.michaelshell.org/
%% for current contact information.
%%
%% This is a skeleton file demonstrating the use of IEEEtran.cls
%% (requires IEEEtran.cls version 1.8a or later) with an IEEE
%% conference paper.
%%
%% Support sites:
%% http://www.michaelshell.org/tex/ieeetran/
%% http://www.ctan.org/tex-archive/macros/latex/contrib/IEEEtran/
%% and
%% http://www.ieee.org/

%%*************************************************************************
%% Legal Notice:
%% This code is offered as-is without any warranty either expressed or
%% implied; without even the implied warranty of MERCHANTABILITY or
%% FITNESS FOR A PARTICULAR PURPOSE! 
%% User assumes all risk.
%% In no event shall IEEE or any contributor to this code be liable for
%% any damages or losses, including, but not limited to, incidental,
%% consequential, or any other damages, resulting from the use or misuse
%% of any information contained here.
%%
%% All comments are the opinions of their respective authors and are not
%% necessarily endorsed by the IEEE.
%%
%% This work is distributed under the LaTeX Project Public License (LPPL)
%% ( http://www.latex-project.org/ ) version 1.3, and may be freely used,
%% distributed and modified. A copy of the LPPL, version 1.3, is included
%% in the base LaTeX documentation of all distributions of LaTeX released 
%% 2003/12/01 or later.
%% Retain all contribution notices and credits.
%% ** Modified files should be clearly indicated as such, including  **
%% ** renaming them and changing author support contact information. **
%%
%% File list of work: IEEEtran.cls, IEEEtran_HOWTO.pdf, bare_adv.tex,
%%                    bare_conf.tex, bare_jrnl.tex, bare_conf_compsoc.tex,
%%                    bare_jrnl_compsoc.tex, bare_jrnl_transmag.tex	
%%*************************************************************************

%\documentclass[conference]{IEEEtran}

%\let\oldvec\vec
\RequirePackage{amsmath}
\let\oldvec\vec
\documentclass{llncs}
\let\vec\oldvec

\hyphenation{op-tical net-works semi-conduc-tor}


\setlength{\paperheight}{11in}
\setlength{\paperwidth}{8.5in}

\let\proof\relax
\let\endproof\relax

%\usepackage{amsmath, listings, amsthm, amssymb, proof, xspace, stmaryrd, times}
\usepackage{listings, amsthm, amssymb, proof, xspace, stmaryrd, times}

\usepackage[pass]{geometry}
\usepackage{verbatim}
\usepackage{url}
\usepackage{color}
\usepackage{multicol}
\usepackage{paralist}
\usepackage{comment}
\let\oldvec\vec
\usepackage{graphicx}
\usepackage{epstopdf}

%\let\vec\oldvec
%\let\proof\relax
%\let\endproof\relax

\usepackage{subfigure}
\usepackage{cite}
\usepackage[nounderscore]{syntax}
\usepackage{newfloat}
\usepackage{esvect}
\usepackage{semantic}
\usepackage{mathpartir}
\usepackage[table]{xcolor}
\usepackage{flushend}
\pdfoptionpdfminorversion=7


\DeclareFloatingEnvironment[
  % the file extension for the file used to create the list:
  fileext   = logr,% don't use log here!
  % the heading for the list:
  listname  = {List of Grammars},
  % the name used in captions:
  name      = Grammar,
  % the default floating parameters if the environment is used
  % without optional argument:
  placement = htp
]{Grammar}

\newcommand{\figref}[1]{Fig.~\ref{#1}}
\newcommand{\lemmaref}[1]{Lemma~\ref{#1}}
\newcommand{\tableref}[1]{Table~\ref{#1}}
\newcommand{\secref}[1]{Section~\ref{#1}}
\newcommand{\lineref}[1]{Line~\ref{#1}}
\newcommand{\corref}[1]{Corollary~\ref{#1}}
\newcommand{\thmref}[1]{Theorem~\ref{#1}}
\newcommand{\algoref}[1]{Algorithm~\ref{#1}}


\newcommand{\setof}[1]{\ensuremath{\left \{ #1 \right \}}}
\newcommand{\tuple}[1]{\ensuremath{\langle #1 \rangle }}

%\newcommand{\mystar}{{\fontfamily{lmr}\selectfont$\star$}}

\definecolor{dkgreen}{rgb}{0,0.6,0}
%\definecolor{gray}{rgb}{0.5,0.5,0.5}
\definecolor{mauve}{rgb}{0.58,0,0.82}
\definecolor{light-gray}{gray}{0.88}

\lstset{frame=tb,
  language=Java,
  aboveskip=3mm,
  belowskip=3mm,
  showstringspaces=false,
  columns=flexible,
  numbers=none,
  numberstyle=\tiny\color{gray},
  commentstyle=\color{dkgreen},
  stringstyle=\color{mauve},
  breaklines=true,
  breakatwhitespace=true,
  tabsize=3,
  morekeywords={proc, post, await, ewait, var, skip, assume, call, return, async, finish, future, isolated, forall}
}

%\newtheorem{definition}{Definition}
%\newtheorem{theorem}{Theorem}
%\newtheorem{lemma}{Lemma}
%\newtheorem{corollary}{Corollary}

\usepackage{algorithm, algpseudocode}

\input{defs.tex}

\begin{document}

\title{Model-checking Task Parallel Programs for Data-race}

% author names and affiliations
% use a multiple column layout for up to three different
% affiliations

\begin{comment}
\author{{Radha Nakade}
{Department of Computer Science\\
Brigham Young University\\
Provo, Utah\\
Email: radha.nakade@gmail.com}
\and
{Eric Mercer}
{Department of Computer Science\\
Brigham Young University\\
Provo, Utah\\
Email: egm@cs.byu.edu}
\and
{Jay McCarthy}
{Department of Computer Science\\
University of Massachusetts Lowell\\
Email: jay.mccarthy@gmail.com}}
\end{comment}

\author{Radha Nakade\inst{1} \and
 Eric Mercer\inst{1} \and
Jay McCarthy\inst{2} }

\institute{Brigham Young University, Provo, Utah
\and
University of Massachusetts Lowell}
% make the title area
\maketitle

% As a general rule, do not put math, special symbols or citations
% in the abstract
\begin{abstract}
Data-race detection is the problem of determining if a concurrent program has a data-race in some execution and input; it has been long studied and often solved for different contexts and goals. The research in this paper reprises the problem of data-race detection but in the context of model checking as opposed to run-time monitoring or static analysis. The programming model is task parallel where computations are hierarchically divided into non-isolated parallel executing tasks and is suitable to describing real-world languages (i.e., Cilk, X10, Chapel, Habanero, etc.). The model semantics are defined to construct a computation graph and a naive algorithm to detect data-race on a computation graph is given. A fragment of the programming model is defined such that the algorithm becomes sound and complete for computation graphs from a single observed program execution. Model checking is applied to programs outside this fragment to enumerate all possible computation graphs. The approach is evaluated in a Java implementation of Habanero using the JavaPathfinder model checker. The results, when compared to existing data-race detectors in Java Pathfinder, show a significant reduction in the time required for data race detection.

\begin{comment}
Task parallel programming languages provide a way for creating asynchronous tasks that can run concurrently. The advantage of using task parallelism is that the programmer can write code that is independent of the underlying hardware. The runtime determines the number of processor cores that are available and the most efficient way to execute the tasks. When two or more concurrently executing tasks access a shared memory location and if at least one of the accesses is for writing, data race is observed in the program. Data races can introduce non-determinism in the program output making it important to have data race detection tools. To detect data races in task parallel programs, a new sound and complete technique based on computation graphs is presented in this work. The data race detection algorithm runs in $\mathcal{O}$(N$^2$) time where N is number of nodes in the graph. A computation graph is a directed acyclic graph that represents the execution of the program. For detecting data races, the computation graph stores shared heap locations accessed by the tasks. An algorithm for creating computation graphs augmented with memory locations accessed by the tasks is also described here. This algorithm runs in $\mathcal{O}$(N) time where N is the number of operations performed in the tasks. A scheduling algorithm that creates all possible computation graph structures for programs containing critical sections is also presented here. This work also presents an implementation of this technique for the Java implementation of the Habanero programming model. The results of this data race detector are compared to Java Pathfinder's precise race detector extension and permission regions based race detector extension. The results show a significant reduction in the time required for data race detection using this technique.

Cut from current abstract

A data-race is where two concurrent executions access the same memory location with at least one of the two accesses being a write. A data-race introduces non-determinism in program execution complicating both test and debug since the programmer is not able to directly control or observe the internals of the the execution run-time. The computation graph captures the partial-order between tasks with memory references. 
\end{comment}

\end{abstract}

%\IEEEpeerreviewmaketitle

\section{Introduction}
A \emph{data-race} is where two concurrent executions access the same memory location with at least one of the two accesses being a write. It introduces non-determinism into the program execution as the behavior may depend on the order in which the concurrent executions access memory. Data-race is problematic because it is not possible to directly control or observe the run-time internals to know if a data-race exists let alone enumerate program behaviors when one does. 

The problem of \emph{Data-race detection}, given a program with its input, is to determine if there exists an execution containing a data-race. The research presented in this paper is concerned with data-race detection for \emph{task parallel models} that impose structure on parallelism by constraining how threads are created and joined and how shared memory is accessed (e.g., Cilk, X10, Chapel, Habanero, etc.). These models rely on run-time environments to implement task abstractions to represent concurrent executions \cite{blumofe1996cilk,charles2005x10,cave2011habanero,imam2014habanero}. The language restrictions on parallelism and shared memory interactions enable properties like \emph{determinism} (i.e., the computation is independent of the execution) or the ability to \emph{serialize} (i.e., removing all task related keywords yields a serial solution). Such properties are predicated on the input programs being data-race free which is not always the case since programmers both intentionally and unintentionally move outside the programming model.

Data-race detection in task parallel models generally prioritizes performance and the ability to scale to many tasks. The predominant \emph{SP-bags} algorithm, with its variants, exploits assumptions on task creation and joining for efficient on-the-fly detection with low overhead \cite{Feng1997EDD258492258493,Cheng1998DDR277651277696,Bender2004OMS10079121007933,Async-Finish-Race,Utterback2016PGP29357642935801}; millions of task are feasible with varying degrees of slow-down (i.e., slow-down increases as parallelism constraints are relaxed) \cite{drdForFutures,Surendran2016}. Other approaches use access histories \cite{mellor1991fly,raman2012scalable} or programmer annotations \cite{westbrook2012practical, westbrook2012permission}.
Performance is a priority, and many solutions are only \emph{complete}, meaning that nothing can be concluded about other executions of the same program, or become complete when parallelism is outside the assumptions.

The research presented in this paper reprises data-race detection in task parallel models in the context of model checking under two assumptions: first, data-race is largely independent of the size of the problem instance; and second, it is possible to instantiate small problem instances. These assumptions are indeed implicit in other model checking solutions for task parallel models---a small problem instance is the best solution to state explosion \cite{gligoric2012x10x,zirkel2013automated,anderson2014jpf}. Prior approaches, however, extensively modify the language run-time and the model checker necessitating source code for both \cite{gligoric2012x10x}, require the user to specify the number of processors modeled making it difficult to generalize \cite{zirkel2013automated}, or rely on user annotations to indicate sharing so data-race may be missed \cite{anderson2014jpf}. The solution here is to make clear the requirements on the run-time, use semantics that are independent of actual hardware, and automatically detect data-race without annotations. 

The approach first defines a \emph{computation graph} to abstractly model parallelism with a naive algorithm to detect data-race. Computation graph construction is then formally defined in a general task parallel model based on partitioning concurrent executions into hierarchical regions with shared locations. Such a model is suitable to describe real-world languages (e.g., Cilk, X10, Chapel, Habanero, etc.).  A fragment of the model is then defined so that data-race detection on a computation graph from a single execution is both sound and complete; this result is similar to existing dynamic analyses for task parallel languages. That fragment is then expanded to show how model checking may be applied to enumerate the space of computation graphs for data-race detection. Finally, the approach is evaluated on a Java implementation of Habenero with Java Pathfinder (JPF). Results over several published benchmarks comparing to JPF's default race detection and a task parallel approach with permission regions show the computation graph to be more efficient in JPF terms with its overhead. The primary contributions are thus
\begin{compactitem}
\item the computation graph construction in terms of general semantics suitable for real-world languages;
\item model checking as a means to exploring the space of computation graphs for a program; and
\item an implementation of the approach for Java Habanero in JPF with results from benchmarks comparing to other solutions in JPF. 
\end{compactitem}
\begin{comment}
Section \ref{sec:drd} defines computation graphs and data-race detection given a computation graph. Section \ref{sec:cg} is the programming model with graph construction. Section \ref{sec:otf-drd} is the model checking algorithm. Section \ref{sec:impl} gives an implementation of the algorithm for Habanero and section \ref{sec:res} discusses the results. Section \ref{sec:rel-work} discusses related work. Section \ref{sec:conclusion} presents the conclusion.
\end{comment}
\begin{comment}
  The increasing use of multi-core processors is motivating parallel programming. Earlier, the speed of processor cores was expected to increase with sustained technological advances and the need for parallel computing was low. Now that processor speeds are no longer increasing, parallelism is the only way of obtaining higher computing performance.

Writing concurrent programs that are free from bugs, however, is very difficult because when programs execute different instructions simultaneously, different thread schedules and memory access patterns are observed that give rise to issues such as data races and deadlocks. Structured parallel languages help users to write parallel programs that are scalable and easy to maintain \cite{blumofe1996cilk, charles2005x10, cave2011habanero}. These properties are achieved by imposing restrictions on the way tasks can be forked and joined.

Data races occur in parallel programs when two or more tasks access a shared memory location such that at least one of the accesses is a write. A race on a shared variable can alter the value of the variable based on the order in which the variable is accessed by the tasks causing the output to be non-deterministic. A data race that is not protected (i.e., marked volatile) also leads to behavior that is not sequentially consistent. It is hard to test all possible outcomes of the program with a data race because the scheduler most often does the same thing. 

A lot of research has gone into the problem of detecting data races in parallel programs. Data race detection techniques are mainly categorized as static, dynamic and model checking. Static race detectors analyze the programs statically and report errors \cite{engler2003racerx,ESC,abadi2006types,naik2006effective,voung2007relay,choi2001static, vechev2011automatic}. Their drawback is that they report data races on variables when in fact there are no data-races; this imprecision makes them hard to use in real world applications. Model checking on the other hand produces precise results but suffers from state space explosion making it impossible to use in large systems\cite{kulikov2010detecting, vakkalanka2008implementing, Godefroid, anderson2014jpf, gligoric2012x10x, zirkel2013automated}.

Dynamic data race detectors analyze the program at runtime and so the data races reported by them are real data races. Dynamic data race detectors however can reason about only a single run \cite{flanagan2009fasttrack, savage1997eraser, mellor1991fly, schonberg1989fly, Feng97efficientdetection, Async-Finish-Race}. Raman et al. created a dynamic race detector for structured parallel programs that can locate races in any schedule of the program by running the program only once using limited access history\cite{raman2012scalable}. Another approach for data race detection for programs with futures uses dynamic task reachibility graphs \cite{drdForFutures}. Both these approaches are efficient, however, they don't provide any functionality to manipulate the scheduler at runtime thereby being unsound for programs with mutual exclusion. When accesses to shared variables are protected, different program outcomes are observed. It is necessary to analyze all program behaviors to ensure data race freedom.

This paper introduces an improved technique for data race detection that combines dynamic race detection for structured parallel languages with model checking to overcome the limitations of both of them. This technique makes use of computation graphs to represent the happens-before relation of the events of the program in the form of a directed acyclic graph \cite{dennis2012determinacy}. The nodes represent the various tasks that are spawned during the program execution and store the references to shared heap locations that have been accessed by those tasks. To detect data races, the task nodes that can execute in parallel are identified in the graph and the memory locations stored in these nodes are compared to detect conflicts. For building such computation graphs, the runtime should have the ability to call-back when threads are forked or joined, and to record memory accesses on heap locations that may be shared.

The model checking part of the solution comes into play for programs with critical sections. In such programs, different computation graph structures can arise based on the order of execution of the critical sections. The technique presented here creates all such computation graphs using a scheduler that checks for critical sections and builds schedules to consider all possible ways in which the program can execute \cite{mercer2015model}. Hence, this method is sound for all programs with a given input.  

This paper presents an implementation of this data race detection technique for the Java implementation of the Habanero programming model. The implementation uses Java Pathfinder (JPF). JPF is a model checker for Java that is fully customizable using various programming patterns and interfaces. The implementation uses JPF's virtual machine to create a specialized runtime for the Habanero language that is targeted specifically for verification\cite{mercer2015model, anderson2014jpf}. 
The performance is compared with two other model checking approaches implemented by JPF: Precise Race Detector and Permission regions \cite{kulikov2010detecting}, \cite{mercer2015model}. The results show a significant reduction in the state space and time needed for verification.

\textbf{Main Contributions:}
  \vspace{-1em}
\begin{enumerate}
\item A data race detection algorithm using computation graphs that runs in $\mathcal{O}$(N$^2$) time where N is number of nodes in the graph.
\item Semantics for task parallel programs that include steps for creating computation graphs.
\item Dynamic improvement to the data race detection algorithm for structured parallel programs.
\item A scheduling algorithm to create all computation graphs for programs containing mutual exclusion.
\item An implementation of the data race detection algorithm for Habanero Java.
\item An empirical study over a set of benchmarks comparing performance of the data race detection algorithm to JPF.
\end{enumerate}
Some proofs are omitted for space but exist in a long technical report to be referenced if the paper is accepted.
\end{comment}

\begin{comment}
  The rest of the paper is divided as follows. Section \ref{sec:drd} introduces the concept of computation graphs for task parallel programs and discusses the data race detection algorithm based on computation graphs. Section \ref{sec:cg} presents syntax and semantics of task parallel languages and discusses the creation of computation graphs. Section \ref{sec:otf-drd} discusses dynamic improvement to the data race detection algorithm using on-the-fly analysis for structured parallel programs and a scheduler for programs with critical sections. Section \ref{sec:impl} gives implementation of the algorithm for HJ and section \ref{sec:res} discusses the results. Section \ref{sec:rel-work} discusses related work. Section \ref{sec:conclusion} presents the conclusion.
\end{comment}

\section{Data Race Detection} \label{sec:drd}
A Computation graph for a task parallel program is a directed acyclic graph that represents the execution of the program \cite{dennis2012determinacy}. It is modified here to track memory locations accessed by tasks. 

\begin{definition}
\textbf{Computation Graph:} A Computation Graph 
$G = \tuple{N, E, \delta, \omega}$ of a task parallel program \textbf{P} with input $\psi$ is a directed acyclic graph where

\begin{itemize}
\item $N$ is a finite set of nodes
\item $E \subseteq N \times N$ is a set of directed edges. 
\item $\delta$ is the function that maps $N$ to the unique identifiers for the shared locations read by the tasks.
\begin{center}
$\delta : (N \mapsto 2^{V})$ 
\end{center}
\item $\omega$ is the function that maps $N$ to the unique identifiers for the shared locations written by the tasks.
\begin{center}
$\omega : (N \mapsto 2^{V})$ 
\end{center}
\end{itemize}
where V is the set of the unique identifiers for the shared locations.
\end{definition}

Fig. \ref{fig:cg} shows a sample computation graph. In this graph, nodes $n_0$, $n_0'$, $n_0''$, $r_1$, and $r_1'$ belong to task $t_0$. $t_0$ spawns two tasks $t_1$ and $t_2$. Node $n_1$ belongs to task $t_1$ and $n_2$ belongs to $t_2$. $r_1$ and $r_1'$ are join nodes for tasks $t_1$ and $t_2$. 

\begin{figure}
  \centering
        \includegraphics[width=0.3\textwidth]{../figs/Fig3-1.pdf}
    \caption{Computation Graph Example.}
    \label{fig:cg}
\end{figure}

Every node in the computation graph represents a block of sequential operations and edges order the nodes. The order between any two nodes $n_1$ and $n_2$ is given as $n_1 \prec n_2$, meaning that $n_1$ happens before $n_2$. Parallel nodes are un-ordered: $n_1 \nprec n_2$ and $n_2 \nprec n_1$. Once these un-ordered nodes are identified, the memory accessed by the operations performed in these nodes  can be checked to detect data races.  For the example in \figref{fig:cg}, nodes $n_1$ and $n_2$ are unordered and both are writing to variable $r_1$---a write-write, $\mathtt{ww}$, data race.

\begin{algorithm}
\caption{Data Race detection in a computation graph } \label{algo:drd}
\begin{algorithmic}[1]
\Function{DetectRace}{$Computation Graph \ G$}
\State N := Topologically ordered nodes in G \label{loc:topo}
\For {i in [1, $|N|$]}
\State $n = N[i]$
\For {j in [i+1, $|N|$]} 
\State $n' = N[j]$
\If {$ (n \nprec n') \land (n' \nprec n)$}  \label{loc:path} \label{loc:forall}
	\State \textbf{bool} $\mathtt{rw} = (\delta(n) \cap \omega(n') \neq \emptyset) $
	\State \textbf{bool} $\mathtt{wr} = (\omega(n) \cap \delta(n') \neq \emptyset) $
	\State \textbf{bool} $\mathtt{ww} = (\omega(n) \cap \omega(n') \neq \emptyset)$
		\If {$( \mathtt{rw} \lor \mathtt{wr}  \lor \mathtt{ww} )$} \label{loc:intersection}
			\State \textbf{Report Data Race and Exit} \label{loc:datarace}
		\EndIf
\EndIf
 \EndFor
 \EndFor
\EndFunction  
\end{algorithmic}
\end{algorithm}

\algoref{algo:drd} detects data races in a computation graph. The nodes in the computation graph are added to a topologically sorted set on \lineref{loc:topo}. The $i^{th}$ node in the set is given by $N[i]$. The nodes are traversed in order and each node is compared to every node that comes later in the topological ordering. \lineref{loc:path} checks if the nodes $n$ and $n'$ are un-ordered. If the nodes are un-ordered, then the sets of memory locations accessed by each node are checked for conflict on \lineref{loc:intersection}. If any of the sets shares an element, then there is a data race. The algorithm checks each node until either a data race is reported or all the nodes have been verified. The algorithm runs in $\mathcal{O}$(N$^2$) time for the number of nodes in the computation graph.
%by virtue of the cost of the topological ordering of nodes. 

\begin{comment}
: $\mathcal{O}$(N$^2$). When nodes are topologically ordered, reachability of nodes can be checked in $\mathcal{O}$(N) time. Therefore, the time required to check if two nodes are executing in parallel is $\mathcal{O}$(N$^2$). The time required to check the intersection of read or write sets of shared locations is $\mathcal{O}$($m_1 +  m_2$) where $m_1$ and $m_2$ are the sizes of the two sets. ($m_1 + m_2$) is much smaller than N. Therefore, the time complexity of Algorithm \ref{algo:drd}  is $\mathcal{O}$(N$^2$).
\end{comment}

\begin{comment}
A data race detection algorithm is sound if it does not miss any data race in a program for a given input (e.g., it may under-approximate the set of data race free programs), and it is complete if it does not report data races in programs that are data race free (e.g., it may over-approximate the set of data race free programs.
\end{comment}

\begin{theorem} \label{thm:graph}
Algorithm \ref{algo:drd} is sound and complete; it neither under-approximates nor over-approximates the set of data-race free programs.
\end{theorem}

\begin{comment}
\begin{proof}
The computation graph is a directed acyclic graph. The transitive closure of the graph gives the reachibility relationship of the tasks. The transitive closure is a strict partial order over the nodes of the graph. The data race detection algorithm checks if nodes $n$ and $n^\prime$ in the graph are unordered on Line \ref{loc:forall}. The statements may be executed in parallel by these nodes. The memory accessed by these tasks is compared and a race is reported if a conflict is detected on Line \ref{loc:datarace}. Therefore, when algorithm \ref{algo:drd} declares a computation graph to be data race free, no race can exist in that graph and when a race is reported by the algorithm, there definitely exists two tasks that execute in parallel and have conflicting accesses to a shared variable. Hence, Algorithm \ref{algo:drd} is sound and complete for a given computation graph.
\end{proof}
\end{comment}

Bouajjani and Emmi created a formal model of isolated hierarchical parallel computation that covers many existing task parallel languages (e.g., Cilk, X10, Chapel, Habanero, etc.) \cite{bouajjani}. Real world task parallel models are not isolated so tasks may share memory (intentionally or otherwise). This paper uses the formalism of Bouajjani and Emmi to define the construction of the computation graph from program execution but adds global variables. As before, a \emph{region} groups tasks by storing task handles, but now each region also holds a variable that can be shared. Tasks are expanded to include access lists to denote region variables available for reading or writing.  

\section{Surface Syntax}

\begin{figure}
  \begin{center}
\[
  \begin{array}{rcl}
\textbf{P} &::=& (\textbf{proc}~p~(\textbf{var}\ \texttt{l} : L)~s)* \\
\textbf{s} &::=& s;~s \alt \texttt{l} := e \alt \texttt{l}(r)\ := e \\
&\alt& \textbf{skip} \alt  \textbf{assume}~e \\
&\alt& \textbf{if}~e~\textbf{then}~s~\textbf{else}~s \alt \textbf{while}~e~\textbf{do}~s \\
&\alt& \textbf{call}~\texttt{l}\ := p~e~\vec{r_\delta}~\vec{r_\omega} \alt \textbf{return}~e \\
&\alt& \textbf{post}~r \leftarrow p~e~\vec{r}~\vec{r_\delta}~\vec{r_\omega}~d \\
&\alt& \textbf{await}~r \alt \textbf{ewait}~r \\
  \end{array}
\]
  \end{center}
  \caption{The surface syntax for task parallel programs.}
  \label{fig:syntax}
\end{figure}

The surface syntax for the language is given in \figref{fig:syntax}. A program \textbf{P} is a sequence of procedures. The procedure name $p$ is taken from a finite set of names \texttt{Proc}. Each procedure has a single $L$-type parameter \texttt{l} taken from a finite set of parameter names \texttt{Vars}. The body of the procedure is inductively defined by $s$. The semantics is abstracted over concrete values and operations, so the possible types of \texttt{l} are not specified nor is the particular expression language, $e$, but assume it includes variables references and Boolean values (\textbf{true} and \textbf{false}). The details of either $L$ or $e$ are never relevant for computation graph construction and are thus omitted. The set of all expressions is given by \texttt{Exprs}. Values are given by the finite set \texttt{Vals} and include at least Boolean values. \texttt{Exprs} contain \texttt{Vals} and the \emph{choice operator} $\star$. 

The statements ($s$) of the language denote the behavior of the procedure. Most statements, like the \textbf{if}-statement, \textbf{;}-statement, and \textbf{while}-statement have their typical meaning. Other statements require further explanations.

Statements are divided into the concurrent statements (\textbf{post}-statement, \textbf{await}-statement, and \textbf{ewait}-statement) and sequential statements (everything else).  Let \texttt{Regs} be a finite set of region identifiers. Associated with each region $r$ is a single variable referenced in the surface syntax by $\texttt{l}(r)$. A task is posted into a region $r$ by indicating the procedure $p$ for the task with an expression for the local variable value $e$, three lists of regions from $\texttt{Regs}^\ast$ (i.e., the Kleene closure on \texttt{Regs}), and a return value handler $d$. For the region lists, $\vec{r}$ are regions whose ownership is transferred from the parent to the new child task (i.e., the child now owns the tasks in those regions), $\vec{r_\delta}$ are regions in which the new task can read the region variables, and $\vec{r_\omega}$ are regions in which the task can write region variables. Let \texttt{Stmts} be the set of all statements and let $\texttt{Rets} \subseteq (\texttt{Vals} \rightarrow \texttt{Stmts})$ be the set of return value handlers. The handler $d$ associates the return value of the procedure with a user defined statement. 

The \textbf{await} and \textbf{ewait} statements synchronize a task with the sub-ordinate tasks in the indicated region. Intuitively, when a task calls \textbf{await} on region $r$, it is blocked until all the tasks it knows about in $r$ finish execution. Similarly, when a task issues an \textbf{ewait} with region $r$, it is blocked until one task it knows about in $r$ completes. A task is termed \emph{completed} when its statement is a \textbf{return}-statement.  

\begin{figure}
  \begin{center}
    \begin{lstlisting}[mathescape=true]
  proc main (var n : int)
  	n := 1;
	post $r_1 \leftarrow p_1~n~\varepsilon~(r_1)~(r_1)~\lambda v.n := n + v$;
	post $r_1 \leftarrow p_2~n~\varepsilon~(r_1)~(r_1)~\lambda v.n := n + v$;
	await $r_1$
  proc $p_1$ (var n : int)
  	$\texttt{l}(r_1) := \texttt{l}(r_1) + n$;
	return (n + 1)
  proc $p_2$ (var n : int)
  	$\texttt{l}(r_1) := \texttt{l}(r_1) + n$;
	return (n + 2)
\end{lstlisting}
  \end{center}
  \caption{A simple example of a task parallel program.}
  \label{fig:hj-async-finish}
\end{figure}

The \textbf{assume}-statement blocks a task until its expression $e$ evaluates to \textbf{true}. By way of definition, \textbf{call}, \textbf{return}, \textbf{post}, \textbf{ewait}, and \textbf{await} are \emph{inter-procedural} statements. All other statements are \emph{intra-procedural}.

\figref{fig:hj-async-finish} shows a simple example program. The main task posts two new tasks $t_1$ and $t_2$ executing procedures $p_1$ and $p_2$ in region $r_1$. $\varepsilon$ denotes an empty region sequence. The tasks $t_1$ and $t_2$ have access to the variable $r_1$. The $main$ task awaits the completion of $t_1$ and $t_2$. The return value handler of procedure $main$ takes the value returned by the tasks $t_1$ and $t_2$ and updates the value of $n$. The computation graph for this program is that in \figref{fig:cg}.

\section{Tree-based Semantics}

The semantics is defined over trees of procedure frames to represent the parallelism in the language rather than stacks which are inherently sequential. That means that the frame of each posted task becomes a child to the parent's frame. The parent-child relationship is transferred appropriately with task passing or when a parent completes without synchronizing with its children. The evolution of the program proceeds by a task either taking an intra-procedural step, posting a new child frame, or removing a frame for a synchronized completed task.

A task $t = \tuple{\ell, s, d, \vec{r_\delta}, \vec{r_\omega}, n}$ is a valuation of the procedure local variable \texttt{l}, along with a statement $s$, a return value handler $d$, a list of regions that it may use for read variables, a list of regions it may use for write variables, and an associated node in the computation graph for this task. When a procedure $p$ is posted as a task, the statement $s$ is the statement defined for the procedure $p$---recall that statements are inductively defined. 

A \emph{tree configuration}, $c = \tuple{t,m}$, is an inductively defined tree with task-labeled vertexes, $t$, and region labeled edges given by the \emph{region valuation} function, $m : \texttt{Regs} \rightarrow \mathbb{M}[\texttt{Configs}]$, where \texttt{Configs} is the set of tree configurations and $\mathbb{M}[\texttt{Configs}]$ are configuration multi-sets. For a given vertex $c = \tuple{t,m}$, $m(r)$ returns the collection of sub-trees connected to the $t$-labeled root by $r$-labeled edges.

The semantics relies on manipulating region valuations for task passing between parents and children. For two region valuations $m_1$ and $m_2$, the notation $m_1 \cup m_2$ is the multi-set union of each valuation. Further, the notation $m\ |_{\vec{r}}$ is the projection of $m$ to the sequence $\vec{r}$ defined as $m\ |_{\vec{r}}(r^\prime) = m(r^\prime)$  when $r^\prime$ is found somewhere in $\vec{r}$, and $m\ |_{\vec{r}}(r^\prime) = \emptyset$ otherwise. 

Let $\llbracket \cdot \rrbracket_e$ be an evaluation function for expressions without any program or region variables such that $\llbracket \star \rrbracket_e = \texttt{Vals}$, and let $\ell(r)$ denote the value of the region variable in $r$.  For convenience in the semantics definition, an evaluation function is defined over a task $t$ that enforces the read rights assigned to the task:
\begin{eqnarray*}
  e(t) &=& e(\tuple{\ell, s, d, \vec{r_\delta}, \vec{r_\omega}, n}) \\
  &=& e(\ell, \vec{r_\delta}) \\
  &=& e(\ell, r_0, r_1, \ldots) \\
  &=& \llbracket e[\ell / \texttt{l},\ell(r_0) / \texttt{l}(r_0), \ell(r_1) / \texttt{l}(r_1), \ldots]  \rrbracket_e
  \end{eqnarray*}
  
If $e[\ell / \texttt{l},\ell(r_0) / \texttt{l}(r_0), \ell(r_1) / \texttt{l}(r_1), \ldots]$ has any free variables, then by definition,\\
$\llbracket e[\ell / \texttt{l},\ell(r_0) / \texttt{l}(r_0), \ell(r_1) / \texttt{l}(r_1), \ldots]  \rrbracket_e$ has no meaning and is undefined (i.e., $e(t) = \emptyset$). As a final convenience for dealing with expressions in the semantics when constructing computation graphs, let the set of regions whose variables appear in $e$ be denoted by $\eta(e)$. 

Contexts are used to further simplify the notation needed to define the semantics.  A \emph{configuration context}, $C$, is a tree with a single $\diamond$-labeled leaf, task-labeled vertexes, and region-labeled edges. The notation $C[c]$ denotes the configuration obtained by substituting a configuration $c$ for the unique $\diamond$-labeled leaf of $C$. The configuration isolates individual task transitions (e.g., $C[\tuple{t,m}] \rightarrow C[\tuple{t^\prime,m}]$ denotes an intra-procedural transition on a task). Similarly, a \emph{statement context} is given as $S = \diamond ; s_1; \dots ;s_i$ and $S[s]$ indicates that $\diamond$ is replaced by $s$ where $s$ is the next statement to be executed. A \emph{task statement context}, $T = \tuple{\ell,  S, d, \vec{r_\delta}, \vec{r_\omega}, n}$ is a task with a statement context in place of a statement, and likewise $T[s]$ indicates that $s$ is the next statement to be executed in the task. Like configuration contexts, task statement contexts isolate the statement to be executed (e.g., $C[\tuple{T[s_1],m}] \rightarrow C[\tuple{T[s_2],m}]$ denotes an intra-procedural transition that modifies the statement in some way). For convenience, $e(t)$ is naturally extended to use contexts as indicated by $e(T)$. 

As indicated previously, a task $t$ is completed when its next to be executed statement $s$ is \textbf{return} $e$. The set of possible return-value handler statements for $t$ is $\mathrm{rvh}(t) = \{d(\ell) \mid \ell \in e(T)\}$ given the task's context. By defnition, $\mathrm{rvh}(t) = \emptyset$ when $t$ is not completed or $e(T)$ is undefined. 

The initial condition for a program $\iota = \tuple{p, \ell}$ is an initial procedure $p \in \texttt{Procs}$ and an initial value $\ell \in \texttt{Vals}$. The initial configuration is created from $\iota$ as $c = \tuple{\tuple{\ell, s_p, d, \vec{r_\delta}, \vec{r_\omega}, n}, m}$, where $s_p$ is the statement for the procedure $p$, $d$ is the identity function (i.e., $\lambda v.v$), $\vec{r_\delta}$ list regions whose variables are read by $p$, $\vec{r_\omega}$ lists regions whose variables are written by $p$, $n$ is a fresh node for the computation graph (i.e., $n = \mathrm{fresh}()$), and $\forall r \in \texttt{Regs}, m(r) = \emptyset$.

\begin{figure}
  \begin{center}
    \mprset{flushleft}
    \begin{mathpar}
      \inferrule[Assign Local]
                {
                  \ell^\prime \in e(\ell,\vec{r_\delta}) \\
                  \delta = \delta \cup (n \mapsto \eta(e))
                }
                {
                  \tuple{\ell, S[\texttt{l} := e], d,
                    \vec{r_\delta}, \vec{r_\omega}, n} \rightarrow
                  \tuple{\ell^\prime, S[\textbf{skip}], d,
                    \vec{r_\delta}, \vec{r_\omega}, n}
                }
      \and
      \inferrule[Assign Region]
                {
                  \ell \in e(T) \\
                  r\ \mathrm{is\ found\ in}\ \vec{r_\omega}(T) \\
                  \ell(r) = \ell \\\\
                  \delta = \delta \cup (n \mapsto \eta(e)) \\
                  \omega = \omega \cup (n \mapsto \{r\})
                }
                {
                  T[\texttt{l}(r)~:=~e] \rightarrow T[\textbf{skip}]
                }
      \and
      \\
      \inferrule[Skip]
                {
                }
                {
                  T[\textbf{skip};~s] \rightarrow T[s]
                }
      \and
      \inferrule[Assume]
                {
                  \mathbf{true} \in e(T) \\
                  \delta = \delta \cup (n \mapsto \eta(e))
                }
                {
                  T[\textbf{assume}~e] \rightarrow T[\textbf{skip}]
                }
      \and
      \inferrule[If-then]
                {
                  \textbf{true} \in e(T) \\
                  \delta = \delta \cup (n \mapsto \eta(e))
                }
                {
                  T[\textbf{if}~e~\textbf{then}~s_1~\textbf{else}~s_2]
                  \rightarrow T[s_1]
                }
      \and
      \inferrule[If-else]
                {
                  \textbf{false} \in e(T) \\
                  \delta = \delta \cup (n \mapsto \eta(e))
                }
                {
                  T[\textbf{if}~e~\textbf{then}~s_1~\textbf{else}~s_2]
                  \rightarrow T[s_2]
                }
      \and
      \inferrule[Do-loop]
                {
                  \textbf{true} \in e(T) \\
                  \delta = \delta \cup (n \mapsto \eta(e))
                }
                {
                  T[\textbf{while}~e~\textbf{do}~s] \rightarrow
                  T[s;~\textbf{while}~e~\textbf{do}~s]
                }
     \and 
     \inferrule[Do-break]
                {
                  \textbf{false} \in e(T) \\
                  \delta = \delta \cup (n \mapsto \eta(e))
                }
                {
                  T[\textbf{while}~e~\textbf{do}~s] \rightarrow
                  T[\textbf{skip}]
                }
    \end{mathpar}
  \end{center}
  \caption{The transition rules for the intra-procedural statements.}
  \label{fig:intra}
\end{figure}

The semantics is now given as a set of transition rules relating tree configurations. The rules assume the presence of a global computation graph, $G = \tuple{N, E, \delta, \omega}$, that is updated as part of the transition. The initial graph contains a single node $N = \{n\}$ from the initial configuration, no edges ($E = \emptyset$), and not read/write information ($\delta(n) = \emptyset$ and $\omega(n) = \emptyset$).

\figref{fig:intra} lists the intra-procedural transition rules. The rules omit the configuration context since intra-procedural statements do not need the region valuation from the context. The rules define the intra-procedural statements in the usual way. Of note is the update of the computation graph to record any read region variables from expressions or any write region variables from an assignment. The notation, $\delta = \delta \cup (n \mapsto \eta(e))$, is understood to update $\delta$ such that $n$ additionally maps to $\eta(e)$. 
%In reading the transition rules, 
The notation $\vec{r_\omega}(T)$ in the assign-region rule is used to indicated the read-region vector in the task or task context, $T = \tuple{\ell,  S, d, \vec{r_\delta}, \vec{r_\omega}, n}$. Similar notation is used in other rules to access the tuple.

\begin{figure*}
  \begin{center}
    \mprset{flushleft}
    \begin{mathpar}
      \inferrule[Call]
                {
                }
                {
                  C[T[\textbf{call}~\texttt{l}\ := p~e~\vec{r_\delta}~\vec{r_\omega}], m] \rightarrow \\\\
                  C[T[\textbf{post}~r_\mathit{call}\leftarrow p~e~\varepsilon~\vec{r_\delta}~\vec{r_\omega}~\lambda v.\texttt{l} := v;~ \textbf{ewait}~r_\mathit{call}], m]
               }
      \and
      \inferrule[Post]
                {
                  n_0^\prime = \mathrm{fresh}() \\
                  n_1 = \mathrm{fresh}() \\\\
                  N = N \cup \{n_0^\prime, n_1\} \\
                  E = E \cup \{\tuple{n_0, n_0^\prime}, \tuple{n_0, n_1}\}\\\\
                  \ell \in e(\ell^\prime,\vec{r_\delta}^\prime) \\
                  \delta = \delta \cup (n_0 \mapsto \eta(e))\\\\
                  m^\prime = (m \setminus m |_{\vec{r}}) \cup
                  (r \mapsto \tuple{
                    \tuple{\ell, s_p, d, \vec{r_\delta}, \vec{r_\omega},n_1},m|_{\vec{r}}})
                }
                {
                  C[\tuple{\ell^\prime,
                      S[\textbf{post}~r \leftarrow                  p~e~\vec{r}~\vec{r_\delta}~\vec{r_\omega}~d],\vec{r_\delta}^\prime,\vec{r_\omega}^\prime,d^\prime, n_0}, m] \rightarrow \\\\
                  C[\tuple{\ell^\prime,
S[\textbf{skip}],\vec{r_\delta}^\prime,\vec{r_\omega}^\prime,d^\prime, n_0^\prime}, m^\prime]
                }
      \and
      \inferrule[Ewait]
                {
                  n^\prime = \mathrm{fresh}() \\\\
                  N= N \cup \{n^\prime\} \\
                  E = E \cup \{\tuple{n, n^\prime}, \tuple{n(t_2),n^\prime}\} \\\\
                  m_1 = (r \mapsto \tuple{t_2,m_2}) \cup m_1^\prime \\
                  s \in \mathrm{rvh}(t_2) 
                }
                {
                  C[\tuple{\ell,
S[\textbf{ewait}~r],\vec{r_\delta},\vec{r_\omega},d, n}, m_1] \rightarrow \\\\
                  C[\tuple{\ell,
S[s],\vec{r_\delta},\vec{r_\omega},d, n^\prime}, m_1' \cup m_2]
          }
      \and
      \inferrule[Await-next]
                {
                  n^\prime = \mathrm{fresh}() \\\\
                  N= N \cup \{n^\prime\} \\
                  E = E \cup \{\tuple{n, n^\prime}, \tuple{n(t_2),n^\prime}\} \\\\
                  m_1 = (r \mapsto \tuple{t_2,m_2}) \cup m_1^\prime \\
                  s \in \mathrm{rvh}(t_2) 
                }
                {
                  C[\tuple{\ell,
S[\textbf{await}~r],\vec{r_\delta},\vec{r_\omega},d, n}, m_1] \rightarrow \\\\
                  C[\tuple{\ell,
S[s;~\textbf{await}~r],\vec{r_\delta},\vec{r_\omega},d, n^\prime}, m_1' \cup m_2]
                }
      \and
      \inferrule[Await-done]
                {
                  m(r) = \emptyset
                }
                {
                  C[T_1[\textbf{await}~r] , m] \rightarrow
                  C[T_1[\textbf{skip}], m]
                }
\end{mathpar}
  \end{center}
  \caption{The transition rules for the inter-procedural statements.}
  \label{fig:inter}
    \label{fig:semantics}
\end{figure*}

\figref{fig:inter} shows semantics for the inter-procedural statements. The \textbf{call} statement is interpreted as a \textbf{post} followed by \textbf{ewait} on some region $r_{call}$. This region $r_{call}$ is exclusive to the task calling the procedure and cannot be used to post new tasks into this region. A call statement does not allow ownership of any tasks to be passed to the newly created task. The region variables that are available to this task for reading and writing are denoted by $\vec{r_\delta}$ and $\vec{r_\omega}$ respectively.

\begin{figure}
  \begin{center}
     \subfigure[]{\includegraphics[scale=0.3]{../figs/Fig1-a.pdf}}   
     \subfigure[]{\includegraphics[scale=0.3]{../figs/Fig1-b.pdf}}
  \caption{Steps involved in computation graph creation.}
   \label{fig:cgcreation}
   \end{center}
\end{figure}

The \textsc{Post} rule is fired when the task forks to create a new child task that potentially runs in parallel with the parent task. When a task $t_1$ executes a \textbf{post} statement, two fresh nodes $n_0^\prime$ and $n_1$ are added to the graph. Node $n_0^\prime$ represents the statements following \textbf{post} and $n_1$ represents the statements executed by $t_2$. The current node $n_0$ of $t_1$ is connected to $n_0^\prime$ and $n_1$ as shown in \figref{fig:cgcreation}(a). The read set $\delta$ of node $n_0$ is updated to additionally map to the regions in $\eta(e)$ (i.e., the regions referenced in the expression $e$). The current node of $t_1$ changes to $n_0^\prime$ after the transition. The region mapping $m$ of task $t_1$ is updated by removing the configurations of regions whose ownership is passed to the newly created task $t_2$ and adding a new configuration that consists of the task $t_2$ along with the regions it now owns.

The \textsc{Ewait} rule blocks the execution of the currently executing task until a task in the indicated region completes. The choice of completed task, $t_2$, in the region is non-deterministic. A node $n^\prime$ is added to the graph to act as a join node. It captures the subsequent statements executed by task $t_1$ after the \textbf{ewait} statement finishes. The current node $n$ of task $t_1$ and the current node of task $t_2$, denoted by $n(t_2)$ are connected to $n^\prime$ as shown in \figref{fig:cgcreation}(b). The configuration ($r \mapsto \tuple{t_2,m_2}$) is removed from the region valuation $m$ of task $t_1$. After the transition, the current node of task $t_1$ is changed to $n^\prime$.  The task $t_1$ is resumed with a return value handler for the completed task ($\mathrm{rvh}(t_2)$) before continuing with its next statement.

The \textsc{Await-next} rule blocks the execution of the currently executing task $t_1$ until all the tasks whose handles are stored in region $r$ that the task $t_1$ owns are executed to completion. The rule is implemented recursively by removing one task from the region at a time and then inserting another \textbf{await}-statement on the same region. Similar to \textsc{Ewait}, a join node $n^\prime$ is added to the graph, the current nodes of $t_1$ and $t_2$ are connected to $n^\prime$ as shown in \figref{fig:cgcreation}(b) and the current node of task $t_1$ is changed to $n^\prime$. When task $t_2$ returns a value to $t_1$, $t_1$ executes the statement from the return value handler $\mathrm{rvh}(t_2)$. The \textsc{Await-done} rule terminates recursion when the region is empty. 

The computation graph for the example in \figref{fig:hj-async-finish} is presented in \figref{fig:cg}. When the program starts executing, a node $n_0$ is added to the graph to represent the procedure $main$. When the task $t_1$ is posted by the procedure $main$ to execute procedure $p_1$, two new nodes $n_0^\prime$ and $n_1$ are added to the graph to represent the statements executed by the procedure $main$ and procedure $p_1$ respectively. Similarly, when task $t_2$ is posted by the procedure main to execute procedure $p_2$, two new nodes $n_0^{\prime\prime}$ and $n_2$ are added to the graph. When the main task calls \textbf{await} on $r_1$, its execution is suspended until $t_1$ and $t_2$ finish execution. When the \textbf{await} is executed, node $r_1$ and $r_1^\prime$ are added to the graph. The read and write to region variable $r_1$ by the tasks $t_1$ and $t_2$ is updated in nodes $n_1$ and $n_2$ using the functions $\delta$ and $\omega$ respectively.

The order of synchronization of tasks $t_1$ and $t_2$ affects the value of the variable $n$ in the $main$ task. The return value handlers of the tasks get executed in different orders under different schedules. This makes the output of the program non-deterministic. In a schedule where task $t_1$ joins $main$ task before $t_2$, the value of $n$ at the end of program execution is 3 and in a schedule where task $t_2$ joins $main$ task before $t_1$, the value of $n$ is 2.

\begin{comment}
\begin{theorem}
Algorithm \ref{algo:drd} is complete for a task parallel program with a given input.
\end{theorem}

\begin{proof}
A task parallel program can have different computation graphs based on the schedule followed by the tasks during the program execution. The semantics described in \figref{fig:semantics} show how a computation graph is built for a task parallel program from an observed program execution. The nodes in the graph depict the correct partial order between the tasks in the program. If Algorithm \ref{algo:drd} does not report a race for a computation graph obtained from some execution of the program, data races may still be present under some other program schedule. Therefore, \algoref{algo:drd} is not sound for a task parallel program with a given input. However, if \algoref{algo:drd} reports a data race, then from \thmref{thm:graph} this is a real data race. Therefore, Algorithm \ref{algo:drd} is complete for a task parallel program with a given input.
\end{proof}
\end{comment}

\begin{theorem} \label{thm:cg}
%The computation graph represents the correct partial order relation between the tasks in the program.
The computation graph represents the correct ordering of events in a program and stores the accesses to shared variables in the program. The sequential events are ordered while the concurrent events are unordered.
\end{theorem}

\begin{proof}
Proof by definition: There are two types of operations performed in a task parallel program: inter-procedural and intra-procedural. The inter-procedural statements create different nodes in the graph and are responsible for maintaining the correct ordering of events in the program. The nodes in the computation graph contain read/write sets to store the accesses to shared variables by the tasks. The intra-procedural operations do not affect the structure of the computation graph; however, they update the read/write sets of the nodes in the computation graph when the tasks access shared variables. These operations are discussed separately below.
 
The semantics for inter-procedural statements are given in \figref{fig:semantics}.
The inter-procedural statements are \textbf{post}, \textbf{ewait}, \textbf{await} and \textbf{call}. When a \textbf{post} statement is executed, the \textsc{Post} rule is fired. It creates two new nodes in the computation graph. One node represents the statements executed by the newly posted task and the other node represents the statements executed by the calling task immediately following the \textbf{post} statement. These nodes are set as the active nodes for these tasks. Any access to the shared memory is stored in the read/write sets for the active node. These two nodes are unordered since the statements are executed concurrently by these tasks.

 The \textbf{ewait} is used to synchronize a child task with its parent task. The \textsc{Ewait} rule creates a join node in the computation graph. Both the child and the parent task's active nodes are connected to the join node. The added edges order this node after the active nodes in the child task and the parent task. The \textbf{await} statement joins all the children tasks posted in a region to the parent task. The \textbf{await} statement fires the \textsc{Await-next} rule that joins one child task at a time to the parent task. Similar to the \textsc{Ewait} rule, \textsc{Await-next} also creates a join node for every child task. The join node is set as the active node for the calling task. Any shared memory accesses by the calling task are registered in the read/write sets of the active node.
 
Finally, the \textbf{call} is semantic sugar for a \textbf{post} followed by an \textbf{ewait}. As such, even though the calling task gets a new active node to reflect its concurrent relationship to the newly created task, the read/write sets in that node are never updated since the calling task executes \textbf{ewait} immediately after the \textbf{post}, which does not read/write any region variables, and once the \textbf{ewait} completes, the task gets a new active node ordered after the join from the task created by the call.

The intra-procedural statements are \textbf{assign}, \textbf{skip}, \textbf{assume}, \textbf{if-then-else} and \textbf{do-while}. The semantics for intra-procedural statements is given in \figref{fig:intra}. The semantic rules for intra-procedural events show that they do not change the structure of the computation graph since none of the rules create any new nodes or edges in the computation graph.

The \textsc{Skip} rules does not interact with any shared variables in the program. The \textbf{if-then-else} and \textbf{do-while} statements fire \textsc{If-then}, \textsc{If-else}, \textsc{Do-loop} and \textsc{Do-break} rules. These rules only read shared variables. Therefore, only the read sets for the active nodes set by the inter-procedural statements are updated by the statements. The \textsc{Assign Local} rule only updates the read set of the active node since this rule does not update any shared variables. Whereas the \textsc{Assign Region} rule updates both read/write sets of the active node, since shared program variables are updated by this rule. As such, by definition, the computation graph exactly reflects the orders defined by the semantics and only updates read/write sets that are defined by the semantics.
\end{proof}

\begin{corollary}
%\algoref{algo:drd} is complete for a task parallel program with a given input.
Applying \algoref{algo:drd} to computation graphs created using the semantics of task parallel programs is complete for data race detection in the given program input -- data race free programs will never be rejected; but, programs with data race may be accepted because the data race did not manifest in the computation graph from the executed schedule.
\end{corollary}

\begin{proof}
Proof by example: A task parallel program can have different computation graphs based on the schedule followed by the tasks during the program execution. If \algoref{algo:drd} does not report a race for a computation graph obtained from some execution of the program, data races may still be present under some other program schedule. 

\begin{figure}
  \begin{center}
    \begin{lstlisting}[mathescape=true]
  proc main(var n : int)
  	n := 1;
  	post $r_1 \leftarrow p_1~n~\varepsilon~\{r_1\}~\{r_1\}~\lambda v. n := v$;
  	post $r_1 \leftarrow p_2~n~\varepsilon~\{r_1\}~\{r_1\}~\lambda v. n := v$;
  	ewait $r_1$ 
  	if $(n == 1)$ then
      post $r_1 \leftarrow p_3~n~\varepsilon~\{r_1\}~\{r_1\}~\lambda v. n$;
      $r_1 := 1$
   await $r_1$
 proc $p_1$(var n : int)
 	return 0
 proc $p_2$(var n : int)
	return 1
 proc $p_3$(var n : int)
	$r_1 := 2$
\end{lstlisting}
  \end{center}
  \caption{Parallel program with different computation graphs under different schedules.}
  \label{fig:diffCGs}
\end{figure}

\begin{figure}
  \centering
  \subfigure[$t_1$ joins first.]{\includegraphics[scale=0.35]{../figs/Figa.pdf}}
  \subfigure[$t_2$ joins first.]{\includegraphics[scale=0.35]{../figs/Figb.pdf}}
  \caption{Computation graphs of example in \figref{fig:hj-isolated}.}
   \label{fig:diffCGsfig}
\end{figure}

Consider the example in \figref{fig:diffCGs}. The task parallel program in the example has a data race under one program schedule and it is data race free under a different schedule. The computation graphs for the different schedules are shown in \figref{fig:diffCGsfig}. If the program follows the first schedule, (i.e., task $t_1$ joins before $t_2$) task $t_3$ is not spawned and there is no data race in the program. If the program, however, follows the second schedule(i.e., task $t_2$ joins first), then a new task $t_3$ is created by task $t_0$ and there is a data race on region variable $r_1$. 
%This example shows that \algoref{algo:drd} is not sound for task parallel programs with a given input. 

\thmref{thm:graph} shows that \algoref{algo:drd} is sound and complete for a given computation graph. When \algoref{algo:drd} is applied to task parallel programs with a given input, it may accept programs as data race free that in reality contain data races. This is evident from example in \figref{fig:diffCGs}. Therefore, determining data race freedom from a single schedule of a task parallel program using \algoref{algo:drd} is complete for the input program because different schedules create different computation graphs. 

%From \thmref{thm:cg}, it can be seen that a computation graph represents the correct partial order relation between the tasks of the program. \thmref{thm:graph} shows that a data race reported in the computation graph for the given program schedule is guaranteed to be a real data race. Therefore, \algoref{algo:drd} is complete for a task parallel program with a given input.
\end{proof}
\section{A Deterministic Fragment of the Model}
\label{sec:otf-drd}
Task parallel languages sometimes define fragments that impose restrictions to achieve determinism and the ability to serialize (e.g.,\cite{cave2011habanero}). Data-race detection using \algoref{algo:drd} on the computation graph from an single execution of a program in this fragment is both sound and complete. The fragment is defined by the following language restrictions:
\begin{compactitem}
\item Passing ownership of tasks from a parent to a child task is not allowed. 
\item Tasks whose return value handlers side effect can be posted in single-task regions only (i.e., regions that contain only a single task). A side-effect of a return value handler can be a change in the state of either the local variable or a region variable. 
\item All the tasks are joined to the main task at the end of the program execution. This is ensured by having the initial program configuration as \tuple{$T[\textbf{post} $ r_0 \leftarrow p_0~e~\varepsilon~\vec{r}~\vec{r}~\lambda$v.v; $\textbf{await}~r_0; \textbf{await}~r_1; \ldots], m_0} on some procedure $p_0$, $\vec{r}$ is the region sequence containing all regions and $\forall r \in \mathtt{Regs}, m_0(r) = \emptyset$
\end{compactitem}
\figref{fig:hj-async-fin} is an example from the equivalent fragment in the Habanero model to show the relationship to real-world programming languages. 

In Habanero, the \textbf{async} construct creates a new asynchronous task that runs in parallel with the parent task. The \textbf{finish} construct is used to collectively synchronize children tasks with their parent task. The \textbf{finish} $s$ statement causes the parent task to execute $s$ and then wait until all tasks created inside the finish-block have completed. The future construct lets tasks return values to other tasks with the operation \texttt{f.get()} that blocks until the task associated with $f$ completes. 

\begin{figure}
  \begin{center}
\begin{lstlisting}
public class Example1{
	static int x = 0;
	public static void main(String[] args) {
			finish {
				async { //Task1
					x = x + 1; }
				finish{
					async { //Task2
						x = x + 2;  } } }
			future f = async { //Task3
				return 5; }
			x = f.get(); } }
\end{lstlisting}
  \end{center}
  \vspace{-2em}
  \caption{An example of a Habanero Java Program.}
   \vspace{-2em}
  \label{fig:hj-async-fin}
\end{figure}

\begin{figure}
  \begin{center}
\begin{lstlisting}[mathescape=true]
  proc $main$ (var n : int)
  	$\texttt{l}(r_1) := 0;$
	post $r_1 \leftarrow p_1~0~\varepsilon~\vec{r}~\vec{r}~\lambda n.n$;
	post $r_2 \leftarrow p_2~0~\varepsilon~\vec{r}~\vec{r}~\lambda n.n$;
	await $r_2$;
	await $r_1$;
	post $r_3 \leftarrow p_2~0~\varepsilon~\vec{r}~\vec{r}~\lambda n. r_1 := n$;
	ewait $r_3$;	
  proc $p_1$ (var n : int)
  	$\texttt{l}(r_1) := \texttt{l}(r_1) + 1$
  proc $p_2$ (var n : int)
  	$\texttt{l}(r_1) := \texttt{l}(r_1) + 2$
  proc $p_3$ (var n : int)
  	return 5
\end{lstlisting}
  \end{center}
    \vspace{-2em}
  \caption{Converted version of the Habanero Java program from \figref{fig:hj-async-fin}.}
        \vspace{-1em}
  \label{fig:hj-async-fin-converted}
\end{figure}

\figref{fig:hj-async-fin-converted} is the equivalent program in the model in this presentation. The procedure $main$ posts task from the outer finish block to region $r_1$ and tasks from the inner finish block to region $r_2$. Since, the inner finish block completes execution first, \textbf{await} on region $r_2$ is called before $r_1$. The future posts a task to region $r_3$ followed by an \textbf{ewait} on $r_3$.

Let $\mathcal{G}( P )$ return the set of computation graphs from all possible schedules of the program $P$ from the deterministic fragment of the model, and let $\mathrm{DRF}( G )$ return true if \algoref{algo:drd} reports the graph to be data race free. 

\begin{lemma} 
\label{lem:drf}
 $(\exists G \in \mathcal{G}( P ),\ \mathrm{DRF}( G )) \rightarrow (\forall G \in \mathcal{G}( P ),\ \mathrm{DRF}( G ))$ 
\end{lemma}

The proof is omitted for space (as are the other non-trivial proofs), but the lemma states that if data-race is not detected in an observed execution of a program from the deterministic fragment, then all other possible executions are data-race free as well; in other words, programs from the fragment are deterministic.
Habanero makes this same claim but does not prove it \cite{cave2011habanero}. The corollary regarding data-race programs and the following theorem are trivial from \lemmaref{lem:drf}.

\begin{corollary}\label{cor:drf}
$(\exists G \in \mathcal{G}( P ),\ \neg\mathrm{DRF}( G )) \rightarrow (\forall G \in \neg\mathcal{G}( P ),\ \mathrm{DRF}( G ))$ 
\end{corollary}

\begin{theorem} \label{thm:strcutured-par-progs}
Using the tree semantics with Algorithm \ref{algo:drd} to detect data-race in the resulting computation graph is sound and complete for a task parallel program with a given input when that program is in the deterministic fragment of the language.
\end{theorem}

\begin{comment}
\section{Optimizations and enhancements}

 Task parallel programs that use mutual exclusion can have different computation graph structures based on the order in which the critical sections get executed. In this section, we discuss an algorithm to create all such computation graphs.
\end{comment}

\section{On-the-fly data race detection}
The data race detection technique presented in this work performs the analysis after the program has finished execution (post-mortem analysis). To improve the efficiency of analysis at run time, this paper presents a dynamic improvement for structured parallel programs. This technique is called on-the-fly data race detection.

The data race detection is run on a region as soon as \textbf{await} finishes execution on that region (i.e., \textsc{Await-done} fires). If no race is reported, all the nodes belonging to that region are merged into an equivalent master node that represents the region. The transformation preserves the partial order relative to other tasks. The variables accessed by the tasks in the region are added to the master node. 

\begin{figure}
  \begin{center}
    \begin{lstlisting}[mathescape=true]
  proc main(var n : int)
  	n := 1;
  	post $r_1 \leftarrow p_1~n~\varepsilon~\{r_1\}~\{r_1\}~\lambda v. n := n + v$;
	await $r_1$
 proc $p_1$(var n : int)
 	post $r_2 \leftarrow p_2~n~\varepsilon~\{r_1\}~\{r_1\}~\lambda v. n := n + v$;
  	post $r_2 \leftarrow p_3~n~\varepsilon~\{r_1\}~\{r_1\}~\lambda v. n := n + v$;  	  	
	await $r_2$
 proc $p_2$(var n : int)
	$\texttt{l}(r_2) := n$ 
 proc $p_3$(var n : int)
	$\texttt{l}(r_2) := n$
\end{lstlisting}
  \end{center}
    \vspace{-2em}
  \caption{Parallel program with nested regions.}
  \label{fig:nested-regions}
\end{figure}

\begin{figure}
  \centering
   \includegraphics[width=0.35\textwidth]{../figs/Fig11.pdf}
   \vspace{-1em}
  \caption{Computation graph of example in \figref{fig:nested-regions}.}
        \vspace{-1em}
   \label{fig:cg-nested-regions}
\end{figure}

\figref{fig:nested-regions} shows an example of a parallel program with nested regions. As soon as the \textbf{await} on region $r_2$ finishes execution, on-the-fly analysis is run on this region to check for data races in the nodes belonging to this region. If a race is not reported, a master node is added to the graph that represents region $r_2$ and the program is executed further. The computation graph for this example is shown in \figref{fig:cg-nested-regions}. The nodes highlighted in blue with dashed lines denote the sub-graph that is replaced by a master node if the region is data race free.

\section{Mutual Exclusion}

\begin{figure*}
  \begin{center}
    \mprset{flushleft}
    \begin{mathpar}
     \and
      \inferrule[Isolated]
                {
				  \mathrm{canIsolate(C)} = \mathit{true} \\
                  n^\prime = \mathrm{fresh}() \\
                  N = N \cup \{n^\prime\} \\
                  E = E \cup \{\tuple{n, n^\prime},  \tuple{\mathit{last},n^\prime}\}\\
                }
{C[\tuple{\ell^\prime,S[\textbf{isolated}~s],\vec{r_\delta}^\prime,\vec{r_\omega}^\prime,d^\prime, n, 0}, m] \rightarrow
                  C[\tuple{\ell^\prime,
				   S[s; \textbf{isolated-end}],\vec{r_\delta}^\prime,\vec{r_\omega}^\prime,d^\prime, n^\prime, 1}, m]
                }
\and
      \inferrule[Isolated-Nested]
                {\mathit{iso} > 0 \\
                  \mathit{iso}^\prime = \mathit{iso} + 1
                }
{C[\tuple{\ell^\prime,S[\textbf{isolated}~s],\vec{r_\delta}^\prime,\vec{r_\omega}^\prime,d^\prime, n, \mathit{iso}}, m] \rightarrow
                  C[\tuple{\ell^\prime,
				   S[s; \textbf{isolated-end}],\vec{r_\delta}^\prime,\vec{r_\omega}^\prime,d^\prime, n, \mathit{iso}^\prime}, m]
}
\and
      \inferrule[Isolated-End-Nested]
                {\mathit{iso} > 1 \\
                  \mathit{iso}^\prime = \mathit{iso} - 1
                }
{C[\tuple{\ell^\prime,S[\textbf{isolated-end}],\vec{r_\delta}^\prime,\vec{r_\omega}^\prime,d^\prime, n, \mathit{iso}}, m] \rightarrow
                  C[\tuple{\ell^\prime,
				   S[\textbf{skip}],\vec{r_\delta}^\prime,\vec{r_\omega}^\prime,d^\prime, n, \mathit{iso}^\prime}, m]
                }
\and
      \inferrule[Isolated-End]
                {
                  n^\prime = \mathrm{fresh()}\\
                  \mathit{last} = n \\
                  N = N \cup \{n^\prime\} \\
                  E = E \cup \{\tuple{n, n^\prime}\}
                }
{C[\tuple{\ell^\prime,S[\textbf{isolated-end}],\vec{r_\delta}^\prime,\vec{r_\omega}^\prime,d^\prime, n, 1}, m] \rightarrow
                  C[\tuple{\ell^\prime,
				   S[\textbf{skip}],\vec{r_\delta}^\prime,\vec{r_\omega}^\prime,d^\prime, n^\prime,0}, m]
                }

\end{mathpar}
  \end{center}
  \caption{The transition rules for isolated statements.}
  \label{fig:isol-semantics}
\end{figure*}

Programs that use mutual exclusion to protect accesses to shared variables can have different computation graphs based on access order. All of these computation graphs have to be enumerated and analyzed for conflicting accesses outside of mutual exclusion.

The task parallel language is extended to model mutual exclusion with a new statement: $\textbf{isolated}~s$.
The statement performs $s$ in mutual exclusion of any other isolated statements. The semantics with how the computation graph is impacted is in \figref{fig:isol-semantics}. The isolation is accomplished by creating a new global variable \textit{last} to track the last node in the computation graph belonging to an isolated statement, by adding to the task context a counter initialized to zero to count the number of nested isolated contexts, and with a new keyword for the rewrite rules: \textbf{isolated-end}. 

Let $\mathrm{canIsolate}(C)$ be a function over configurations to Boolean that returns true for a configuration tree if all the task counters are 0; otherwise it returns false. If no other isolated statements are running, then the \textsc{Isolated} rule increments the task counter to indicate isolation and inserts after the isolated statement $\mathit{s}$ the new \textbf{isolated-end} keyword. The computation graph gets a new node to track accesses in the isolated statement with an appropriate edge from the previous node. A sequencing edge from $\mathit{last}$ is also added so the previous isolated statement happens before this new isolated statement. As a note, $\mathit{last}$ is initialized to an empty node when execution starts. The \textsc{Isolated-Nested} rule simply increments the counter if the task is already in isolation.

The \textsc{Isolated-End-Nested} rule processes the new \textbf{isolated-end} keyword and decrements the counter. When the counter reaches the outer-most isolated context, the \textsc{Isolated-End} rule creates a new node in the computation graph to denote the end of isolation, and it updates $\mathit{last}$ to properly sequence any future isolation. As a note, the on-the-fly data race detection is modified to not reduce sub-graphs with isolation.

\begin{algorithm}
\caption{Scheduling algorithm for Isolated blocks} \label{algo:isolated}
\begin{algorithmic}[1]
  \Function{schedule}{$t$, $\mathtt{Regs}$, $\mathtt{Tasks}$}
  \State \texttt{loop}:\ ($\mathtt{Regs}$, $\mathtt{Tasks}$) $:=$ \texttt{run}($t$, $\mathtt{Regs}$, $\mathtt{Tasks}$)\label{loc:run}
  \State $s :=$ \texttt{status}($t$)
  \State $R :=$ \texttt{runnable}($\mathtt{Tasks}$)
  \If{ $s =$ ISOLATED}\label{loc:entry:isolated}
  \ForAll{$t_i \in R$}\label{loc:prsched}
  \State \texttt{schedule}($t_i$, $\mathtt{Regs}$, $\mathtt{Tasks}$)
  \EndFor
  \Else
  \State $t_i$ := \texttt{random}($R$)\label{loc:rand}
  \State \texttt{schedule}($t_i$, $\mathtt{Regs}$, $\mathtt{Tasks}$)
  \EndIf
  \EndFunction
\end{algorithmic}
\end{algorithm}

Algorithm \ref{algo:isolated} presents a scheduling algorithm to enumerate all computation graphs resulting from isolation \cite{mercer2015model}. The algorithm considers a simplified state of the program with $\mathtt{Regs}$ being the set of region variables that are shared among the tasks, $\mathtt{Tasks}$ being the set of tasks, $t$ being a task, and $R$ being the set of runnable tasks. The algorithm also implements sequential semantics where only one task runs at a time and that task runs until it waits, completes, or isolates at which time a scheduling choice is made. Sequential semantics are viable  by \lemmaref{lem:drf} and \corref{cor:drf} that establish independence in the computation graph and execution schedule in the absence of data races. 

Line 2 updates the region variables and pool of tasks by running task $t$ until it exits, waits, or reaches an \textbf{isolated}-construct. The function \texttt{status} on Line 3 returns the status of the task $t$. On Line 4, the function \texttt{runnable} is used to obtain a list of all the tasks that can be run from the pool of all tasks. If the status of the currently running task $t$ becomes ISOLATED (i.e., the task encounters an \textbf{isolated} construct), the task is preempted and all the tasks that are runnable, including the task that is trying to isolate, are scheduled by the runtime. When the task completes, a new task is randomly selected from the set of runnable tasks.

\begin{figure}
  \begin{center}
    \begin{lstlisting}[mathescape=true]
  proc main(var n : int)
  	n := 1;
  	post $r_1 \leftarrow p_1~n~\varepsilon~\{r_1\}~\{r_1\}~\lambda v. n := n + v$;
  	post $r_1 \leftarrow p_2~n~\varepsilon~\{r_1\}~\{r_1\}~\lambda v. n := n + v$;
  	await $r_1$
 proc $p_1$(var n : int)
 	isolated $\texttt{l}(r_1) := n + 1$
 proc $p_2$(var n : int)
	isolated if $(\texttt{l}(r_1) = n)$ then
	  	post $r_1 \leftarrow p_3~n~\varepsilon~\{r_1\}~\{r_1\}~\lambda v. n := n + v$;
	else
		$\texttt{l}(r_1) := n - 1$
 proc $p_3$(var n : int)
	$\texttt{l}(r_1) := n+2$
\end{lstlisting}
  \end{center}
  \vspace{-1em}
  \caption{Parallel Program with Mutual exclusion.}
  \vspace{-1em}
  \label{fig:hj-isolated}
\end{figure}

\begin{figure}
  \centering
  \subfigure[p2 runs before p1.]{\includegraphics[scale=0.2]{../figs/Fig5-a.pdf}}
  \subfigure[p1 runs before p2.]{\includegraphics[scale=0.2]{../figs/Fig5-b.pdf}}
  \caption{Computation graphs of example in \figref{fig:hj-isolated}}
  \vspace{-1em}
   \label{fig:cg-isolated}
\end{figure}

For the example in \figref{fig:hj-isolated}, two different computation graph structures can be formed based on the order of execution of isolated blocks. The computation graphs are shown in \figref{fig:cg-isolated}. If the scheduler runs the isolated section of task $t_1$ first, the computation graph in \figref{fig:cg-isolated}(a) is formed. Task $t_1$ changes the values of shared variable $r_1$ to 2. Hence, when task $t_2$ executes its isolated section, the if-condition fails and an additional task is not spawned by $t_2$. If the scheduler runs task $t_2$ first, the computation graph of \figref{fig:cg-isolated}(b) is formed. In this schedule, task $t_2$ executes its isolated section first. Since the value of variable $r_1$ is 1, the if-condition is met and a new task is created by $t_2$.

\begin{theorem}
\algoref{algo:isolated} finds all unique computation graphs for structured parallel programs with isolated sections making it sound and complete with \algoref{algo:drd}.
%\algoref{algo:drd} is sound and complete for structured parallel programs with mutual exclusion.
\end{theorem}

\begin{proof}
%Theorem \ref{thm:strcutured-par-progs} states that Algorithm \ref{algo:drd} is sound and complete for structured parallel programs that do not contain isolated sections. If mutual exclusion is present, Algorithm \ref{algo:drd} does not remain sound since different computation graph structures can be formed for such programs. Algorithm \ref{algo:isolated} helps to enumerate all such computation graph structures. Therefore, the data race detection using Algorithm \ref{algo:drd} becomes sound and complete when it is used along with Algorithm \ref{algo:isolated} for structured parallel programs that have mutual exclusion.
From \thmref{thm:strcutured-par-progs} and \algoref{algo:isolated}, \algoref{algo:drd} is sound and complete for structured parallel programs with mutual exclusion.
\end{proof}

\section{Results}

\begin{table*}[t]
\centering
\caption{Verification of HJ Micro-benchmarks using CGRaceDetector}
\label{tab:perf}
\begin{tabular}{|c|c|c|c|c|c|c|c|c|}
\hline
        &        & \multicolumn{3}{c|}{CGRaceDetector} & \multicolumn{3}{c|}{Precise Race Detector}
 \\ \hline
Test Case Name & SLOC & Tasks & States  & Time   & Error Info & States  & Time   & Error Info 
\\ \hline
Search Count & 50 & 4 & 195 & 0:00:01 & No Race & 145139 & 0:00:45 & No Race 
 \\ \hline
Existence of an occurrence & 45 & 4 & 174 & 0:00:01 & Detected Race & 50197 & 0:00:15 & Detected Race 
\\ \hline
Index of occurrence & 38 & 4 & 197 & 0:00:01 & Detected Race & 68806 & 0:00:29 & Detected Race 
\\ \hline
Existence of occurrence with no task & 45 & 2 & 117 & 0:00:00 & Detected Race & 296 & 0:00:00 & Detected Race
\\ 
creation after instance is found & &  &  &  & & & &
\\ \hline
Search Index With No task creation & 48  & 2 & 119 & 0:00:00 & Detected Race & 326 & 0:00:00 & Detected Race
\\
after Instance is Found &  &  &  & & & & &
\\ \hline
\end{tabular}
\end{table*}

We verified some of the HJ microbenchmarks that make use of only the basic parallel constructs such as async and finish using the CGRaceDetector listener. The CGRaceDetector is able to build computation graphs of the HJ programs by exploring very few states. We compared the output of CGRaceDetector to the output of PreciseRaceDetector and found that CGRaceDetctor was able to correctly identify races in all programs. These micro-benchmarks are variations of a linear search algorithm. The first test finds the count of occurrences of a search string in a given text string. The second test confirms the existence of search string in the given text string. The third test returns the index of occurrence of the search string. In case of multiple occurrences, the output becomes non-deterministic. The fourth test also confirms the existence of the search string in the given text. However, as soon as the search text is found, no more processes are spawned to search the text and the program is terminated. Similarly, in the fifth test, as soon as a process returns the index of occurrence of search text, the program terminates. The results are presented in Table I. The sizes of the programs are indicated by the SLOC column and Tasks columns represents the number of tasks created in every program. The results of CGRaceDetector and Precise Race Detector are compared. The Precise Race Detector systematically explores the entire state space of the program. The CGRaceDetector just uses one thread interleaving to detect data races. Hence, the time required by CGRaceDetector is considerably smaller than the time required by Precise Race Detector to execute.



\section{Related Work}

There is an existing extension for JPF for the X10 Language
\cite{conf:icst:GligoricMM12,x10}. Habanero is closely related to X10 in many of
its constructs. In the extension, JPF operates directly on the actual X10
runtime system. To accomplish the integration, JPF is modified, the X10 runtime
is modified, the X10 compiler is extended, and a new static analysis is
presented to help control state explosion. The extension represents a
significant effort that affects all aspects of the X10 framework to enable JPF
verification. 

There is a formal model for the Chapel language with an accompanying model
checker that employs symbolic execution \cite{chapel}. The formal model is an
intermediate representation (IR) suitable for concurrent constructs. The
approach compiles Chapel programs into the IR and the model checker then
verifies the IR for deadlock and data-race freedom. Creating a compiler and
model checker is a significant undertaking beyond the approach in this paper.
More critically, the verification tool models the runtime including the number
of available worker threads to service tasks; thus, the verification results are
dependent on the number of worker threads in the configuration rather than the
semantics of the Chapel language. 

\begin{comment} In this paper, correctness is a property of the HJ language
semantics with the given HJ program and not any aspect of the runtime
implementation. Verifying the HJ runtime system implements HJ semantics is a
verification problem separate from verifying that an HJ program is data-race
free.  \end{comment}

Another approach to verifying concurrent languages is to leverage the production
level language runtime system itself
 \cite{Vakkalanka:2008:DVM:1427782.1427794,Vo:2009:FVP:1594835.1504214,5644885,6113841}.
These approaches typically require instrumentation of the source program,
wrappers to intercept calls into the runtime, and a way to control runtime
behavior. Although they are typically able to generate states faster than JPF,
verification results are dependent on the employed runtime correctly
implementing the language semantics. 

Many tools for dynamic race detection have been developed \cite{Eraser,
Eraser-Upgrade, Cilk-Dynamic}. These tools track the set of locks held by each
task during execution and use these sets to determine if a shared resource is
insufficiently protected. These tools produce results that are independent of
thread interleavings. This is an improvement as compared to previous tools that are
dependent upon the thread interleavings of the current
execution \cite{Lamport-Comparison, Mesa, Lamport-Online}. 
Race detection algorithms have also been developed for task-parallel
languages \cite{Async-Finish-Race, SP-BAGS}. These approaches utilize the
structured parallelism of the language to quickly detect races. However, the
results of this approach are also limited to a single execution.

Many different approaches have been developed to statically detect race
conditions in programs \cite{ESC, Warlock, RacerX, Relay}. Each of these techniques
require varying levels of instrumentation by the user. RacerX infers
the resources each lock protects, code contexts which are multithreaded, and
race conditions that have a "dangerous" effect upon the running program.
RacerX relies upon the user to annotate the location of the method for
performing the lock/unlock operations. Any other annotations by the user
are acceptable, but not required.

Relay performs a static lockset analysis. The Relay algorithm computes relative
locksets that belong to each function in the program. This bottom-up approach
scales very nicely, however, this approach remains unsound.

General permission regions (GPR) is another static analysis strategy
that infers the locations in which to place program annotations \cite{Westbrook:2012:PPR:2367163.2367201}. Once the annotations have been
statically injected, a dynamic analysis is run to detect the presence of race
conditions. Unlike, RacerX, GPR doesn't require any user annotations, although
it will honor any annotations introduced by the user. GPR correctly detects race
conditions for most common parallel programming paradigms. 

\begin{comment} Such a dependence can be avoided, as in this work, by creating a
non-performance oriented runtime that is simple enough to manually verify for
correctness \cite{Morse:2012:MAM:2189257.2189279}. It is much easier to access
the state and direct behavior, as a model checker, in such systems.
\end{comment}

Recent work proves the problem of state-reachability to be decidable and
EXPSPACE hard for finite-valued programs in languages such as X10/Habanero
\cite{Bouajjani:2012:ARP:2103621.2103681}. The result is limited to a subset of
the powerful task constructs in such languages and justifies a model checking
effort. The computability and complexity of the more advanced constructs such as
phasers is yet to be determined.

\section{Conclusions \& Future Work}
This paper presents a model checking algorithm to prove when a
Habanero program does not contain any data races, deadlocks, assertion
violations, or exceptions for a given program input. The algorithm,
based on permission regions, only considers scheduling points in the
search tree at the boundaries of permission regions and {\tt
  isolated}-constructs. The paper includes a proof of soundness for
the algorithm, meaning that the algorithm may reject a correct program
due to the size of the permission regions.

The effectiveness of the algorithm is shown in several benchmark
programs that cover many of the Habanero concurrency constructs. The
analysis is done using a new Java library implementation of the Habanero
runtime that is intended for debugging and verification. The new algorithm, with permission regions,
is implemented as an extension to the \jpf\ model checker. The results
from the benchmark programs indicate a significant cost reduction when
using the new algorithm.

Future work includes automating the annotation of permission regions
based on the sharing detection in \jpf; automating the validation of any
counter-example; developing techniques to automatically refine
permission regions from counter-examples when needed; a partial order
reduction over permission regions; static-analysis to prevent
scheduling on regions that cannot race; applying symbolic techniques to reason over input; and studying
benchmarks that are representative of real world Habanero programs.


\clearpage
\bibliographystyle{IEEEtran}
\bibliography{../bib/paper}

\end{document}
