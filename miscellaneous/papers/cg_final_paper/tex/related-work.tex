\section{Related Work} \label{sec:rel-work}

Different types of data race detection techniques have been developed. The static race detectors analyze the source code to detect races. The dynamic ones use information from the actual program executions. Another technique for data race detection is model checking. In this method, a model of the system being analyzed is created and whether this model meets the specifications is exhaustively checked.

Static data race detectors require program instrumentation by the users. They can reason about all possible program runs. The major drawback of these systems is that they produce a large number of false-positives. \cite{engler2003racerx,ESC,abadi2006types,naik2006effective,voung2007relay,choi2001static, vechev2011automatic}. 

Dynamic race detectors use use different techniques to detect data races at runtime. The lock-set based tools track the set of locks held by each task during execution. These sets are then used to determine conflicts over shared memory references \cite{savage1997eraser, EraserUpgrade, elmas2006goldilocks, elmas2007goldilocks}. 

Dimitrov et al. developed a dynamic commutativity race detector \cite{dimitrov2014commutativity}. It uses vector clocks along with a commutativity specification to generate a structural representation of parallel programs that is used to locate races. Dynamic race detectors based on hashing asserts if different runs of a parallel program with same input produce different outputs \cite{nistor2010instantcheck}.

Lamport defined the happens-before relation in parallel programs \cite{lamport1978time}. The happens-before relation defines a partial order among all the operations in all the threads of a parallel program. The happen-before relation has been used in various data race detection techniques \cite{kahlon2009static, kahlon2010universal, flanagan2009fasttrack, mellor1991fly, schonberg1989fly, miller1988mechanism}. This approach has also been applied to task parallel languages such as Cilk and X10 \cite{Feng97efficientdetection, Async-Finish-Race}. Two approaches based on the happens-before relation, discussed in the introduction, have been developed for HJ programs \cite{raman2012scalable, drdForFutures}. 

Model checking systematically explores the entire state space of the programs to detect concurrency issues \cite{kulikov2010detecting, vakkalanka2008implementing, Godefroid}. The major drawback of model checking is the explosion in the state space as the program size increases. This technique has been extended to verify various task parallel languages such as HJ, X10 and Chapel\cite{anderson2014jpf, gligoric2012x10x, zirkel2013automated}. As opposed to model checking, predictive analysis observes only a single program execution and generalizes the verification results to all possible schedules. This approach has been applied to detecting communication deadlocks in MPI programs \cite{forejt2014precise}.

Various methods have been developed to tackle the state explosion problem of model checking. Rely-guarantee reasoning verifies threads individually using assertions about other threads \cite{xu1997rely, popeea2012compositional}. Thread modular analysis relies on a similar technique. It verifies threads individually using an abstraction of steps that may be performed by other threads \cite{flanagan2003thread, malkis2007precise, henzinger2003thread, gotsman2007thread}.

Hybrid race detection systems have been developed that combine various techniques to overcome some of the limitations of these methods. Permission regions use static program instrumentation combined with dynamic analysis to detect races \cite{westbrook2012practical, westbrook2012permission}. Gradual permission regions use a similar program instrumentation along with model checking \cite{mercer2015model}. 

This work makes use of the happens-before relation for dynamic analysis of programs and use model checking to ensure all schedules are considered in programs with mutual exclusion. A lot of different techniques create model of programs from program executions and use the models for verification. SATCheck observes the program execution to build a concrete behavior model of program execution and using a SAT solver, it tries to find other interesting behaviors \cite{demsky2015satcheck}. Coverage driven testing uses program execution to create a model of the thread interleavings and shared memory accesses to identify unexplored thread interleavings \cite{hong2012testing, yu2012maple}. Regression testing tools for concurrent programs use changes in the program model to identify shared memory accesses that might be affected by the code changes and identifying thread interleavings that must be explored to expose regression bugs \cite{terragni2015recontest, yu2014simrt}. Dynamic symbolic execution is combined with unfolding of petri-nets to create minimal test-suites for testing multi-threaded programs \cite{leon2015unfolding, kahkonen2015unfolding}.