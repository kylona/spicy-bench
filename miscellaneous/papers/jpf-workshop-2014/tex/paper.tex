% THIS IS SIGPROC-SP.TEX - VERSION 3.1
% WORKS WITH V3.2SP OF ACM_PROC_ARTICLE-SP.CLS
% APRIL 2009
%
% It is an example file showing how to use the 'acm_proc_article-sp.cls' V3.2SP
% LaTeX2e document class file for Conference Proceedings submissions.
% ----------------------------------------------------------------------------------------------------------------
% This .tex file (and associated .cls V3.2SP) *DOES NOT* produce:
%       1) The Permission Statement
%       2) The Conference (location) Info information
%       3) The Copyright Line with ACM data
%       4) Page numbering
% ---------------------------------------------------------------------------------------------------------------
% It is an example which *does* use the .bib file (from which the .bbl file
% is produced).
% REMEMBER HOWEVER: After having produced the .bbl file,
% and prior to final submission,
% you need to 'insert'  your .bbl file into your source .tex file so as to provide
% ONE 'self-contained' source file.
%
% Questions regarding SIGS should be sent to
% Adrienne Griscti ---> griscti@acm.org
%
% Questions/suggestions regarding the guidelines, .tex and .cls files, etc. to
% Gerald Murray ---> murray@hq.acm.org
%
% For tracking purposes - this is V3.1SP - APRIL 2009

\documentclass{sig-alternate}
\setlength{\paperheight}{11in}
\setlength{\paperwidth}{8.5in}
\usepackage[pass]{geometry}
\usepackage{verbatim}
\usepackage{url}
\usepackage{listings}
\usepackage{color}
\usepackage{multicol}
\usepackage{paralist}
\usepackage{comment}
\usepackage{hyperref}

\newcommand{\figref}[1]{Figure~\ref{#1}}
\newcommand{\tableref}[1]{Table~\ref{#1}}
\newcommand{\secref}[1]{Section~\ref{#1}}

\definecolor{dkgreen}{rgb}{0,0.6,0}
\definecolor{gray}{rgb}{0.5,0.5,0.5}
\definecolor{mauve}{rgb}{0.58,0,0.82}

\lstset{frame=tb,
  language=Java,
  aboveskip=3mm,
  belowskip=3mm,
  showstringspaces=false,
  columns=flexible,
  basicstyle={\small\ttfamily},
  numbers=none,
  numberstyle=\tiny\color{gray},
  keywordstyle=\color{blue},
  commentstyle=\color{dkgreen},
  stringstyle=\color{mauve},
  breaklines=true,
  breakatwhitespace=true
  tabsize=3
}

\begin{document}

\title{JPF Verification of Habanero Java Programs using Gradual Type Permission Regions\titlenote{Research funded by NSF grant CCF-1302524}}

%
% You need the command \numberofauthors to handle the 'placement
% and alignment' of the authors beneath the title.
%
% For aesthetic reasons, we recommend 'three authors at a time'
% i.e. three 'name/affiliation blocks' be placed beneath the title.
%
% NOTE: You are NOT restricted in how many 'rows' of
% "name/affiliations" may appear. We just ask that you restrict
% the number of 'columns' to three.
%
% Because of the available 'opening page real-estate'
% we ask you to refrain from putting more than six authors
% (two rows with three columns) beneath the article title.
% More than six makes the first-page appear very cluttered indeed.
%
% Use the \alignauthor commands to handle the names
% and affiliations for an 'aesthetic maximum' of six authors.
% Add names, affiliations, addresses for
% the seventh etc. author(s) as the argument for the
% \additionalauthors command.
% These 'additional authors' will be output/set for you
% without further effort on your part as the last section in
% the body of your article BEFORE References or any Appendices.

\numberofauthors{4} %  in this sample file, there are a *total*
% of EIGHT authors. SIX appear on the 'first-page' (for formatting
% reasons) and the remaining two appear in the \additionalauthors section.
%
\author{
\alignauthor
Peter Anderson\\
       \affaddr{Brigham Young University}\\
       \affaddr{Provo, Utah}\\
       \email{anderson.peter@byu.edu}
\alignauthor
Nick Vrvilo\\
    \affaddr{Rice University}\\
    \affaddr{Houston, Texas}\\
    \email{nv4@rice.edu}
\and
\alignauthor
Eric Mercer \\
       \affaddr{Brigham Young University}\\
       \affaddr{Provo, Utah}\\
       \email{egm@cs.byu.edu}
\alignauthor
Vivek Sarkar \\
        \affaddr{Rice University}\\
        \affaddr{Houston, Texas}\\
        \email{vsarkar@rice.edu}
}

    
% There's nothing stopping you putting the seventh, eighth, etc.
% author on the opening page (as the 'third row') but we ask,
% for aesthetic reasons that you place these 'additional authors'
% in the \additional authors block, viz.
\additionalauthors{Additional authors: John Smith (The Th{\o}rv{\"a}ld Group,
email: {\texttt{jsmith@affiliation.org}}) and Julius P.~Kumquat
(The Kumquat Consortium, email: {\texttt{jpkumquat@consortium.net}}).}
\date{30 July 1999}
% Just remember to make sure that the TOTAL number of authors
% is the number that will appear on the first page PLUS the
% number that will appear in the \additionalauthors section.

\maketitle
\begin{abstract}
The Habanero Java Library (HJ-lib) is a Java 8 library implementation of the
Habanero Java (HJ) programming model. Calls into this pure Java library provide
support for all HJ primitives, including async, finish, and phasers. In
previous work, we presented VR, a
custom verification runtime designed to be used within Java Pathfinder (JPF) to
verify a subset of HJ programs. In this work, we present VR-lib, a library
implementation of HJ, which supports verification of a larger subset of
programs than VR. Additionally, we present the implementation of gradually typed
permission regions (GPRs). PRs provide a building block for dynamically
detecting violations of conditions sufficient to guarantee race-freedom.
Lastly, we present results for benchmarks using PRs in combination with VR-lib
to verify HJ programs.  
\end{abstract}
%\category{D.2.4}{Software/Program Verification}{Model Checking}
%\terms{Model Checking}
%\keywords{Model checking, high performance computing, Java Pathfinder} % NOT required for Proceedings
%\section{Introduction}
A \emph{data-race} is where two concurrent executions access the same memory location with at least one of the two accesses being a write. It introduces non-determinism into the program execution as the behavior may depend on the order in which the concurrent executions access memory. Data-race is problematic because it is not possible to directly control or observe the run-time internals to know if a data-race exists let alone enumerate program behaviors when one does. 

The problem of \emph{Data-race detection}, given a program with its input, is to determine if there exists an execution containing a data-race. The research presented in this paper is concerned with data-race detection for \emph{task parallel models} that impose structure on parallelism by constraining how threads are created and joined and how shared memory is accessed (e.g., Cilk, X10, Chapel, Habanero, etc.). These models rely on run-time environments to implement task abstractions to represent concurrent executions \cite{blumofe1996cilk,charles2005x10,cave2011habanero,imam2014habanero}. The language restrictions on parallelism and shared memory interactions enable properties like \emph{determinism} (i.e., the computation is independent of the execution) or the ability to \emph{serialize} (i.e., removing all task related keywords yields a serial solution). Such properties are predicated on the input programs being data-race free which is not always the case since programmers both intentionally and unintentionally move outside the programming model.

Data-race detection in task parallel models generally prioritizes performance and the ability to scale to many tasks. The predominant \emph{SP-bags} algorithm, with its variants, exploits assumptions on task creation and joining for efficient on-the-fly detection with low overhead \cite{Feng1997EDD258492258493,Cheng1998DDR277651277696,Bender2004OMS10079121007933,Async-Finish-Race,Utterback2016PGP29357642935801}; millions of task are feasible with varying degrees of slow-down (i.e., slow-down increases as parallelism constraints are relaxed) \cite{drdForFutures,Surendran2016}. Other approaches use access histories \cite{mellor1991fly,raman2012scalable} or programmer annotations \cite{westbrook2012practical, westbrook2012permission}.
Performance is a priority, and many solutions are only \emph{complete}, meaning that nothing can be concluded about other executions of the same program, or become complete when parallelism is outside the assumptions.

The research presented in this paper reprises data-race detection in task parallel models in the context of model checking under two assumptions: first, data-race is largely independent of the size of the problem instance; and second, it is possible to instantiate small problem instances. These assumptions are indeed implicit in other model checking solutions for task parallel models---a small problem instance is the best solution to state explosion \cite{gligoric2012x10x,zirkel2013automated,anderson2014jpf}. Prior approaches, however, extensively modify the language run-time and the model checker necessitating source code for both \cite{gligoric2012x10x}, require the user to specify the number of processors modeled making it difficult to generalize \cite{zirkel2013automated}, or rely on user annotations to indicate sharing so data-race may be missed \cite{anderson2014jpf}. The solution here is to make clear the requirements on the run-time, use semantics that are independent of actual hardware, and automatically detect data-race without annotations. 

The approach first defines a \emph{computation graph} to abstractly model parallelism with a naive algorithm to detect data-race. Computation graph construction is then formally defined in a general task parallel model based on partitioning concurrent executions into hierarchical regions with shared locations. Such a model is suitable to describe real-world languages (e.g., Cilk, X10, Chapel, Habanero, etc.).  A fragment of the model is then defined so that data-race detection on a computation graph from a single execution is both sound and complete; this result is similar to existing dynamic analyses for task parallel languages. That fragment is then expanded to show how model checking may be applied to enumerate the space of computation graphs for data-race detection. Finally, the approach is evaluated on a Java implementation of Habenero with Java Pathfinder (JPF). Results over several published benchmarks comparing to JPF's default race detection and a task parallel approach with permission regions show the computation graph to be more efficient in JPF terms with its overhead. The primary contributions are thus
\begin{compactitem}
\item the computation graph construction in terms of general semantics suitable for real-world languages;
\item model checking as a means to exploring the space of computation graphs for a program; and
\item an implementation of the approach for Java Habanero in JPF with results from benchmarks comparing to other solutions in JPF. 
\end{compactitem}
\begin{comment}
Section \ref{sec:drd} defines computation graphs and data-race detection given a computation graph. Section \ref{sec:cg} is the programming model with graph construction. Section \ref{sec:otf-drd} is the model checking algorithm. Section \ref{sec:impl} gives an implementation of the algorithm for Habanero and section \ref{sec:res} discusses the results. Section \ref{sec:rel-work} discusses related work. Section \ref{sec:conclusion} presents the conclusion.
\end{comment}
\begin{comment}
  The increasing use of multi-core processors is motivating parallel programming. Earlier, the speed of processor cores was expected to increase with sustained technological advances and the need for parallel computing was low. Now that processor speeds are no longer increasing, parallelism is the only way of obtaining higher computing performance.

Writing concurrent programs that are free from bugs, however, is very difficult because when programs execute different instructions simultaneously, different thread schedules and memory access patterns are observed that give rise to issues such as data races and deadlocks. Structured parallel languages help users to write parallel programs that are scalable and easy to maintain \cite{blumofe1996cilk, charles2005x10, cave2011habanero}. These properties are achieved by imposing restrictions on the way tasks can be forked and joined.

Data races occur in parallel programs when two or more tasks access a shared memory location such that at least one of the accesses is a write. A race on a shared variable can alter the value of the variable based on the order in which the variable is accessed by the tasks causing the output to be non-deterministic. A data race that is not protected (i.e., marked volatile) also leads to behavior that is not sequentially consistent. It is hard to test all possible outcomes of the program with a data race because the scheduler most often does the same thing. 

A lot of research has gone into the problem of detecting data races in parallel programs. Data race detection techniques are mainly categorized as static, dynamic and model checking. Static race detectors analyze the programs statically and report errors \cite{engler2003racerx,ESC,abadi2006types,naik2006effective,voung2007relay,choi2001static, vechev2011automatic}. Their drawback is that they report data races on variables when in fact there are no data-races; this imprecision makes them hard to use in real world applications. Model checking on the other hand produces precise results but suffers from state space explosion making it impossible to use in large systems\cite{kulikov2010detecting, vakkalanka2008implementing, Godefroid, anderson2014jpf, gligoric2012x10x, zirkel2013automated}.

Dynamic data race detectors analyze the program at runtime and so the data races reported by them are real data races. Dynamic data race detectors however can reason about only a single run \cite{flanagan2009fasttrack, savage1997eraser, mellor1991fly, schonberg1989fly, Feng97efficientdetection, Async-Finish-Race}. Raman et al. created a dynamic race detector for structured parallel programs that can locate races in any schedule of the program by running the program only once using limited access history\cite{raman2012scalable}. Another approach for data race detection for programs with futures uses dynamic task reachibility graphs \cite{drdForFutures}. Both these approaches are efficient, however, they don't provide any functionality to manipulate the scheduler at runtime thereby being unsound for programs with mutual exclusion. When accesses to shared variables are protected, different program outcomes are observed. It is necessary to analyze all program behaviors to ensure data race freedom.

This paper introduces an improved technique for data race detection that combines dynamic race detection for structured parallel languages with model checking to overcome the limitations of both of them. This technique makes use of computation graphs to represent the happens-before relation of the events of the program in the form of a directed acyclic graph \cite{dennis2012determinacy}. The nodes represent the various tasks that are spawned during the program execution and store the references to shared heap locations that have been accessed by those tasks. To detect data races, the task nodes that can execute in parallel are identified in the graph and the memory locations stored in these nodes are compared to detect conflicts. For building such computation graphs, the runtime should have the ability to call-back when threads are forked or joined, and to record memory accesses on heap locations that may be shared.

The model checking part of the solution comes into play for programs with critical sections. In such programs, different computation graph structures can arise based on the order of execution of the critical sections. The technique presented here creates all such computation graphs using a scheduler that checks for critical sections and builds schedules to consider all possible ways in which the program can execute \cite{mercer2015model}. Hence, this method is sound for all programs with a given input.  

This paper presents an implementation of this data race detection technique for the Java implementation of the Habanero programming model. The implementation uses Java Pathfinder (JPF). JPF is a model checker for Java that is fully customizable using various programming patterns and interfaces. The implementation uses JPF's virtual machine to create a specialized runtime for the Habanero language that is targeted specifically for verification\cite{mercer2015model, anderson2014jpf}. 
The performance is compared with two other model checking approaches implemented by JPF: Precise Race Detector and Permission regions \cite{kulikov2010detecting}, \cite{mercer2015model}. The results show a significant reduction in the state space and time needed for verification.

\textbf{Main Contributions:}
  \vspace{-1em}
\begin{enumerate}
\item A data race detection algorithm using computation graphs that runs in $\mathcal{O}$(N$^2$) time where N is number of nodes in the graph.
\item Semantics for task parallel programs that include steps for creating computation graphs.
\item Dynamic improvement to the data race detection algorithm for structured parallel programs.
\item A scheduling algorithm to create all computation graphs for programs containing mutual exclusion.
\item An implementation of the data race detection algorithm for Habanero Java.
\item An empirical study over a set of benchmarks comparing performance of the data race detection algorithm to JPF.
\end{enumerate}
Some proofs are omitted for space but exist in a long technical report to be referenced if the paper is accepted.
\end{comment}

\begin{comment}
  The rest of the paper is divided as follows. Section \ref{sec:drd} introduces the concept of computation graphs for task parallel programs and discusses the data race detection algorithm based on computation graphs. Section \ref{sec:cg} presents syntax and semantics of task parallel languages and discusses the creation of computation graphs. Section \ref{sec:otf-drd} discusses dynamic improvement to the data race detection algorithm using on-the-fly analysis for structured parallel programs and a scheduler for programs with critical sections. Section \ref{sec:impl} gives implementation of the algorithm for HJ and section \ref{sec:res} discusses the results. Section \ref{sec:rel-work} discusses related work. Section \ref{sec:conclusion} presents the conclusion.
\end{comment}

\section{Introduction}
The increasing use of multicore processors is motivating the use of parallel programming. However, it is very difficult to write concurrent programs that are free from bugs. When programs execute different instructions simultaneously, different thread schedules and memory access patterns are observed that give rise to various issues such as data-races, deadlocks etc. To make writing concurrent programs easier, Rice University developed Habanero Java Programming model \cite{Cave:2011:HNA:2093157.2093165}. It provides safety guarantees such as deadlock freedom, deterministic output and serialization for subsets of constructs provided in the programming model. These guarantees hold only in the absence of data-races. The Habanero Java library (HJ-Lib) \cite{hj-lib} is a Java 8 library implementation of the Habanero Java programming model.

VR-lib \cite{Anderson:2015:JVH:2693208.2693245}, a verification runtime for HJ programs was built at Brigham Young University. VR-lib facilitates the verification of HJ programs using JPF. VR-lib can be used along with JPF to build computation graphs of HJ programs. A Computation Graph (CG) is an acyclic directed graph that consists of a set of nodes, where each node represents a step consisting of some sequential execution of the program and a set of edges that represent the ordering of the steps. A CG stores the memory locations accessed and updated by each of the operators. It also correctly reflects the control flow structure of the program.

The CGRaceDetector listener presented in this work monitors the various object creations and destructions, instruction executions etc to build a computation graph for the HJ program under a single schedule. It later analyzes this graph to verify any data access violations to report data races. For structurally deterministic programs, verifying the HJ program under a single schedule is enough to detect data races.

Section 2 of this paper presents an overview of the Habanero Java programming model and gives a brief description for the various parallel constructs of HJ language. Section 3 describes the computation graphs and its various elements. It also gives the implementation details of computation graph builder and analyzer for HJ programs created with the help of JPF. Section 4 describes the preliminary results of the CGRaceDetector on some HJ micro-benchmarks. Section 5 concludes and outlines the ways to  extend this work.

\begin{figure}[t]
\centering
\includegraphics[width=3.25in]{../figs/async-finish}
\caption{An example with {\tt async} and {\tt finish}.}
\label{fig:async-finish}
\end{figure}


\section{Habanero Programming Model}

The Habanero programming model is built around a task-parallel view of
concurrency \cite{Cave:2011:HNA:2093157.2093165}. \figref{fig:async-finish} illustrates Habanero in its
simplest form \cite{Cave:2011:HNA:2093157.2093165}.

The \texttt{async}-construct is a mechanism for
creating a new asynchronous task: {\tt async}
$\langle${\em stmt}$\rangle$ causes the calling task (i.e., the
parent) to create a new child task to execute {\em
  $\langle$stmt$\rangle$} (logically) in parallel with the parent
task. {\em $\langle$stmt$\rangle$} can read or write any data in the
heap and can read (but not write) any local variable belonging to the
parent task's lexical scope. The task created by any
\texttt{async}-construct is scheduled at the point it is declared in
the program.

The \texttt{finish}-construct is a generalized join operation for
collective synchronization: {\tt finish} $\langle${\em
  stmt}$\rangle$ causes the parent task to execute {\em
  $\langle$stmt$\rangle$} and then wait until all tasks created within
{\em $\langle$stmt$\rangle$} have completed, including transitively
created tasks.  Each dynamic instance of a task has a unique {\em
  immediately-enclosing-finish} (IEF) during program execution. That IEF is the
innermost {\tt finish}-construct containing the task.  There is an implicit {\tt
  finish}-construct surrounding the entry point of the program so the program only terminates after
all tasks have completed.

A computation graph illustrating the semantics of the \texttt{async}
and \texttt{finish} constructs is on the right side of
\figref{fig:async-finish}. In the graph, task $T_0$ enters the
\texttt{finish}-construct, creates task $T_1$ at the
\texttt{async}-construct, and then continues on to
\texttt{STMT2}. After \texttt{STMT2}, $T_0$ waits for $T_1$ to
complete before moving on to \texttt{STMT3}. Note that \texttt{STMT1}
and \texttt{STMT2} are not ordered by the semantics and represent
parallel execution.

Habanero supports more advanced forms of tasking beyond creation and
collective synchronization. The \texttt{isolated}-construct, {\tt isolated}~$\langle${\it
  stmt1}$\rangle$, ensures that $\langle${\it stmt1}$\rangle$ is
evaluated in mutual exclusion with all other {\tt
  isolated}-constructs.  There are two subtle nuances in the Habanero
model for the \texttt{isolated}-construct:
\begin{compactenum}
\item The construct ensures mutual exclusion between \texttt{isolated}-constructs and not mutual exclusion on a particular memory location. Mutual exclusion on a particular memory location is implemented by wrapping operations on that memory location in \texttt{isolated}-constructs. 
\item Any Habanero implementation may relax mutual-exclusion between
  \texttt{isolated}-constructs as long as the constructs do not
  interfere with one another. Interference in this context means that
  multiple \texttt{isolated}-constructs access a common memory
  location and at least one of those accesses is a write.
\end{compactenum}

The \texttt{future}-construct lets tasks
return values to other tasks: \textbf{future} {\em f} $=$ \textbf{async}
  $\langle${\em expr}$\rangle$ creates a new child task to evaluate $\langle${\em expr}$\rangle$.  The local
variable {\em f} contains a \emph{future handle} to the newly created
task that can be used to obtain the value produced by $\langle${\em expr}$\rangle$. The blocking operation {\em f.get()} returns that value when the
child task completes.

The most complex construct in the Habanero model is the
\textit{phaser} \cite{Shirako:2008:PUD:1375527.1375568}. A phaser is a form of a barrier that provides
point-to-point fine-grain synchronization between tasks to coordinate
their movement through \emph{phases} of computation. Like barriers, phasers order execution of
portions of the program into phases and restrict tasks from
entering the next phase until the current phase is complete. Unlike
barriers though, phasers allow tasks to specify point-to-point relationships on
multiple phasers, and tasks can dynamically join or leave the phaser.

Tasks register with an instance of a phaser, and on registration,
declare the mode that control how that task
synchronizes relative to other tasks registered on the same
barrier. Synchronization takes place with the \texttt{next}-construct
which may block depending on the state of the phaser, and on how the task is registered with the phaser.
\begin{compactitem}
\item \texttt{SIG}: signal registration means all tasks that have designated themselves as signalers must signal the phaser
in order for the phase to advance.  The \texttt{next}-construct for a signal-only task signals the phaser and immediately advances to the next phase.  The phaser remembers each phase completed by any task.
\item \texttt{SIG\_WAIT}: \emph{signal-wait} registration means the task signals the phaser and then waits for other tasks to complete the phase. This registration mode functions like a traditional barrier. The \texttt{next}-construct for a signal-wait task reports phase completion and then blocks for the other signalers to complete the phase too.
\item \texttt{WAIT}: \emph{wait} registration means that the task blocks at the \texttt{next}-construct until the phase advances. 
\end{compactitem}
Phasers may also be bounded to specify slack in the number of phases that may separate
waiters and signalers so signalers can work ahead of waiters
up to a bound.\footnote{Omitted in this presentation of phasers is the ability of a single task to execute constructs after the end of one phase and before the start of the next phase.}

Habanero includes several other constructs such as
\texttt{foreach}-constructs, \texttt{forall}-constructs, \emph{data
  driven futures}, \emph{actors}, etc. most of which are syntactic
sugar for the presented constructs.


\section{Gradual Permission Regions}

\begin{figure}[t]
  \centering
  \begin{lstlisting}
<T> void acquireR(T xs)
<T> void acquireR(T xs, int idx)
<T> void acquireR(T xs, int start, int end)

<T> void releaseR(T xs)
<T> void releaseR(T xs, int idx)
<T> void releaseR(T xs, int start, int end)
  \end{lstlisting}
\caption{The permission-region annotation interface for read acquisition and release in \jpf.}
\label{fig:pr-interface}
\end{figure}

\begin{figure}
  \begin{center}
    \begin{lstlisting}
public static void main(final String[] argv) {
  launchHabaneroApp(() -> {
    Stack stk = initStack();  

    finish(() -> {

      async(() -> {
        acquireW(stk);
        stk.push(5);
        releaseW(stk);
      });
      
      acquireR(stk);
      stk.peek();
      releaseR(stk);
    });
  });
}
\end{lstlisting}
  \end{center}
  \caption{The \hj\ program from \figref{fig:hj-async-finish} with additional permission region annotations.}
  \label{fig:hj-async-finish-pr}
\end{figure}

\begin{figure}[t]
\centering
\includegraphics[width=3.25in]{../figs/state-machine}
\caption{State machine for permission regions operating on a single object.}
\label{fig:state-machine}
\end{figure}


\begin{figure}[t]
\centering
\includegraphics[width=3.25in]{../figs/stack-violation}
\caption{Different schedules for the program in \figref{fig:hj-async-finish} with the right-most schedule detecting a violation.}
\label{fig:permission-violation-state}
\end{figure}

Gradual permissions with permission regions is a hybrid
static-dynamic approach to detecting data-race in task-parallel
programs \cite{Westbrook:2011:PRR:2341616.2341627,hj-grad-perm}. The
programmer annotates regions of the program text that access shared
objects. Those regions are indicated as accessing shared objects in
read mode or write mode. When the program runs, a state machine is
associated with each shared object to track access permissions on that
object as indicated by the program annotations. If access permissions
from distinct tasks on the same object conflict, then a dynamic run-time
error is reported.

Permission regions are distinctly different from
\texttt{isolated}-constructs. Foremost, \texttt{isolated}-constructs define atomic regions that run mutually
exclusive to other regions in \texttt{isolated}-constructs. As such, isolation
restricts concurrency by serializing atomic regions. Permission regions
do not serialize atomic regions to restrict concurrency. They only
check if concurrent accesses to atomic regions are free of data-race. Isolation is
always best avoided given its impact on speedup in Amdahl's law.

Permission regions are annotated for \jpf\ using the interface in
\figref{fig:pr-interface}. Although the figure only shows the read
interface, the write interface follows the same pattern. The first method on the 
interface has arguments to manage permissions on a single object. The
second and third methods have arguments to manage permissions on
arrays. Arrays are somewhat more nuanced. Consider the following code:
\begin{lstlisting}
  f[i] = new C();
  f[i].write(j);
\end{lstlisting}
The first line writes to the array location $i$. The second line
writes to the object stored in the array location $i$. These two
accesses must be treated separately. As such, the permission regions interface provides
methods to manage permissions on a specific index in an array or a
range of indexes in the array. Permissions on the actual object in the array use the first method in the interface. The annotated code snippet from above is as follows.
\begin{lstlisting}
  acquireW(f,i);
  f[i] = new C();
  releaseW(f,i);
  
  acquireR(f,i);
  acquireW(f[i]);
  f[i].write(j);
  releaseW(f[i]);
  releaseR(f,i);
\end{lstlisting}

\figref{fig:hj-async-finish-pr} is the permission-region annotated
version of the program in \figref{fig:hj-async-finish} from the
previous section. The operations on the shared object \texttt{stk} are
now wrapped in calls to acquire or release permissions on the shared
object. Note that the permission regions span all the code 
in the \texttt{stk.push} and \texttt{stk.peek} methods, so anytime the object is referenced, it is covered. In this
way, permission regions can be as large or as small as desired. If the
regions are too large, however, then the approach may report a data-race
where no race exists
\cite{Westbrook:2011:PRR:2341616.2341627,hj-grad-perm}. 

\figref{fig:permission-violation-state} is the state-machine to track
permissions on shared objects and detect violations. That machine is included here for convenience directly
from \cite{hj-grad-perm}. The machine starts in the double-circled
\textit{Null} state. On acquisition or release, the machine updates to
the appropriate state based on its current state. The machine signals
a violation if it ever detects conflicting accesses by different tasks.

\figref{fig:permission-violation-state} shows two possible schedules
for the annotated program in \figref{fig:hj-async-finish-pr}. The solid
filled ovals and solid lines represent the main task and the dotted
filled ovals and dashed lines represent the task created by the
\texttt{async}-statement. The squares indicate the current state of
the state machine that is tracking accesses to the shared object
\texttt{stk}.

The left branch of the tree is the schedule where the main task runs
until it is blocked by the \texttt{finish}-statement where it must
join with the tasked created by the \texttt{async}-statement. The main
task acquires and releases private read privileges on the region
before it blocks. After the main task blocks, the newly created task
runs, acquiring and releasing private write privileges, and it then
exits. If this schedule is followed in the run-time, then no
violation is reported even through data-race exists in the
program. The approach is run-time dependent and not exhaustive.

The right branch is an alternate schedule that is possible in the
program. In this schedule, the newly created task from the
\texttt{async}-statement runs just after the main task acquires private
read privileges on the shared object. When the new task tries to
acquire write privileges, the state-machine that manages permissions on
the shared object moves into the violation state to report the error.

\subsection{Permission Regions in \jpf}

The implementation of permission regions in \jpf\ spans 1036
lines of code and covers 11 distinct class objects. It leverages
\jpf's ability to track thread IDs of all accesses to objects, so it not only reports
violations on the permission regions, but also identifies shared
accesses that are not annotated by permission regions or covered by \texttt{isolated}-constructs. In this way,
\jpf\ updates the user when a shared access has been missed in the
annotations.

The implementation uses two key features of \jpf: byte-code listeners
and object attributes. It installs a byte-code listener to watch for
instances of the \texttt{INVOKE}-code. The actual methods for the permission regions interface in
\figref{fig:pr-interface} are empty stubs. When
the listener sees an instance of the \texttt{INVOKE}-code that calls a
method on the interface, it gets the method's parameters from
the stack and updates the associated state-machines appropriately.

The state machines themselves reside in an attribute of the object. Every object
in \jpf\ has an associated attribute that can hold arbitrary information. For
example, attributes are used to implement symbolic execution in \jpf\
\cite{DBLP:journals/ase/PasareanuVBGMR13}. The important property of attributes
is that they follow heap objects through the entirety of state space
exploration. The state machines to track permission region accesses are stored
in those attributes. For arrays, a separate permissions state-machine is stored
for every index. The program annotations acquire and release permissions on
individual indexes (or a range of indexes) as mentioned previously.

With or without permission regions, \jpf\ finds the data-race in
\figref{fig:hj-async-finish-pr} using its built-in precise-data-race
listener with \hjv. Unfortunately, \jpf\ times-out on larger programs due to
state-explosion as shown in the results section. Permission regions
are utilized to improve this limitation.

\section{JPF and VR}

JPF and VR both use Java reflection. When the user verifies an HJ program, the user runs JPF with VR as the target, and the HJ program as the input to the target. JPF uses reflection to open VR with the HJ program as an argument. VR in turn uses reflection to create a new thread to execute the HJ program and gives any remaining arguments as arguments to the HJ program. Reflection is intentional because VR has to be flexible to run an HJ program specified by the user and still provide functionality to HJ semantics. However, any property violations that JPF finds are within the HJ program and not within VR or JPF.

\begin{comment}
\begin{figure}
\begin{center}
{\small
\begin{verbatim}
public static void main (String args[]) {
   Class<?> verifyClass = 
      Class.forName((args[0])+"$Main");
   Class<?> stringArrayClass = args.getClass();
   Constructor<?> constructor = 
      verifyClass.getConstructor(stringArrayClass);
   String[] newArgs = new String[args.length -1];
   for (int i = 0; i < args.length-1; i++)
      newArgs[i] = args[i+1];
   java.lang.Object object = 
      constructor.newInstance(
         (java.lang.Object) newArgs);
   Class<?> thread = Class.forName("java.lang.Thread");
   Method method = thread.getDeclaredMethod("start");
   method.invoke(object, (java.lang.Object[]) null);}
\end{verbatim}
}
\end{center}
\caption{A simplified version of the main method in VR.}
\label{fig:main}
\end{figure}
\end{comment}

VR heavily relies on the soundness of JPF, and JPF's race detector. JPF's soundness is based on its partial order reduction and global search object ID. JPF's partial order reduction is critical: JPF must produce and examine essential interleaves for each thread and stop at locations where a data race occurs for checking. JPF does this by creating \texttt{ChoiceGenerators}, copying the current state of the machine, at certain parts of thread execution. JPF then systematically explores the state space, executing one thread at a time, which is called state expansion. The global search object ID is necessary for checks to ensure objects used by two threads are the same, even at different choice generators. JPF's \texttt{PreciseRaceDetector} from the default distribution is important because it checks for and reports data races that JPF has located.

VR was made with the intention to be used for verification, specifically by JPF. It was built in the simplest way possible to still produce correct results. We did not include any specific optimizations. To increase efficiency, an optimized scheduler can be integrated that is more suited with VR for scheduling, state expansion, and executing different interleavings than the default JPF scheduler factory.

The default JPF scheduler factory produces choice generators at several key locations in the partial order reduction, such as thread creation, thread termination, shared field access, monitor access, etc. However, not all state expansions from choice generators are interesting or relevant to VR. The optimized scheduler will only create choice generators for thread creation, shared array element access, and shared field access. \figref{fig:pseudocode} shows  psuedocode for the optimized scheduler. The optimized scheduler thus creates a smaller state space for JPF to verify than the default scheduler and removes the unnecessary choice generators that the default scheduler gives: thread suspense, thread resume, thread sleep, thread interrupt, thread yield, thread notify, thread terminate, shared object expose, wait, monitor enter, monitor exit, sync method enter, and sync method exit. Even after removing all these choice generators, the optimized scheduler still offers the same data race freedom guarentees that the default scheduler does. The optimized scheduler is given in the same repository as VR.

\begin{figure}
\begin{center}
{\small
\begin{verbatim}
createCG (string type, Thread[] threadsAvailable) {
    if (type.equals("THREAD_START"))
      createDefaultCG(type,threadsAvailable);
    else if (type.equals("SHARED_ARRAY_ACCESS"))
        createDefaultCG(type,threadsAvailable);
    else if (type.equals("SHARED_FIELD_ACCESS"))
        createDefaultCG(type,threadsAvailable);
    else
        noChoiceGenerator();}
\end{verbatim}
}
\end{center}
\caption{Psuedocode for the optimized scheduler.}
\label{fig:pseudocode}
\end{figure}

An informal proof for the optimized schedule factory's soundness and correctness follows. Suppose an HJ program data races on Thread A and B. In order for the two threads to data race, the threads will need to be alive at one point of execution. \figref{fig:proof} shows a simplified execution process for JPF using the optimized scheduler to find a data race with \texttt{ChoiceGenerators} with creation of new threads and shared field accesses.

First, there is a \texttt{ChoiceGenerator} that has these two threads before the shared field is accessed (such as the creation of thread B). Thread A is executed first (i). The shared field is accessed by A, and JPF stores this access using global search object IDs. However, on this execution, JPF is unaware that Thread B accesses the shared field, and continues running. Thread A terminates, and JPF switches to a different thread, though a new \texttt{ChoiceGenerator} is not created. Thread B executes, and runs till B accesses the shared field. JPF stores this access using the global search object ID and finds that A also accesses it. However, since Thread A no longer is alive, a \texttt{ChoiceGenerator} is not created, and B continues to run till it terminates.

At the end of the execution, JPF back tracks to the last \texttt{ChoiceGenerator} (ii) and picks to execute Thread B. JPF executes B until B accesses the shared field. The global search object ID once again determines that Thread A also accesses the field. However, since Thread A is alive, a \texttt{ChoiceGenerator} is created. Thread B continues (iii) execution until termination, and Thread A then executes. As before (i), Thread A executes till it reaches the shared field, but finds that Thread B is no longer alive, so no \texttt{ChoiceGenerator} is created. Thread A terminates, and JPF backtracks to the last \texttt{ChoiceGenerator} and executes a different thread (iv). Thread A is selected and executed. When Thread A reaches the shared field access, JPF checks its global search object ID, finds that Thread B is still alive, and inserts another \texttt{ChoiceGenerator}.

At this point, JPF's \texttt{PreciseRaceDetector} searches at the \texttt{ChoiceGenerator} (and has been doing at all \texttt{ChoiceGenerators} thus far) for two threads that have a shared field access on the same field. If there are, which Thread A and B are, then \texttt{PreciseRaceDetector} makes an additional check for at least one being a write in order to report a data race. Since Thread A and B do, JPF with the optimized scheduler reports a data race.
\begin{figure}[t]
\begin{center}
\includegraphics[width=3.25in]{../figs/InformalProofDiagram}
\end{center}
\vspace{-10pt}
\caption{A JPF search illustrating the scheduler.}
\label{fig:proof}
\end{figure}




\section{Results}

\begin{table*}[t]
\centering
\caption{Verification of HJ Micro-benchmarks using CGRaceDetector}
\label{tab:perf}
\begin{tabular}{|c|c|c|c|c|c|c|c|c|}
\hline
        &        & \multicolumn{3}{c|}{CGRaceDetector} & \multicolumn{3}{c|}{Precise Race Detector}
 \\ \hline
Test Case Name & SLOC & Tasks & States  & Time   & Error Info & States  & Time   & Error Info 
\\ \hline
Search Count & 50 & 4 & 195 & 0:00:01 & No Race & 145139 & 0:00:45 & No Race 
 \\ \hline
Existence of an occurrence & 45 & 4 & 174 & 0:00:01 & Detected Race & 50197 & 0:00:15 & Detected Race 
\\ \hline
Index of occurrence & 38 & 4 & 197 & 0:00:01 & Detected Race & 68806 & 0:00:29 & Detected Race 
\\ \hline
Existence of occurrence with no task & 45 & 2 & 117 & 0:00:00 & Detected Race & 296 & 0:00:00 & Detected Race
\\ 
creation after instance is found & &  &  &  & & & &
\\ \hline
Search Index With No task creation & 48  & 2 & 119 & 0:00:00 & Detected Race & 326 & 0:00:00 & Detected Race
\\
after Instance is Found &  &  &  & & & & &
\\ \hline
\end{tabular}
\end{table*}

We verified some of the HJ microbenchmarks that make use of only the basic parallel constructs such as async and finish using the CGRaceDetector listener. The CGRaceDetector is able to build computation graphs of the HJ programs by exploring very few states. We compared the output of CGRaceDetector to the output of PreciseRaceDetector and found that CGRaceDetctor was able to correctly identify races in all programs. These micro-benchmarks are variations of a linear search algorithm. The first test finds the count of occurrences of a search string in a given text string. The second test confirms the existence of search string in the given text string. The third test returns the index of occurrence of the search string. In case of multiple occurrences, the output becomes non-deterministic. The fourth test also confirms the existence of the search string in the given text. However, as soon as the search text is found, no more processes are spawned to search the text and the program is terminated. Similarly, in the fifth test, as soon as a process returns the index of occurrence of search text, the program terminates. The results are presented in Table I. The sizes of the programs are indicated by the SLOC column and Tasks columns represents the number of tasks created in every program. The results of CGRaceDetector and Precise Race Detector are compared. The Precise Race Detector systematically explores the entire state space of the program. The CGRaceDetector just uses one thread interleaving to detect data races. Hence, the time required by CGRaceDetector is considerably smaller than the time required by Precise Race Detector to execute.



\section{Conclusions \& Future Work}
This paper presents a model checking algorithm to prove when a
Habanero program does not contain any data races, deadlocks, assertion
violations, or exceptions for a given program input. The algorithm,
based on permission regions, only considers scheduling points in the
search tree at the boundaries of permission regions and {\tt
  isolated}-constructs. The paper includes a proof of soundness for
the algorithm, meaning that the algorithm may reject a correct program
due to the size of the permission regions.

The effectiveness of the algorithm is shown in several benchmark
programs that cover many of the Habanero concurrency constructs. The
analysis is done using a new Java library implementation of the Habanero
runtime that is intended for debugging and verification. The new algorithm, with permission regions,
is implemented as an extension to the \jpf\ model checker. The results
from the benchmark programs indicate a significant cost reduction when
using the new algorithm.

Future work includes automating the annotation of permission regions
based on the sharing detection in \jpf; automating the validation of any
counter-example; developing techniques to automatically refine
permission regions from counter-examples when needed; a partial order
reduction over permission regions; static-analysis to prevent
scheduling on regions that cannot race; applying symbolic techniques to reason over input; and studying
benchmarks that are representative of real world Habanero programs.


\bibliographystyle{abbrv}
\bibliography{../bib/paper}  
\balancecolumns
\end{document}
