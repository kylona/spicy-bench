\section{Results}

A new Java library was implemented to evaluate \algoref{algo:search}
in the \jpf\ model checker (\hjv). The library uses Lambda support in Java 8,
and it is purposed for verification in \jpf. \algoref{algo:search}
itself is an extension to \jpf\ that implements permission regions and
the search (\jpfhj).  Each is briefly discussed before the
results from several benchmark programs are presented.

\subsection{\hj\ Library for Verification}

\hjv\ is a new Java library implementation of the Habanero model designed
specifically for debugging and verification. It consists of roughly 1,300
lines of code in 32 classes. Most of the classes address the
programmer interface rather than the library
internals. \figref{fig:hj-async-finish} is the interface
using Java 8 Lambda functions and is identical to other Java library implementations of
the Habanero model \cite{hj-lib}.

\begin{comment}
The implementation of the \texttt{async} and \texttt{finish}
constructs uses Java threads and the ability to join those
threads. Flattening the Lambda function in the Java 8 interface to an anonymous inner class elucidates the
structure of the library. That code for the \texttt{async} call in
\figref{fig:hj-async-finish} using an anonymous inner-class is shown below:
\begin{lstlisting}
  async(new HjRunnable() {
    public void run() {
      X.push(5);
    }
  });
\end{lstlisting}
The parameter for the call is an instance of an \texttt{HjRunnable} object
object, and an \texttt{HjRunnable} is an extension to the standard Java
thread. The \texttt{run} method for the thread is specialized in the
anonymous inner-class. A programmer may use either syntax with \hjv.

Staying at a high-level view of the implementation, tasks are threads
with extra information to implement the Habanero model. To support the
\texttt{finish}-construct, that thread includes the notion of a
\emph{finish-scope}. A finish-scope holds references to any child
thread created within a \texttt{finish}-construct, and a stack of
finish-scopes tracks the nesting of \texttt{finish}-constructs within
a task. When a task is created, it is added to the current running
thread's active finish-scope. In this way, when a parent reaches the
end of a \texttt{finish}-construct, it is able to join on all threads
in the current finish-scope. After joining, the finish-scope is popped
from the stack making the next outer finish-scope the active scope.
\end{comment}

\hjv\ supports all of the constructs in the Habanero model including
phasers. To increase confidence in the correctness of \hjv, test cases were created
to utilize specific features of the runtime. Each of these test
cases were run within \jpf\ with full scheduling enabled (i.e., it
schedules on every bytecode related to thread synchronization or
sharing). Thus, for each case, \jpf\ is used to determine that \hjv\ is
free of data races and deadlocks. In total, 22 test cases were created
consisting of approximately 1,000 lines of source code.

\subsection{\jpf\ Implementation}

The implementation of permission regions in \jpf\ spans 1,036
lines of code and covers 11 distinct class objects. It leverages
\jpf's ability to track thread IDs of all accesses to objects, so it not only reports
violations on the permission regions, but it also identifies shared
accesses that are not annotated by permission regions or covered by \texttt{isolated}-constructs. In this way,
\jpf\ updates the user when a shared access has been missed in the
annotations.

The implementation uses two key features of \jpf: byte-code listeners
and object attributes. It installs a byte-code listener to watch for
instances of the byte-code that calls methods. The actual methods for
the permission regions interface are empty stubs, and when the
listener activates on the interface, it gets the method's parameters
from the stack and updates the associated state machines
appropriately. The state machines themselves reside in an attribute of the object \cite{DBLP:journals/ase/PasareanuVBGMR13}. The important property of attributes
is that they follow heap objects through the entirety of state space
exploration. For arrays, a separate permissions state machine is stored
for every index. 

The \jpf\ implementation of \algoref{algo:search} exploits the
extensible nature of the tool by providing a new
\emph{scheduling-factory}.  A scheduling-factory is activated on
preemption, when a thread is no longer able to run, or if there is
input non-determinism.  It decides what
threads are scheduled by inserting
\emph{choice-generators} into the state search to enumerate the available choices. The search iterates over those
choices starting a new search for each choice.

The default scheduling-factory of \jpf\ is replaced with a new factory
that does not insert any choices on thread actions, locks,
synchronization, or shared accesses to objects. Anything related to
concurrency is turned off except for forced context switches such as a
thread exiting or a thread blocking. In those cases, the new
scheduling-factory inserts a choice generator with a single choice
that represents a random thread that is runnable.

To insert the preemption points for permission regions and
\texttt{isolated}-constructs, the byte-code listener from the
implementation of permission regions is extended to also listen for
the calls to \texttt{isolated}. At the
entrance to permission regions, the permission regions' state machine for the
object is updated as before, but after the update, a choice-generator
is inserted into the search that includes choices for all runnable
threads. Similarly, a choice-generator is inserted at the
\texttt{isolated} call. The entire factory with the listener extension is only a few
hundred lines of code but significantly reduces the verification
cost.

\emph{Benchmark Results}

\jpf's default code to detect data races is named
\emph{PreciseRaceDetector}. \tableref{tab:perf} compares the performance
of \jpfhj\ with \emph{PreciseRaceDetector} over several benchmark
programs taken from materials used to teach the Habanero model or test the Habanero
runtimes. These programs are not necessarily indicative of actual Habanero
programs in the real world but do contain the breadth
 of \hj\ constructs: {\tt async}, {\tt isolated},
{\tt finish}, {\tt future}, and phasers. Many benchmarks also include
arrays and shared arrays. The sizes of the benchmarks are
indicated by the \emph{SLOC} column (i.e., the number of program locations)
and the \emph{Tasks} column (i.e., the number of tasks created).

In \tableref{tab:perf}, an entry of N/A in the time column indicates a
running time greater than 30 minutes on a Macbook Pro, core I7, with 8
Gb of ram. The \emph{Error} column indicates if a data race or deadlock is discovered. An entry of \emph{Detected Race*} is an incorrect
result: the program is actually free from data races.

The first thing to notice in the table is that the
\emph{PreciseRaceDetector} is more likely to not complete in the time
bound due to exploring an excessive number of task schedules: it
schedules on individual byte-codes where \jpfhj\ schedules on
regions. The second thing to notice is that the
\emph{PreciseRaceDetector} reports data races in \emph{TwoDimArrays},
\emph{Add}, and \emph{ScalarMultiply} benchmarks where no data races
exists. This error report is a limitation of
\emph{PreciseRaceDetector} where it is not able to discern accesses to
the arrays on disjoint indexes; rather, it reports an error because the access on the array object looks like a data race and exits. If \jpf\ is configured to not exit on error discovery, then the entries become N/A. That is why the \emph{Time} entries on those
examples appear to improve over the entries for \jpfhj.

The final thing to notice in the table is that the cost of
verification is difficult to predict based solely on the static number of
region annotations, as it depends not just on that number, but the size
of the regions, and how many tasks contain the regions. For example, the
\emph{PrimeNumberCounter} and its variants have very few regions, as
defined by the annotations in the code, but those regions are part of every created
task, so those benchmarks have a much larger number of schedules to explore.

\begin{table*}
\centering
\caption{Performance Comparison between \jpfhj\ and \emph{PreciseRaceDetector}}
\label{tab:perf}
\scalebox{0.9}{\begin{tabular}{|c|c|c|c|c|c|c|c|c|c|}
\hline
        &      &       & \multicolumn{4}{c|}{Permission Regions} &
\multicolumn{3}{c|}{PreciseRaceDetector} \\ \hline
Test ID & SLOC & Tasks & States  & Time  & Regions  & Error Note & States      &
Time      & Error Note     \\ \hline
PrimitiveArrayNoRace & 29 & 3 & 5 & 0:00:00 & 0 & No Race & 11,852 & 0:00:00 &
No Race \\ \hline
PrimitiveArrayRace & 39 & 3 & 5 & 0:00:00 & 2 & No Race & 220 & 0:00:00 &
Detected Race \\ \hline
TwoDimArrays & 30 & 11 & 15 & 0:00:00 & 0 & No Race & 597 & 0:00:00 &
Detected Race* \\ \hline
ForAllWithIterable & 38 &  2 & 9 & 0:00:00 & 0 & No Race & N/A & N/A & N/A
\\ \hline
IntegerCounterIsolated  & 54 & 10 & 1,013,102 & 0:05:53 & 3 & No Race & N/A & N/A 
& N/A \\ \hline
PipelineWithFutures & 69 & 5 & 34 & 0:00:00 & 1 & No Race & N/A & N/A & N/A
\\ \hline
SubstringSearch & 83 & 59 & 8 & 0:00:00 & 2 & Detected Race & N/A & N/A & N/A
\\ \hline
BinaryTrees & 80 & 525 & 632 & 0:00:03 & 0 & No Race & N/A & N/A & N/A
\\ \hline
PrimeNumCounter & 51 & 25 & 231,136 & 0:01:08 & 2 & No Race & N/A & N/A & N/A
\\ \hline
PrimeNumCounterForAll & 52 & 25 & 6 & 0:00:00 & 2 & Detected Race* & N/A & N/A & N/A
\\ \hline
PrimeNumCounterForAsync & 44 & 11 & 449,511 & 0:02:51 & 2 & No Race & N/A & N/A
& N/A \\ \hline
ReciprocalArraySum & 58 & 2 & 32 & 0:00:06 & 2 & No Race & N/A & N/A & N/A
\\ \hline
Add & 67 & 3 & 62,374 & 0:00:33 & 6 & No Race & 4,930 & 0:00:03 & Detected Race*
\\ \hline
ScalarMultiply & 55 & 3 & 55,712 & 0:00:30 & 2 & No Race & 826 & 0:00:01 & Detected Race*
\\ \hline
VectorAdd & 50 & 3 & 17 & 0:00:00 & 4 & No Race & 46,394 & 0:00:19 & No Race
\\ \hline
AsyncTest1 & 23 & 51 & 54 & 0:00:00 & 0 & No Race & N/A & N/A & N/A
\\ \hline
AsyncTest2 & 32 & 3 & 4 & 0:00:00 & 2 & Detected Race & 11,534 & 0:00:04 
& Detected Race \\ \hline
FinishTest1 & 32 & 3 & 6 & 0:00:00 & 0 & No Race & 2,354 & 0:00:02 & No Race
\\ \hline
FinishTest2 & 33 & 3 & 5 & 0:00:00 & 0 & No Race & 25,243 & 0:00:09 & No Race
\\ \hline
FinishTest3 & 44 & 4 & 7 & 0:00:00 & 0 & No Race & 34,459 & 0:00:12 & No Race
\\ \hline
ClumpedAccess & 30 & 3 & 15 & 0:00:00 & 2 & No Race & N/A & N/A & N/A
\\ \hline
\end{tabular}}
\end{table*}

