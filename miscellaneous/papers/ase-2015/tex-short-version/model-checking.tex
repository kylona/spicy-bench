\begin{algorithm}
\caption{Permission Region Informed Search}\label{algo:search}
\begin{algorithmic}[1]
  \Function{search}{$t$, $h$, $T$}
  \State \textbf{loop:}\ ($h$, $T$) $:=$ \texttt{run}($t$, $h$, $T$)\label{loc:run}
  \State
  \State \textit{s} $:=$ \texttt{status}($t$, $T$)
  \State \textit{permission\_violation} = \textbf{false}
  \If{\textit{s} $=$ PR\_ENTRY} \label{loc:PR}
  \State ($h$, $T$, \textit{permission\_violation}) $:=$ \texttt{acquire}($t$, $h$, $T$)\label{loc:acquire}
  \ElsIf{\textit{s} $=$ PR\_EXIT}
  \State ($h$, $T$) $:=$ \texttt{release}($t$, $h$, $T$)\label{loc:release}
  \State \textbf{goto} loop
  \EndIf
  \State
  \If{\textit{permission\_violation}}\label{loc:datarace}
  \State report permission\_violation and exit
  \EndIf
  \State
  \State $R = $ \texttt{runnable}($T$)
  \If{$R = \emptyset$}
  \If{blocked($T$) $\neq \emptyset$}\label{loc:deadlock}
  \State report deadlock and exit
  \Else
  \State report any uncovered sharing and exit\label{loc:term}
  \EndIf
  \EndIf
  \State
  \If{$(h,T) \not\in S$}\Comment{$S$ is a global variable}\label{loc:visited}
  \State $S = S \cup \{(h, T)\}$
  \If{\textit{s} $=$ PR\_ENTRY $\vee$ \textit{s} $=$ ISOLATED}\label{loc:entry:isolated}
  \ForAll{$t_i \in R$}\label{loc:prsched}
  \State \texttt{search}($t_i$, $h$, $T$)
  \EndFor
  \Else
  \State $t_i$ := \texttt{random}($R$)\label{loc:rand}
  \State \texttt{search}($t_i$, $h$, $T$)
  \EndIf
  \EndIf
\EndFunction
\end{algorithmic}
\end{algorithm}

\section{\jpfhj\ Search Algorithm}

Permission regions create natural scheduling boundaries for model checking that can be
leveraged to mitigate state explosion while preserving the essential
behaviors of the program that lead to data races, deadlocks, assertion violations, or exceptions since they represent points of execution where sharing is expected.  The intuition is that given a fixed
program input, erroneous behavior can only arise from interactions between tasks
on shared memory. As such, it is only necessary to preempt running
tasks at the entrance to permission regions and
\texttt{isolated}-constructs. If a program has any deadlocks, data
races, assertion violations, or exceptions for a fixed program input,
then such a deadlock, data race, assertion violation, or exception exists
in one of the schedules that is explored from those preemption points.

\algoref{algo:search} is the pseudo-code for the algorithm to explore
all task schedules created at entry to permission regions and
\texttt{isolated}-constructs. The pseudo-code only covers the detection of data
races (i.e., permission region violations) or deadlocks; though, assertion violations and exceptions can be
detected similarly. The state of the program in the pseudo-code is
simplified for clarity; it is represented by a heap, $h$, and a set of
tasks, $T$. The lowercase $t$ indicates a task. \lineref{loc:run}
updates the heap and pool of tasks by running task $t$ until it
blocks, exits, reaches a permission region boundary (i.e., entry or
exit), or reaches an \texttt{isolated}-construct.

At the entry point of the permission region (PR\_ENTRY), \lineref{loc:acquire}
updates the state machine for the acquired permissions on the object in the heap and
checks to see if the acquisition signals a permission violation. At the exit
point of the permission region (PR\_EXIT), \lineref{loc:release} updates the
state machine for the released permissions on the object in the heap, and the algorithm
restarts task $t$ running anew at \lineref{loc:run}.

If there is a permission violation, then it is detected on
\lineref{loc:datarace}. Similarly, \lineref{loc:deadlock} detects a deadlock. A
deadlock state is indicated when there are no runnable tasks (i.e.,
$R = \emptyset$) and there exists tasks that are blocked. A report
for either a permission violation or a deadlock includes a witness trace for
validation and debugging. In the absence of a deadlock or a permission violation,
and when there are simply no more tasks to run, \lineref{loc:term}
terminates the search and reports any detected sharing that was not
annotated by a permission region or covered by an
\texttt{isolated}-construct. Such sharing is detected by tracking tasks on every heap access.

The set $S$ on \lineref{loc:visited} is a global set to track the
visited states. \lineref{loc:prsched} does the actual scheduling by
considering all runnable tasks, including the currently
running task $t$, as a next task to run. Note that in the current
state, if the task $t$ is preempted because it enters a permission
region, then that state reflects the acquired permissions on that
region. In the case that task $t$  blocked,
\lineref{loc:rand} chooses a random runnable task to schedule next.

\figref{fig:permission-violation-state}, shown previously, is the
state space explored by the search algorithm for the permission region
annotated version of the program in
\figref{fig:hj-async-finish}. Recall that the example has two tasks
that access the shared object {\tt stk}: one reading and the other writing. The
ovals in the diagram represent scheduling points, and as before, the
blocks on the left represent the state of the state machine tracking permissions. As
indicated by the pseudo-code, the algorithm only preempts running
tasks at the entrance to permission regions. In this example, it
schedules the child task after the main task acquires read permissions
to elicit the permission violation. By observation, if the permission regions in
the annotated program were replaced
with \texttt{isolated}-constructs, then the explored state space would
no longer include the violation, but it would include all schedules
that interleave the atomic blocks defined by the \texttt{isolated}-constructs.

\begin{algorithm}[t]
\caption{Procedure to Validate a Program}\label{algo:validate}
\begin{algorithmic}[l]
  \Procedure{validate}{$p$}
  \State ($h$, $T$) $:=$ init($p$)
  \State $R := $ runnable($T$)
  \State $t$ $:=$ random($R$)
  \State $S := \emptyset$
  \State search($t$, $h$, $T$)
  \While{uncovered sharing is reported}
  \State Add permissions or {\tt isolated} on sharing
  \State ($h$, $T$) $=$ init($p$)
  \State $S := \emptyset$
  \State search($t$, $h$, $T$)
  \EndWhile
  \EndProcedure
\end{algorithmic}
\end{algorithm}

\algoref{algo:validate} is a procedural flow describing the process of
program validation using the new search in
\algoref{algo:search}. When \algoref{algo:search}
finishes, the algorithm reports any heap locations that have been accessed by
more than one distinct task outside a permission region or an
\texttt{isolated}-construct with the input program location where that
access occurs. Using this information, a user is able to manually
annotate the program location appropriately, and then repeat the
search. The process terminates when a permission violation or a deadlock is
discovered, or no more sharing outside of permission regions or \texttt{isolated}-constructs exists.

\begin{theorem}
  \algoref{algo:search} is sound in that it only
  accepts programs that have no permission violations or deadlock on a given
  input under the restriction that the programs terminate and have all sharing correctly
  annotated with permission regions or wrapped in
  \texttt{isolated}-constructs.
\end{theorem}
\begin{proof}
The soundness proof reasons over a slightly modified version of the
algorithm that is iterative and takes as an additional input
a search tree, which is similar to \figref{fig:permission-violation-state}, that
captures all possible sequences of release and acquire statements
explored thus far. The algorithm traverses that input tree and at each
leaf node tries to extend that node by one generation if
possible. After the traversal, the algorithm returns the new tree. The
algorithm is called in an iterative manner until the tree reaches a
fix-point (which is guaranteed since the program terminates).

Let $P(n)$ be the statement that this modified search algorithm
returns all interesting sequences of acquire and release statements of
length $n$ or less for a given input program, where interesting means
containing a permission violation or deadlock.

\noindent\textbf{Basis Step:} the algorithm produces all interesting
sequences of length $n \leq 1$. This case is trivially established with
the initial state of the program that represents a sequence of length
$n \leq 1$ and cannot contain a permission violation or deadlock since the program has not yet
done anything. As such, it includes all interesting sequences.

\noindent\textbf{Inductive Step:} assume the modified algorithm has
correctly generated a tree representing all interesting sequences of $n$ or less;
it is necessary to show that from such a tree the algorithm is able to generate all
interesting sequences of length $n+1$ or less.  There are three possible outcomes at any
leaf of the input tree:
\begin{compactenum}
\item the leaf cannot be extended as it is already an interesting sequence having a permission violation or deadlock;
\item the leaf cannot be extended as there are no more tasks to run, in which case it is not interesting; or
\item the leaf is able to be extended with one or more immediate descendants.
\end{compactenum}
The first two cases are directly covered by \lineref{loc:datarace} through Line 24 of the
algorithm; there is no way to have any descendants in those
situations and the sequences are already classified as interesting or not.

For the third case, first consider \lineref{loc:entry:isolated} of the algorithm that creates
the next generation in the tree for permission regions and {\tt isolated}-constructs. Every runnable task is scheduled (\lineref{loc:prsched}) and each of those tasks
must reach an immediate successor. Such a successor may be a permission violation or a deadlock, making it an interesting sequence, a preemption, a
block condition, or exit by the constraint that the input program must
terminate. As such, any $n+1$ length sequence that exists, is
generated.

Further, any interesting $n+1$ sequence is generated. To see
this outcome, it is important to understand that the order of
acquisition relative to read or write does not matter in detecting a
violation. The state machines on the objects are not dependent on
acquisition order; they only depend on what tasks hold read or write permissions
at the time of acquisition. As the algorithm always first
acquires a permission and then schedules other tasks, it generates all
the interesting $n+1$ sequences if any exist. In this case, a sequence is interesting due to a permission violation. If a permission violation does not exist in a sequence, then a deadlock is
detected as usual.

To complete the inductive step, the code under Line 32 must be considered. That code covers
a blocked or exited task. The input program has all sharing
annotated or {\tt isolated} by the theorem statement, meaning that any non-determinism
due to scheduling is enumerated by \lineref{loc:prsched} so all reachable
program paths on the input are considered. If an interesting sequence exists
because of a deadlock, then it is either found in the $n+1$ step, by
having selected the correct task, or in a later step when the
correct task is chosen. If the deadlock depends on a particular sequence of
task executions, then those sequences are enumerated by \lineref{loc:prsched}. As
such, the deadlock is either deterministic (i.e., independent of the
schedule) or non-deterministic (i.e., a product of a data race on some
shared object). In the former, the choice of task does not matter,
and in the latter, \lineref{loc:prsched} enumerates all possible orders over permission region blocks and 
isolated blocks.
\end{proof}
As a side note, \algoref{algo:search} is complete when all regions
cover a single operation (i.e., an individual byte-code in the case of Java). Such completeness is at the cost of the number of explored schedules.
