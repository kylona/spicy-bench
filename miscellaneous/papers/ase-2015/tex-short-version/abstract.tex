Habanero is a task parallel programming model that provides correctness
guarantees to the programmer.  Even so, programs may contain data
races that lead to non-determinism, which complicates
debugging and verification. This paper presents a sound algorithm based on
permission regions to prove data race and deadlock freedom in
Habanero programs. Permission regions are user annotations to indicate
the use of shared variables over spans of code. The verification
algorithm restricts scheduling to permission region boundaries
and isolation to reduce verification cost.  The effectiveness of the algorithm is shown in
benchmarks with an implementation in the Java
Pathfinder (\jpf) model checker. The implementation uses a
verification specific library for Habanero that is tested using \jpf\ for correctness. The
results show significant reductions in cost, where cost is controlled
with the size of the permission regions, at the risk of rejecting
programs that are actually free of any data race or deadlock.
