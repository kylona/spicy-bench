\section{Conclusions \& Future Work}
This paper presents a model checking algorithm to prove when a
Habanero program does not contain any data races, deadlocks, assertion
violations, or exceptions for a given program input. The algorithm,
based on permission regions, only considers scheduling points in the
search tree at the boundaries of permission regions and {\tt
  isolated}-constructs. The paper includes a proof of soundness for
the algorithm, meaning that the algorithm may reject a correct program
due to the size of the permission regions.

The effectiveness of the algorithm is shown in several benchmark
programs that cover many of the Habanero concurrency constructs. The
analysis is done using a new Java library implementation of the Habanero
runtime that is intended for debugging and verification. The new algorithm, with permission regions,
is implemented as an extension to the \jpf\ model checker. The results
from the benchmark programs indicate a significant cost reduction when
using the new algorithm.

Future work includes automating the annotation of permission regions
based on the sharing detection in \jpf; automating the validation of any
counter-example; developing techniques to automatically refine
permission regions from counter-examples when needed; a partial order
reduction over permission regions; static-analysis to prevent
scheduling on regions that cannot race; applying symbolic techniques to reason over input; and studying
benchmarks that are representative of real world Habanero programs.

