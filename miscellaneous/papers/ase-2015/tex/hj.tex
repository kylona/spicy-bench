\section{Habanero Java}
\begin{comment}
\noindent\textbf{Nick} will write this section.

Tight overview of Habanero Java. Motivate the importance of the
language rather than try to introduce all of the language
features. Emphasize the guarantees on correctness in the absence of
data race. Contain to one-page plus figures.
\end{comment}

The necessity of parallelism for getting performance from software
is a rising trend in modern computing.
However, since the majority of programmers are not concurrency experts,
it becomes important to provide mainstream programmers with the tools
they need to safely add concurrency as they write new applications.
Habanero Java (HJ)---a language under development at Rice University---aims
to address this need by extending the Java language with high-level concurrency
constructs that have strong safety guarantees \cite{hj-overview}.
These new constructs are exposed through simple extensions to Java's syntax.
Alternatively, a pure-library implementation called HJ-lib is also available \cite{hj-lib}.
HJ-lib applications use Java 8's lambda expressions to succinctly leverage HJ concurrency constructs without the need for syntax extensions and a custom compiler.
HJV builds upon the HJ-lib API, allowing for easy integration with JPF.
\lstset{basicstyle=\scriptsize\ttfamily,caption={A naive method using HJ-lib for calculating the $n$th Fibonacci number, where $n \geq 1$.},label=lst:hj-fib}
\begin{lstlisting}
long fib(long n) {
  if (n < 3) return 1;
  HjFuture<Long> a = future(() -> fib(n-2));
  HjFuture<Long> b = future(() -> fib(n-1));
  return a.get() + b.get();
}
\end{lstlisting}


Listing~\ref{lst:hj-fib} shows a recursive definition of the Fibonacci function.
%The HJ \hjKwd{finish} construct ensures that all tasks created within its
%dynamic scope (including transitively created tasks) are complete before the
%\hjKwd{finish} statement completes.
The \hjKwd{future} construct creates an asynchronous task, and returns an
\hjKwd{HjFuture} as a placeholder for the task's return value.
In this case, \hjCode{a} contains the placeholder for \hjCode{fib(n-2)},
and \hjCode{b} for \hjCode{fib(n-1)}.
The calls to \hjCode{a.get()} and \hjCode{b.get()} block until the respective
tasks have completed, and return the resulting values.

HJ provides safety guarantees for subsets of its features\footnote{
See \cite{hj-overview} for a more complete discussion of HJ's features
and its continuum of safety guarantees.}.
For example, the subset of features used by the code in listing~\ref{lst:hj-fib}
guarantee that all calls to \hjCode{fib} are deadlock-free, yield deterministic
results, and have a serializable schedule---but only if the program is also free
of data races. The \hjCode{fib} method is obviously data-race-free because it
performs no mutations; however, for general programs that mutate shared variables,
proving data-race-freedom is usually much more difficult.

Due to the difficulty of statically-proving that a general HJ program is
data-race-free, \newTerm{gradual permission regions} were added to the HJ
language \cite{hj-grad-perm}.
Gradual permission regions allow a programmer to annotate shared objects with
read or write permissions.
The permission regions act as run-time assertions that throw a Java exception
whenever a data-race is detected on the annotated objects.
The HJ compiler can also automatically insert these annotations.
However, the compiler cannot distinguish shared from non-shared objects.
This results in extraneous annotations and unnecessary run-time overhead.

By leveraging existing features of JPF, HJV can be used to insert permission
regions for all shared objects in an HJ-lib application.
When an HJ-lib program is verified using HJV, it is proved deadlock-free and
race-free.
This is true even when using concurrency features that do not automatically
have these safety guarantees, such as HJ's phasers.

% What exactly is our contribution?
% There's already an algorithm for automatically inserting
% permission regions in Eddie's paper.
% Am I correct in saying that the static algorithm inserts
% too many permission regions (and thus adds unnecessary overhead),
% whereas this algorithm leverages JPF to dynamically check permissions
% of only the variables that are actually shared?
