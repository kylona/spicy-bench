\section{Gradual Permissions in JPF}
\begin{comment}
Define gradual permission regions with their mathematical implications
for correctness. The key being that if you define the regions to be
too big then you \st{miss data-race} limit parallelism;
if you define them too small, then you can miss a data race,
and you get state explosion in JPF.

\noindent \textbf{Critical}: we need to precisely define how the size of
permission regions affects the meaning of the verification result. If regions
are too big, then it seems like there would be false positives (i.e., a
data-race would be reported when one does not exist). Similarly, we have
examples where no data-race is found when a data race exists. Let's review the
paper and be able to state something definitive. 

Detail how the permission regions are implemented in JPF. Describe the process
of annotation.
\begin{compactitem}
\item Run JPF scheduling only on permission regions (initial program has no regions)
\item If there is sharing detected outside of a permission region, then annotate the permission region (limit to scope of variable)
\item Repeat
\end{compactitem}


JPF Implementation
\begin{compactitem}
\item Permission Violation detection is performed by a state machine. This state
machine tracks the permission status of the variable contained within the
annotation header. In JPF this is implemented using attribute info objects (meta
data that can be attached to any heap object).  
\item State machine is updated by listeners (observer pattern) on acquire and
release statements. Also, scheduling points are inserted into the state space by
these listeners at acquire statements.
\end{compactitem}
Contain to two pages plus figures.
\end{comment}

Permission regions can be seen as an extension to dynamic data race detection
tools. They increase the granularity of checks to sections of code instead of
individual memory accesses. Permission regions are a part of the language
semantics as opposed to dynamic data race detectors. This allows programmers to
control the granularity of checks without worrying about how the checks are
actually performed. Permission regions annotate read and write sets of variables
over a section of code. Two permission regions are said to be conflicting if
write set of one overlaps the read or write set of the other. If a permission
region starts executing when a conflicting permission region is already
executing in parallel an exception is thrown. Gradual permission regions allow
programmers to  iterartively add typing annotations to the programs to refine
the size of the permission regions.

In VR-lib,  Gradual Permission regions (GPRs) are delimited by acquireR/acquireW
(Object X) at the beginning and releaseR/ releaseW (Object X) at the end of the
region with R and W representing read and write resp. A call to
acquireR/acquireW asserts that it is safe to read/write the specified object in
the enclosing section of code until matching releaseR/releaseW is encountered.
If a conflicting GPR starts executing in parallel, an exception is thrown. Thus,
the verification problem becomes one of ensuring that no acquireW/ acquireR
calls result in a permission exception. GPRs are implemented in VR-Lib using
JPF. GPRs utilize JPF's ability to perform dynamic checks for all legal
schedules for a given input. 

\textbf{GPR Implementation using JPF:} VR-Lib uses a combination of custom
listener, a scheduler factory and attribute info for permission regions
verification using JPF. The components are discussed briefly below:
\begin{enumerate} 
\item HjListener: HjListener extends the JPF's property
adapter class. Consequently it inherits the behavior of the property adapter
class, but with two modifications: 1) executeInstruction and 2)
instructionExecuted. In JPF, these two methods are used to observe/modify the
system state before an instruction has been executed (executeInstruction) or
after the instruction has been executed (instructionExecuted). In both cases,
the method observes whether the instruction is a permission region relevant
instruction (acquire/release). In the former case, HjListener will insert a
choice generator with all runnable threads enabled and in the latter case it
will perform appropriate permission checks on all read/write accesses performed
in the permission region on the annotated variable.
\item HjScheduler: HjScheduler mimics the behavior of JPF's default schedule
with one exception: choice generators are not inserted when threads access
shared variables. Currently, JPF inserts shared access choice generators
dynamically as it discovers sharing. JPF relies upon the placement of these
choice generators to correctly detect race conditions. When a shared variable is
discovered, JPF will insert a choice generator at the shared variable's
location. This choice generator will cause JPF to interleave the execution of
all runnable threads. If JPF (more precisely, JPF's PreciseRaceDetector) detects
simultaneous read/write access of the shared variable it will notify the user of
a data race. However, if one of the conflicting threads is no longer runnable
(terminated, blocked, etc) then JPF will not report the error. In other words,
JPF's partial-order reduction is incomplete. In contrast, when using GPRs,
programmer annotations dictate the location of the inserted choice generators. A
gradual type system can be used to reduce the annotation burden for programmers.
\item HjAttribute: HjAttribute is a series of classes that handle the actual
permission violation checks. Each object that is guarded by a permission region
is assigned an attribute info. This attribute info is an object-specific state
machine. This machine defines four valid states: Null, Private-Read,
Private-Write, and Shared-Read. See Figure{} Null is the state of an object for
which no task has acquired permissions. An object which has been acquired by a
single task for read/write permissions has state Private-Read/Private-Write
respectively. Shared-Read is the state of an object when multiple tasks
concurrently acquire read permissions.
\end{enumerate}
When HjListener detects that an acquire/release statement has been executed on
object X, it signals a state transition on the state machine belonging to X. If
a legal transition is not defined then a runtime exception is thrown. Upon
backtracking, JPF performs a shallow copy on attribute infos. This means that
the object itself will be restored, but none of the underlying changes to the
object. HjAttribute was designed to be entirely immutable (transitions from
state X to Y are simulated by replacing the machine in state X with a new
machine in state Y); therefore, when JPF backtracks, all of the attribute infos
are properly restored.  A walk-through of this system for a typical data race
will highlight the interplay among different pieces of the system. Figure{}
shows an example of concurrent accesses on a common data structure (X). Since
two threads have conflicting permission requirements that can occur in parallel,
we expect that a permission violation should be detected.
The first task to begin execution will request permission acquisition on object
X. HjListener will detect that an acquisition is pending and will insert a
choice generator at the current location in the schedule. HjListener will also
attach a state machine to X as seen in Figure ??. If task A is the first task to
run then the state machine will be set to the private write state. JPF may now
choose to either execute task B or finish the execution of task A. We will
consider the case where task B is chosen. Task B will then request permission
acquisition on X. HjListener will insert another choice generator and then will
examine the state machine attached to X. There is no transition defined from the
private write state to the private read state or vice-versa, thus an exception
will be thrown. This exception is then reported by JPF as a permission
violation. Similarly, if Task B executed first, then we’d get an error when task
A tried to assert write permission.



