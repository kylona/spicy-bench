\section{Introduction}
The Habanero effort at RICE University is intended to address the
multicore challenge \cite{habanero}.
%
It provides a powerful set of task parallel programming constructs that are added as
simple extensions to existing languages such as Java or C/C++.
%
Unique to the Habanero approach is its emphasis on usability and safety
in the parallel constructs and supporting runtime systems.
%
Habanero guarantees properties such as deadlock freedom or determinism over subsets of its language.
%
Such an approach may be able to open the promised performance of
multicore architectures to programmers who are not experts in
concurrency.

%
Habanero-Java (HJ) is the implementation of the Habanero paradigm in
the Java language complete with compiler and runtime support \cite{Cave:2011:HNA:2093157.2093165}. 
%
HJ adds to Java new keywords and constructs for task parallel
programming: \texttt{async} to create tasks, \texttt{finish} to join
tasks, \texttt{isolated} to give mutual exclusion between tasks, and a
\texttt{phaser} construct with the \texttt{next} keyword for software
pipelines and data-flow computation. 
%
Habanero, as mentioned previously, guarantees safety properties over subsets of its language. 
%
As an example for HJ, programs that only employ the \texttt{async} and
\texttt{finish} keywords are guaranteed to be deadlock-free,
serializable, and deterministic if and only if the program is
data-race free.
%
The goal of the work in this paper is to verify that Habanero Java programs are data-race free. 

Safety guarantees in HJ are predicated on data-race free execution. 
%
The current HJ distribution provides a tool to detect data-race as a
program is executed in the HJ runtime system;
%
thus, the tool only
validates a single schedule of the program execution and is only able to
tell the developer if a data-race occurred in that execution and not that the program is
data-race free \cite{Westbrook:2012:PPR:2367163.2367201,Westbrook:2011:PRR:2341616.2341627}. 
%
Further, the tool requires both compiler and runtime
support to annotate the source HJ program and track extra data
structures.

This paper presents a new verification runtime (VR) system for HJ that
is suitable to use in Java Pathfinder (JPF) to prove an HJ program is
data-race free in all legal schedules.
%
The VR system supports \texttt{async}, \texttt{finish},
\texttt{isolated}, and future data-driven forms of \texttt{async} and
is able to run standalone as a new runtime system for HJ.
%
The approach in this paper differs from recent efforts for X10 in
that it creates a new HJ runtime for JPF rather than modify an
existing HJ runtime, modify JPF to work with that modified runtime, and then
modify the HJ compiler to better meet JPF's unique needs \cite{conf:icst:GligoricMM12}.
% 
Further, the VR system is able to exploit
nuances in JPF's virtual machine, such as low overhead in thread
creation, to implement a runtime system that is small enough to manually
inspect for correctness as it only comprises 992 locations in 12 different classes. 

This paper further presents a new scheduler factory for JPF that is optimized for HJ program semantics. 
%
In particular, the new scheduler factory removes choice generators
from schedules from the normal JPF factory that are not
interesting. These uninteresting points include thread termination, entering a monitor, thread notification, etc.
% 
The new scheduler factory yields a significant reduction in states when compared to the default scheduler. 
To summarize, the main contributions in this paper are:
\begin{compactitem}
\item a new runtime system for HJ small enough to understand through manual inspection and use in JPF; 
\item a new scheduler factory for JPF; and
\item results from several benchmark examples showing performance over the HJ the default scheduler.
\end{compactitem}

\begin{comment}
The rapid shift to multicore platforms is leading to a programming
crisis. 
%
A typical programmer is not able to exploit the additional
computation resources because there is no obvious good programming paradigm for concurrency.
%
Languages are either exceptionally low level requiring an expert in concurrency or are too abstract to perform adequately on the hardware.
%
Bugs in concurrent programs are also especially pernicious and hard to reproduce.
%
Unintended (and unexpected) interactions between concurrent elements are usually dependent on external runtime systems outside the control of the developer.

As a result, it is inherently difficult (or impossible) to obtain any
guarantee on test coverage, and if a test fails, it may fail
non-deterministically making debugging equally impossible.
\end{comment}
